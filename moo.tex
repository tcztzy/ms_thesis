\chapter{模型校正及验证}\label{chap:moo}

\section{研究方法}

通过计算每个指标的相对均方根误差 (NRMSE),将 12 个农业生态系统指标的测量和模拟数据 (表 \ref{tab:metrics}) 汇总到12个模拟情景 (2019-2020 两年每年 6 个不同的灌水处理) 中。%
这 12 个指标按优先顺序列出,以便使用多目标优化算法对 Cotton2K 农业生态系统模型进行评价。

\begin{table}
    \caption{12 种棉花的农业生态系统指标}\label{tab:metrics}
    \centering
    \begin{tabular}{llll}
        \toprule
        指标 & 描述           & 单位                   & 数量 \\
        \midrule
        LY   & 皮棉产量       & $\mathrm{kg\ ha^{-1}}$ & 24   \\
        TAGB & 地上总干物质量 & $\mathrm{kg\ ha^{-1}}$ & 240  \\
        LAI  & 叶面积指数     & $\mathrm{m^2\ m^{-2}}$ & 240  \\
        LI   & 光能截获率     & \%                     & 240  \\
        SCY  & 籽棉产量       & $\mathrm{m^2\ m^{-2}}$ & 24   \\
        PH   & 株高           & cm                     & 240  \\
        MSN  & 主茎节点数     & 个                     & 240  \\
        LDM  & 叶干物质量     & $\mathrm{m^2\ m^{-2}}$ & 240  \\
        PDM  & 叶柄干物质量   & $\mathrm{m^2\ m^{-2}}$ & 240  \\
        SDM  & 茎干物质量     & $\mathrm{m^2\ m^{-2}}$ & 240  \\
        BLN  & 铃数           & 个                     & 240  \\
        BDM  & 铃干物质量     & $\mathrm{m^2\ m^{-2}}$ & 240  \\
        \bottomrule
    \end{tabular}
\end{table}

\begin{equation}\label{eq:NRMSE}
    RMSE = \left \{ \frac{\sum_{i=1}^n (O_i - \overline{O}) (P_i - \overline{P})}{\sqrt{\sum_{i=1}^n (O_i - \overline{O})^2} \sqrt{\sum_{i=1}^n (P_i - \overline{P})^2} \overline{O}} \right \}^2
\end{equation}
式中 $NRMSE$ 为相对均方根误差, $O_i$ 为第 $i$ 个观测值, $\overline{O}$ 为观测值的平均值, $P_i$ 为第 $i$ 个预测值, $\overline{P}$ 为预测值的平均值, $n$ 为观测值和预测值的个数。

\subsection{Python 中的多目标优化}
本章采用 pymoo 框架提供的非支配排序遗传算法 (NSGA-III) 多目标优化算法确定参数最优值。该框架由美国密歇根州立大学的 Kalaynmoy Deb 教授带领的计算优化与创新实验室 (COIN) 的 Julian Blank 开发维护的。%
在大多数情况下,pymoo 框架被用于解决连续问题, 但其他变量类型也可使用。%
pymoo 框架中的算法 (如NSGA-III算法) 是模块化组织的,用户可以通过编写代码自定义采样、交叉和突变等模块,自行组装符合自身需求的算法。%

\subsection{参数优化步骤}
根据第 \ref{chap:sa} 章模型敏感性参数分析结果,选择 35 个对模拟结果影响较大的参数。
然后再次使用 Sobol' 方法选择 Cotton2K 模型参数集,但仅在模型敏感性参数分析出的影响较大的参数中进行,第二次选择的参数集没有进行敏感性分析。%
相反,第二次 Monte Carlo 方法采样后的模拟结果被用于模型参数优化,以识别在测量结果和模拟结 果之间提供最佳一致性的参数集。%
具体步骤如下:
\begin{enumerate}
    \item 选择第 \ref{chap:sa} 章对模型结果影响较大的 35 个品种,确定各参数的取值范围和均匀分布形式。
    \item 采用 Python 语言编写程序,生成 Cotton2K 的输入参数,以及模型输出的损失函数。
    \item 运行 NSGA-III 算法调用 Cotton2K 模型,并将结果导入到数据库。
    \item 采用剪枝算法对 Pareto 解集的结果进行剪枝。
    \item 根据剪枝后的 Pareto 解集确定最终模型敏感性参数最优值。
\end{enumerate}

\section{参数校正结果}
本研究采用 NSGA-III 算法,设置初始种群数量为 200,迭代 200 代,其他算法参数采用默认值。对 Pareto 解集采用剪枝算法,最终得到结果如表 \ref{tab:parameters} 所示。
\begin{longtable}{llrrcrr}
    \caption{Cotton2K 参数列表及率定值}\label{tab:parameters}                         \\
    \toprule
    参数     & 功能描述                   & 下限   & 上限   & GSA & 原版    & 改版    \\
    \midrule\endfirsthead
    \caption*{续表\ref{tab:parameters}}                                               \\
    \toprule
    参数     & 功能描述                   & 下限   & 上限   & GSA & 原版    & 改版    \\
    \midrule
    \endhead
    \bottomrule
    \multicolumn{7}{r}{\textit{接下页}}                                               \\
    \endfoot
    \bottomrule
    \endlastfoot
    VARPAR01 & 种植密度对生长的影响       & 0.00   & 0.08   & *   & 0.04344 & 0.03416 \\
    VARPAR02 & 现蕾前的叶生长             & 0.00   & 0.60   &     & 0.3     & 0.3     \\
    VARPAR03 & 现蕾前的叶生长             & 0.00   & 0.10   &     & 0.014   & 0.014   \\
    VARPAR04 & 现蕾前的叶生长             & 0.00   & 1.00   & *   & 0.138   & 0.646   \\
    VARPAR05 & 主茎叶生长                 & 0.50   & 3.00   &     & 1.6     & 1.6     \\
    VARPAR06 & 主茎叶生长                 & 0.00   & 0.02   &     & 0.010   & 0.010   \\
    VARPAR07 & 主茎叶生长                 & 18.0   & 28.0   &     & 24.0    & 24.0    \\
    VARPAR08 & 果汁页生长                 & 0.00   & 0.20   &     & 0.10    & 0.10    \\
    VARPAR09 & 铃生长周期                 & 18.9   & 38.0   &     & 28.0    & 28.0    \\
    VARPAR10 & 铃生长速率                 & 0.10   & 0.45   &     & 0.3293  & 0.3293  \\
    VARPAR11 & 最大铃干物质               & 3.00   & 15.0   &     & 8.8     & 8.8     \\
    VARPAR12 & 方铃展前茎生长             & 0.00   & 2.00   & *   & 0.284   & 0.590   \\
    VARPAR13 & 方铃展前茎生长             & 0.00   & 2.00   &     & 0.040   & 0.040   \\
    VARPAR14 & 方铃展前茎生长             & 0.00   & 0.08   &     & 0.014   & 0.014   \\
    VARPAR15 & 方铃展后茎生长             & 1.00   & 4.00   & *   & 3.971   & 2.001   \\
    VARPAR16 & 方铃展后茎生长             & 0.50   & 2.00   &     & 2.4     & 2.4     \\
    VARPAR17 & 方铃展后茎生长             & 0.00   & 1.00   &     & 0.10    & 0.10    \\
    VARPAR18 & 方铃展后茎生长             & 0.00   & 0.20   &     & 0.140   & 0.140   \\
    VARPAR19 & 株高生长                   & 0.00   & 0.50   &     & 0.20    & 0.20    \\
    VARPAR20 & 株高生长                   & 0.00   & 0.10   &     & 0.02    & 0.02    \\
    VARPAR21 & 株高生长                   & 10.0   & 18.0   & *   & 14.541  & 13.868  \\
    VARPAR22 & 株高生长                   & -4.00  & -2.00  & *   & -2.686  & -2.077  \\
    VARPAR23 & 株高生长                   & 0.09   & 0.12   &     & 0.10    & 0.10    \\
    VARPAR24 & 株高生长                   & 0.00   & 0.50   &     & 0.175   & 0.175   \\
    VARPAR25 & 株高生长                   & 1.00   & 4.00   &     & 2.20    & 2.20    \\
    VARPAR26 & 株高生长                   & 0.70   & 1.30   & *   & 1.2500  & 0.7500  \\
    VARPAR27 & 碳胁迫下的主茎节点延迟     & 0.50   & 1.00   & *   & 0.731   & 0.743   \\
    VARPAR28 & 碳胁迫下的坐果节点延迟     & 1.00   & 3.00   &     & 2.15    & 2.15    \\
    VARPAR29 & 碳胁迫下的坐果节点延迟     & 1.00   & 3.00   &     & 1.36    & 1.36    \\
    VARPAR30 & 温度对方铃形成的影响       & 0.70   & 1.30   & *   & 1.281   & 1.249   \\
    VARPAR31 & 现蕾前节点的发育           & 1.00   & 5.00   & *   & 3.210   & 2.175   \\
    VARPAR32 & 现蕾前节点的发育           & 1.00   & 3.00   & *   & 1.625   & 1.664   \\
    VARPAR33 & 现蕾前节点的发育           & 0.50   & 1.00   &     & 0.80    & 0.80    \\
    VARPAR34 & 初始叶面积                 & 0.01   & 0.10   & *   & 0.0407  & 0.0822  \\
    VARPAR35 & 果枝发育                   & -32.0  & -26.0  & *   & -26.118 & -27.108 \\
    VARPAR36 & 坐果节点的发育             & -57.0  & -49.0  &     & -54.00  & -54.00  \\
    VARPAR37 & 坐果节点的发育             & 0.00   & 2.00   &     & 0.80    & 0.80    \\
    VARPAR38 & 落叶剂对不同叶龄叶片的影响 & 0.00   & 5.00   &     & 3.20    & 3.20    \\
    VARPAR39 & 温度对吐絮的影响           & -306.0 & -240.0 &     & -292.0  & -292.0  \\
    VARPAR40 & 温度对吐絮的影响           & 0.70   & 1.30   &     & 1.08    & 1.08    \\
    VARPAR41 & 温度对衣份率的影响         & 45.0   & 60.0   & *   & 51.813  & 47.436  \\
    VARPAR42 & 温度对衣份率的影响         & 0.30   & 0.90   & *   & 0.844   & 0.341   \\
    VARPAR43 & 碳胁迫下的脱落强度         & 0.30   & 0.70   & *   & 0.514   & 0.601   \\
    VARPAR44 & 水分胁迫下的脱落强度       & 0.30   & 0.70   &     & 0.48    & 0.48    \\
    VARPAR45 & 方铃脱落的概率             & 0.10   & 0.50   &     & 0.24    & 0.24    \\
    VARPAR46 & 方铃脱落的概率             & 0.00   & 0.15   &     & 0.08    & 0.08    \\
    VARPAR47 & 铃脱落的概率               & 1.00   & 10.0   & *   & 3.833   & 6.439   \\
    VARPAR48 & 铃脱落的概率               & 0.30   & 1.50   & *   & 0.532   & 0.554   \\
    VARPAR49 & 铃脱落的概率               & 5.00   & 15.0   & *   & 7.714   & 6.867   \\
    VARPAR50 & 铃脱落的概率               & 0.20   & 1.80   & *   & 1.610   & 1.216   \\
    LIPAR01  & 光能截获参数               & -1.60  & 0.00   & *   & -       & -3.347  \\
    LIPAR02  & 光能截获参数               & -1.60  & 0.00   & *   & -       & -2.199  \\
    LIPAR03  & 光能截获参数               & -1.60  & 0.00   & *   & -       & -2.788  \\
    LIPAR04  & 光能截获参数               & -1.60  & 0.00   & *   & -       & -1.262  \\
    LIPAR05  & 光能截获参数               & -1.60  & 0.00   & *   & -       & -0.253  \\
    LIPAR06  & 光能截获参数               & -1.60  & 0.00   & *   & -       & -1.602  \\
    LIPAR07  & 光能截获参数               & -1.60  & 0.00   & *   & -       & -1.251  \\
    LIPAR08  & 光能截获参数               & -1.60  & 0.00   & *   & -       & -3.464  \\
    LIPAR09  & 光能截获参数               & -1.60  & 0.00   & *   & -       & -0.207  \\
    LIPAR10  & 光能截获参数               & -1.60  & 0.00   & *   & -       & -0.776  \\
    LIPAR11  & 光能截获参数               & -1.60  & 0.00   & *   & -       & -0.758  \\
    LIPAR12  & 光能截获参数               & -1.60  & 0.00   & *   & -       & -1.110  \\
    LIPAR13  & 光能截获参数               & -1.60  & 0.00   & *   & -       & -2.517  \\
    LIPAR14  & 光能截获参数               & -1.60  & 0.00   & *   & -       & -1.871  \\
    LIPAR15  & 光能截获参数               & -1.60  & 0.00   & *   & -       & -2.984  \\
    LIPAR16  & 光能截获参数               & -1.60  & 0.00   & *   & -       & -0.166  \\
    LIPAR17  & 光能截获参数               & -1.60  & 0.00   & *   & -       & -3.303  \\
    LIPAR18  & 光能截获参数               & -1.60  & 0.00   & *   & -       & -1.743  \\
    LIPAR19  & 光能截获参数               & -1.60  & 0.00   & *   & -       & -1.504  \\
    LIPAR20  & 光能截获参数               & -1.60  & 0.00   & *   & -       & -2.155  \\
\end{longtable}

\section{模型验证和评估}

本研究选取 2019 年棉花实验数据进行参数优化,2020 年试验数据用于进行模型的验证和评估。%
采用多种统计方法作为验证和评价指标来评价模型校正和验证结果的可靠性,包括绝对相对误差 ARE、相对均方根误差 NRMSE。%
NRMSE 的值越大,模拟值与实测值偏离越大,两者的一致性越好,模型的结果越准确可靠。


\subsection{生育期日期模拟}
由表 \ref{tab:lifecycle} 发现,原版模型对于首花日、首絮日日期,较为准确,误差仅为 0-4d,但首蕾日与60\% 吐絮日的误差较大。%
2019 年首蕾日误差较大,这是由于模型对播种后连续的低温和降雨导致的花期延迟过于敏感,%
60\% 吐絮日误差较大,则可能由于模型对吐絮的模拟参数有待调整。%
这一点在改版后的模型的表现一致得以验证。
对于改版后的模型,在 2019 年的模拟中,首蕾日与首花日分布延迟16d和15d,而首絮日和60\% 吐絮日误差比原版模型更好。%
2020 年的结果与之类似。综合来说,改版后的模型在生育日期的模拟精度上与原版模型没有显著改进。

\begin{table}
    \caption{各生育期日期观测与模拟预测结果}\label{tab:lifecycle}
    \begin{tabular}{cccccc}
        \toprule
        年份                  & 日期        & 原版模拟值 & 改版模拟值 & 观测值  & 误差(原版/改版) (d) \\
        \midrule
        \multirow{4}{*}{2019} & 首蕾日      & 6月24日    & 6月23日    & 6月7日  & 17/16               \\
                              & 首花日      & 7月6日     & 7月21日    & 7月6日  & 0/15                \\
                              & 首絮日      & 8月30日    & 8月26日    & 8月26日 & 4/0                 \\
                              & 60\% 吐絮日 & 9月20日    & 9月10日    & 9月2日  & 18/8                \\
        \hline
        \multirow{4}{*}{2020} & 首蕾日      & 6月4日     & 6月8日     & 6月5日  & -1/3                \\
                              & 首花日      & 7月3日     & 7月16日    & 7月5日  & -2/11               \\
                              & 首絮日      & 8月30日    & 8月29日    & 8月27日 & 3/2                 \\
                              & 60\% 吐絮日 & 9月17日    & 9月5日     & 9月3日  & 14/2                \\
        \bottomrule
    \end{tabular}
\end{table}

\subsection{主要生理指标模拟}

NSGA-III 算法进行了 200 代,为每种模型确定了 22 个元素的非支配解集。
在每一代的结果中,原始和修改后的方法最优的解决方案被选中。

在修改后的模型中,叶重、LAI和叶柄重量在生长季节的后期有所下降。
作为对比,原始模型没有这种下降。
在 Cotton2K 中,嫩叶从茎部生长,这意味着当叶子出现时,茎部重量将减去叶子的重量。
在原始模型的模拟中,茎的重量为负值,因为叶片的生长速度被明显高估了。
而且没有对正数重量进行验证。
当测量的 LAI $\ge 1.5$ 时,模拟的LAI大多被高估,而当 LAI 小于 1.5 时,则被低估。
当使用原版模型时,当 LAI 小于 1.5 时,模拟的LAI大多被高估,这样的行为与 \authoryearcite{thorp2019} 报告相似。
改版之后的模型模拟中,LAI 的 NRMSE从65.88\% 下降到31.99\%,在大多数情况下略微低估了。
大多数情况下,当 LAI 大于 3 时,分散性显示出增加的迹象。
因此,该模型在充分灌溉条件下对植物生长条件的响应能力较弱,正如它所声称的那样
该模型侧重于干旱地区和亏损灌溉策略。
尽管修改后的模型模拟的结果较差,但模拟的SCY的准确度是中等的。
总的来说,用Cotton2K模型模拟的LAI和SCY比用附近的差分模型处理的要差。

原有模型模拟的光能截获率的值系统地偏高,在测量值只有 0.6 左右的情况下因为超过 1 而被截断。
修改后的模型模拟中的光能截获率表现更好,NRMSE从 29.78\% 下降到21.90\%。
这充分表明对冠层分层模拟的有效性。

修改后的模型所模拟的 SCY 在所有的模拟情况下都被低估了,
这使得 NRMSE 大于原始模型的模拟结果。
这可能是由于模拟中晚期灌溉减少而导致的过度的棉铃脱落。

\begin{figure}
    \centering
    \includegraphics[scale=0.5]{1v1.png}
    \caption{主要生理指标原版和改版 Cotton2K 的模拟值与实测值对比}
\end{figure}

\begin{figure}
    \centering
    \includegraphics[scale=0.5]{time-series.png}
    \caption{以 2020 年 W5 处理为例,12 个农业生态系统指标的时序变化}
\end{figure}
\section{修改前后的 Cotton2K 模型对比}

根据第 \ref{chap:moo} 章的结果,选取修改前后的 Cotton2K 模型的模拟结果中选择的 Pareto 最优解决方案,%
对 2019 年与 2020 年两年各六个水分处理的共 12 个试验场景重新运行模拟,将模拟结果中的 12 个农业生态指标与实测田间数据进行对比。%
这 12 个指标包括:LY, AGM, LAI, LI, SCY, PH, MSN, LDM, PDM, SDM, BLN, BDM。%
结果如表 \ref{tab:stats},每个指标的统计学上的更优的方法以加粗突出显示。

\begin{table}
    \caption{修改前后的 Cotton2K 模型模拟结果方差分析结果}\label{tab:stats}
    \centering
    \begin{tabular}{lrlrr}
        \toprule
        指标 & F       & p 值      & 原版            & 改版            \\
        \midrule
        LY   & 4.8627  & 0.0382*   & 14.12           & \textbf{5.95}   \\
        SCY  & 6.2853  & 0.0201*   & 77.99           & \textbf{68.83}  \\
        AGM  & 1.1180  & 0.2925    & 62.28           & \textbf{44.84}  \\
        LAI  & 4.4933  & 0.0367*   & 29.90           & \textbf{28.92}  \\
        LI   & 1.8028  & 0.1831    & 33.70           & \textbf{24.72}  \\
        PH   & 21.9813 & 0.0000*** & 40.67           & \textbf{32.84}  \\
        MSN  & 4.6529  & 0.0330*   & 41.89           & \textbf{31.12}  \\
        LDM  & 24.4956 & 0.0000*** & 146.45          & \textbf{53.68}  \\
        PDM  & 35.8514 & 0.0000*** & 220.96          & \textbf{95.23}  \\
        SDM  & 1.4610  & 0.2292    & \textbf{132.93} & 212.96          \\
        BLN  & 0.0053  & 0.9422    & 71.44           & \textbf{62.46}  \\
        BDM  & 0.5421  & 0.4630    & 158.35          & \textbf{121.90} \\
        \bottomrule
    \end{tabular}
\end{table}

从表 \ref{tab:stats} 可以看出,除了干物质(SDM)外,修改后的模型在其他指标比较中的 NRMSE 都较小。%
SDM 的不理想可能是由于修改后的模型修改了叶片与茎的生长过程。%

\section{不同灌水处理下得叶面积时空分布}

使用修改后的 Cotton2K 模型依据对 2019 年的田间实验数据对不同灌水处理分别运行模拟,模拟的叶面积指数的时空分布如图 \ref{fig:laiDist} 所示。

\begin{figure}
    \centering
    \includegraphics[scale=0.6]{lai_dist_2019.png}
    \caption{不同水分处理下得叶面积指数时空分布}\label{fig:laiDist}
\end{figure}

可以看出,总体上,随着灌水定额的增加,叶面积指数在空间上,各层级分布更多,分布的层级更高;时间上,叶面积指数增长的时间更早,减少的时间更晚。%
多数叶面积指数集中在 10 cm 至 15 cm 的区间,出苗后 85 天至 120 天之间。%
叶面积指数的分布呈现斜置的纺锤状结构,初期、底层的叶面积指数和后期、高层的叶面积指数低,中期、中层的叶面积指数高。%
这说明在现蕾前的叶片被上层叶片遮蔽后逐渐失能脱落,后期上层增长的叶面积较少,这可能是由于新疆普遍实行的打顶操作引起的。%
叶面积指数在充分灌水的条件下会比不充分灌水的条件下有更广泛的时空分布。%
这说明提高灌水可以有效增加冠层的光能利用效率,从而提高光合同化产量。%
但灌水的增加也会增加营养生长的比例,试验中,最高灌水的 W6 处理的籽棉产量就低于 W5 灌水处理。
注意到叶面积指数在垂直方向上的分布存在断层,这可能是由两个原因导致的:\begin{enumerate*}
    \item 模型中在现蕾之后的株高增长过快或者
    \item 分配叶面积到各层的方法有待改进
\end{enumerate*}。