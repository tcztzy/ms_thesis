\chapter{模型校正及验证}\label{chap:moo}

\section{研究方法}
\begin{table}
    \caption{对中国新疆阿拉尔市田间12种棉花管理方案的农业生态系统指标进行评估。
        这12个指标按优先顺序列出,以便使用多目标优化技术对Cotton2K农业生态系统模型进行评价。}
    \centering
    \begin{tabular}{llll}
        \toprule
        指标 & 描述           & 单位                   & 数量 \\
        \midrule
        LY   & 皮棉产量       & $\mathrm{kg\ ha^{-1}}$ & 24   \\
        TAGB & 地上总干物质量 & $\mathrm{kg\ ha^{-1}}$ & 240  \\
        LAI  & 叶面积指数     & $\mathrm{m^2\ m^{-2}}$ & 240  \\
        LI   & 光能截获率     & \%                     & 240  \\
        SCY  & 籽棉产量       & $\mathrm{m^2\ m^{-2}}$ & 24   \\
        PH   & 株高           & cm                     & 240  \\
        MSN  & 主茎节点数     & 个                     & 240  \\
        LDM  & 叶干物质量     & $\mathrm{m^2\ m^{-2}}$ & 240  \\
        PDM  & 叶柄干物质量   & $\mathrm{m^2\ m^{-2}}$ & 240  \\
        SDM  & 茎干物质量     & $\mathrm{m^2\ m^{-2}}$ & 240  \\
        BLN  & 铃数           & 个                     & 240  \\
        BDM  & 铃干物质量     & $\mathrm{m^2\ m^{-2}}$ & 240  \\
        \bottomrule
    \end{tabular}
\end{table}

\begin{table}
    \caption{实测值与模拟值对比统计参数计算公式及描述}
    \begin{tabular}{p{0.15\linewidth}p{0.7\linewidth}p{0.15\linewidth}}
        \toprule
        参数     & 计算公式                                                                                                                                                                      & 描述         \\
        \midrule
        $b$      & \[\frac{\sum_{i=1}^n (O_i - \overline{O}) (P_i - \overline{P})}{\sum_{i=1}^n (O_i - \overline{O})}\]                                                                          & 回归系数     \\
        $R^2$    & \[\left \{ \frac{\sum_{i=1}^n (O_i - \overline{O}) (P_i - \overline{P})}{\sqrt{\sum_{i=1}^n (O_i - \overline{O})^2} \sqrt{\sum_{i=1}^n (P_i - \overline{P})^2}} \right \}^2\] & 决定系数     \\
        RMSE     & \[\sqrt{\frac{\sum_{i=1}^n (O_i - P_i)^2}{n}}\]                                                                                                                               & 均方根误差   \\
        AAE      & \[\frac{\sum_{i=1}^n |O_i - P_i|}{n}\]                                                                                                                                        & 平均绝对误差 \\
        EF       & \[1 - \frac{\sum_{i=1}^n (O_i - P_i)^2}{\sum_{i=1}^n (O_i - \overline{O})^2}\]                                                                                                & 模型有效性   \\
        $d_{IA}$ & \[1 - \frac{\sum_{i=1}^n (O_i - P_i)^2}{\sum_{i=1}^n (|P_i - \overline{O}| + |O_i - \overline{O}|^2)}\]                                                                       & 拟合度       \\
        \bottomrule
    \end{tabular}
\end{table}
\subsection{Python 中的多目标优化}
本章采用 pymoo 框架提供的非支配排序遗传算法 (NSGA-III) 多目标优化算法确定参数最优值。该框架由美国密歇根州立大学的 Kalaynmoy Deb 教授带领的计算优化与创新实验室 (COIN) 的 Julian Blank 开发维护的。%
在大多数情况下,pymoo 框架被用于解决连续问题, 但其他变量类型也可使用。%
pymoo 框架中的算法 (如NSGA-III算法) 是模块化组织的,用户可以通过编写代码自定义采样、交叉和突变等模块,自行组装符合自身需求的算法。%

\subsection{参数优化步骤}
根据第 \ref{chap:sa} 章模型敏感性参数分析结果,选择 35 个对模拟结果影响较大的参数。
然后再次使用 Sobol' 方法选择 Cotton2K 模型参数集,但仅在模型敏感性参数分析出的影响较大的参数中进行,第二次选择的参数集没有进行敏感性分析。%
相反,第二次 Monte Carlo 方法采样后的模拟结果被用于模型参数优化,以识别在测量结果和模拟结 果之间提供最佳一致性的参数集。%
具体步骤如下:
\begin{enumerate}
    \item 选择第 \ref{chap:sa} 章对模型结果影响较大的 35 个品种,确定各参数的取值范围和均匀分布形式。
    \item 采用 Python 语言编写程序,生成 Cotton2K 的输入参数,以及模型输出的损失函数。
    \item 运行 NSGA-III 算法调用 Cotton2K 模型,并将结果导入到数据库。
    \item 采用剪枝算法对 Pareto 解集的结果进行剪枝。
    \item 根据剪枝后的 Pareto 解集确定最终模型敏感性参数最优值。
\end{enumerate}
\begin{longtable}{llrrcrr}
    \caption{Cotton2K 参数列表及率定值}\label{tab:parameters}                         \\
    \toprule
    参数     & 功能描述                   & 下限   & 上限   & GSA & 原版    & 改版    \\
    \midrule\endfirsthead
    \caption*{续表\ref{tab:parameters}}                                               \\
    \toprule
    参数     & 功能描述                   & 下限   & 上限   & GSA & 原版    & 改版    \\
    \midrule
    \endhead
    \bottomrule
    \multicolumn{7}{r}{\textit{接下页}}                                               \\
    \endfoot
    \bottomrule
    \endlastfoot
    VARPAR01 & 种植密度对生长的影响       & 0.00   & 0.08   & *   & 0.04344 & 0.03416 \\
    VARPAR02 & 现蕾前的叶生长             & 0.00   & 0.60   &     & 0.3     & 0.3     \\
    VARPAR03 & 现蕾前的叶生长             & 0.00   & 0.10   &     & 0.014   & 0.014   \\
    VARPAR04 & 现蕾前的叶生长             & 0.00   & 1.00   & *   & 0.138   & 0.646   \\
    VARPAR05 & 主茎叶生长                 & 0.50   & 3.00   &     & 1.6     & 1.6     \\
    VARPAR06 & 主茎叶生长                 & 0.00   & 0.02   &     & 0.010   & 0.010   \\
    VARPAR07 & 主茎叶生长                 & 18.0   & 28.0   &     & 24.0    & 24.0    \\
    VARPAR08 & 果汁页生长                 & 0.00   & 0.20   &     & 0.10    & 0.10    \\
    VARPAR09 & 铃生长周期                 & 18.9   & 38.0   &     & 28.0    & 28.0    \\
    VARPAR10 & 铃生长速率                 & 0.10   & 0.45   &     & 0.3293  & 0.3293  \\
    VARPAR11 & 最大铃干物质               & 3.00   & 15.0   &     & 8.8     & 8.8     \\
    VARPAR12 & 方铃展前茎生长             & 0.00   & 2.00   & *   & 0.284   & 0.590   \\
    VARPAR13 & 方铃展前茎生长             & 0.00   & 2.00   &     & 0.040   & 0.040   \\
    VARPAR14 & 方铃展前茎生长             & 0.00   & 0.08   &     & 0.014   & 0.014   \\
    VARPAR15 & 方铃展后茎生长             & 1.00   & 4.00   & *   & 3.971   & 2.001   \\
    VARPAR16 & 方铃展后茎生长             & 0.50   & 2.00   &     & 2.4     & 2.4     \\
    VARPAR17 & 方铃展后茎生长             & 0.00   & 1.00   &     & 0.10    & 0.10    \\
    VARPAR18 & 方铃展后茎生长             & 0.00   & 0.20   &     & 0.140   & 0.140   \\
    VARPAR19 & 株高生长                   & 0.00   & 0.50   &     & 0.20    & 0.20    \\
    VARPAR20 & 株高生长                   & 0.00   & 0.10   &     & 0.02    & 0.02    \\
    VARPAR21 & 株高生长                   & 10.0   & 18.0   & *   & 14.541  & 13.868  \\
    VARPAR22 & 株高生长                   & -4.00  & -2.00  & *   & -2.686  & -2.077  \\
    VARPAR23 & 株高生长                   & 0.09   & 0.12   &     & 0.10    & 0.10    \\
    VARPAR24 & 株高生长                   & 0.00   & 0.50   &     & 0.175   & 0.175   \\
    VARPAR25 & 株高生长                   & 1.00   & 4.00   &     & 2.20    & 2.20    \\
    VARPAR26 & 株高生长                   & 0.70   & 1.30   & *   & 1.2500  & 0.7500  \\
    VARPAR27 & 碳胁迫下的主茎节点延迟     & 0.50   & 1.00   & *   & 0.731   & 0.743   \\
    VARPAR28 & 碳胁迫下的坐果节点延迟     & 1.00   & 3.00   &     & 2.15    & 2.15    \\
    VARPAR29 & 碳胁迫下的坐果节点延迟     & 1.00   & 3.00   &     & 1.36    & 1.36    \\
    VARPAR30 & 温度对方铃形成的影响       & 0.70   & 1.30   & *   & 1.281   & 1.249   \\
    VARPAR31 & 现蕾前节点的发育           & 1.00   & 5.00   & *   & 3.210   & 2.175   \\
    VARPAR32 & 现蕾前节点的发育           & 1.00   & 3.00   & *   & 1.625   & 1.664   \\
    VARPAR33 & 现蕾前节点的发育           & 0.50   & 1.00   &     & 0.80    & 0.80    \\
    VARPAR34 & 初始叶面积                 & 0.01   & 0.10   & *   & 0.0407  & 0.0822  \\
    VARPAR35 & 果枝发育                   & -32.0  & -26.0  & *   & -26.118 & -27.108 \\
    VARPAR36 & 坐果节点的发育             & -57.0  & -49.0  &     & -54.00  & -54.00  \\
    VARPAR37 & 坐果节点的发育             & 0.00   & 2.00   &     & 0.80    & 0.80    \\
    VARPAR38 & 落叶剂对不同叶龄叶片的影响 & 0.00   & 5.00   &     & 3.20    & 3.20    \\
    VARPAR39 & 温度对吐絮的影响           & -306.0 & -240.0 &     & -292.0  & -292.0  \\
    VARPAR40 & 温度对吐絮的影响           & 0.70   & 1.30   &     & 1.08    & 1.08    \\
    VARPAR41 & 温度对衣份率的影响         & 45.0   & 60.0   & *   & 51.813  & 47.436  \\
    VARPAR42 & 温度对衣份率的影响         & 0.30   & 0.90   & *   & 0.844   & 0.341   \\
    VARPAR43 & 碳胁迫下的脱落强度         & 0.30   & 0.70   & *   & 0.514   & 0.601   \\
    VARPAR44 & 水分胁迫下的脱落强度       & 0.30   & 0.70   &     & 0.48    & 0.48    \\
    VARPAR45 & 方铃脱落的概率             & 0.10   & 0.50   &     & 0.24    & 0.24    \\
    VARPAR46 & 方铃脱落的概率             & 0.00   & 0.15   &     & 0.08    & 0.08    \\
    VARPAR47 & 铃脱落的概率               & 1.00   & 10.0   & *   & 3.833   & 6.439   \\
    VARPAR48 & 铃脱落的概率               & 0.30   & 1.50   & *   & 0.532   & 0.554   \\
    VARPAR49 & 铃脱落的概率               & 5.00   & 15.0   & *   & 7.714   & 6.867   \\
    VARPAR50 & 铃脱落的概率               & 0.20   & 1.80   & *   & 1.610   & 1.216   \\
    LIPAR01  & 光能截获参数               & -1.60  & 0.00   & *   & -       & -3.347  \\
    LIPAR02  & 光能截获参数               & -1.60  & 0.00   & *   & -       & -2.199  \\
    LIPAR03  & 光能截获参数               & -1.60  & 0.00   & *   & -       & -2.788  \\
    LIPAR04  & 光能截获参数               & -1.60  & 0.00   & *   & -       & -1.262  \\
    LIPAR05  & 光能截获参数               & -1.60  & 0.00   & *   & -       & -0.253  \\
    LIPAR06  & 光能截获参数               & -1.60  & 0.00   & *   & -       & -1.602  \\
    LIPAR07  & 光能截获参数               & -1.60  & 0.00   & *   & -       & -1.251  \\
    LIPAR08  & 光能截获参数               & -1.60  & 0.00   & *   & -       & -3.464  \\
    LIPAR09  & 光能截获参数               & -1.60  & 0.00   & *   & -       & -0.207  \\
    LIPAR10  & 光能截获参数               & -1.60  & 0.00   & *   & -       & -0.776  \\
    LIPAR11  & 光能截获参数               & -1.60  & 0.00   & *   & -       & -0.758  \\
    LIPAR12  & 光能截获参数               & -1.60  & 0.00   & *   & -       & -1.110  \\
    LIPAR13  & 光能截获参数               & -1.60  & 0.00   & *   & -       & -2.517  \\
    LIPAR14  & 光能截获参数               & -1.60  & 0.00   & *   & -       & -1.871  \\
    LIPAR15  & 光能截获参数               & -1.60  & 0.00   & *   & -       & -2.984  \\
    LIPAR16  & 光能截获参数               & -1.60  & 0.00   & *   & -       & -0.166  \\
    LIPAR17  & 光能截获参数               & -1.60  & 0.00   & *   & -       & -3.303  \\
    LIPAR18  & 光能截获参数               & -1.60  & 0.00   & *   & -       & -1.743  \\
    LIPAR19  & 光能截获参数               & -1.60  & 0.00   & *   & -       & -1.504  \\
    LIPAR20  & 光能截获参数               & -1.60  & 0.00   & *   & -       & -2.155  \\
\end{longtable}
