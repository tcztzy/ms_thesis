\chapter{结果}

\section{修改前后的 Cotton2K 模型对比}

根据第 \ref{chap:moo} 章的结果,选取修改前后的 Cotton2K 模型的模拟结果中选择的 Pareto 最优解决方案,%
对 2019 年与 2020 年两年各六个水分处理的共 12 个试验场景重新运行模拟,将模拟结果中的 12 个农业生态指标与实测田间数据进行对比。%
这 12 个指标包括:LY, AGM, LAI, LI, SCY, PH, MSN, LDM, PDM, SDM, BLN, BDM。%
结果如表 \ref{tab:stats},每个指标的统计学上的更优的方法以加粗突出显示。

\begin{table}
    \label{tab:stats}
    \caption{修改前后的 Cotton2K 模型模拟结果方差分析结果}
    \centering
    \begin{tabular}{lrlrr}
        \toprule
        指标 & F       & p 值      & 原版            & 改版            \\
        \midrule
        LY   & 4.8627  & 0.0382*   & 14.12           & \textbf{5.95}   \\
        SCY  & 6.2853  & 0.0201*   & 77.99           & \textbf{68.83}  \\
        AGM  & 1.1180  & 0.2925    & 62.28           & \textbf{44.84}  \\
        LAI  & 4.4933  & 0.0367*   & 29.90           & \textbf{28.92}  \\
        LI   & 1.8028  & 0.1831    & 33.70           & \textbf{24.72}  \\
        PH   & 21.9813 & 0.0000*** & 40.67           & \textbf{32.84}  \\
        MSN  & 4.6529  & 0.0330*   & 41.89           & \textbf{31.12}  \\
        LDM  & 24.4956 & 0.0000*** & 146.45          & \textbf{53.68}  \\
        PDM  & 35.8514 & 0.0000*** & 220.96          & \textbf{95.23}  \\
        SDM  & 1.4610  & 0.2292    & \textbf{132.93} & 212.96          \\
        BLN  & 0.0053  & 0.9422    & 71.44           & \textbf{62.46}  \\
        BDM  & 0.5421  & 0.4630    & 158.35          & \textbf{121.90} \\
        \bottomrule
    \end{tabular}
\end{table}

从表 \ref{tab:stats} 可以看出,除了干物质(SDM)外,修改后的模型在其他指标比较中的 NRMSE 都较小。%
SDM 的不理想可能是由于修改后的模型修改了叶片与茎的生长过程。%

\section{不同灌水处理下得叶面积时空分布}

使用修改后的 Cotton2K 模型依据对 2019 年的田间实验数据对不同灌水处理分别运行模拟,模拟的叶面积指数的时空分布如图 \ref{fig:laiDist} 所示。

\begin{figure}\label{fig:laiDist}
    \centering
    \includegraphics[scale=0.6]{lai_dist_2019.png}
    \caption{不同水分处理下得叶面积指数时空分布}
\end{figure}

可以看出,总体上,随着灌水定额的增加,叶面积指数在空间上,各层级分布更多,分布的层级更高;时间上,叶面积指数增长的时间更早,减少的时间更晚。%
多数叶面积指数集中在 10 cm 至 15 cm 的区间,出苗后 85 天至 120 天之间。%
叶面积指数的分布呈现斜置的纺锤状结构,初期、底层的叶面积指数和后期、高层的叶面积指数低,中期、中层的叶面积指数高。%
这说明在现蕾前的叶片被上层叶片遮蔽后逐渐失能脱落,后期上层增长的叶面积较少,这可能是由于新疆普遍实行的打顶操作引起的。%
叶面积指数在充分灌水的条件下会比不充分灌水的条件下有更广泛的时空分布。%
这说明提高灌水可以有效增加冠层的光能利用效率,从而提高光合同化产量。%
但灌水的增加也会增加营养生长的比例,试验中,最高灌水的 W6 处理的籽棉产量就低于 W5 灌水处理。
注意到叶面积指数在垂直方向上的分布存在断层,这可能是由两个原因导致的:\begin{enumerate*}
    \item 模型中在现蕾之后的株高增长过快或者
    \item 分配叶面积到各层的方法有待改进
\end{enumerate*}。
