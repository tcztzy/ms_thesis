\chapter{结论与展望}
本文对机理性很强的棉花生产模拟模型 Cotton2K 进行了详细阐述,并以新疆阿拉尔市 201 9年至 2020 年的“中棉 619”的田间试验为基础,通过 Sobol' 方法对 Cotton2K 模型的品种参数进行全局敏感性分析,%
并对敏感性指数较大的品种参数采用多目标优化算法进行优化,%
对模拟结果与实测数据进行分析,进而评估了研究区域棉花的模拟精度。

\section{主要结论}

本文 Cotton2K 源码被修改加入更细节的分层冠层子模块。%
事实证明,这样的修改有助于提高模型的模拟精度。
与其他模型,如CSM-Cropgro-Cotton和OZCOT相比,Cotton2K模型的模拟精度仍有很大的提升空间。%
这可能是由于多种原因引起的。%
模拟结果显示,地上总生物量和籽棉产量随着灌溉量的增加而增加,这与实际测量结果相当。%
这些模拟结果可作为南疆无膜栽培棉花的种植的参考数据。

\section{存在的问题与展望}

本文仍存在一些不足:
\begin{enumerate*}
    \item 在参数率定时考虑的生理指标过少
    \item 未检验模型对除阿拉尔地区以外的其他地区的有效性
\end{enumerate*}。下一步需要继续开展田间试验,增加新疆其他地区田间数据,并考虑多个生育指标对模型参数进行优化,从而进一步提高模型的模拟精度。