\chapter{结论与展望}

本文 Cotton2K 源码被修改加入更细节的分层冠层子模块。%
事实证明,这样的修改有助于提高模型的模拟精度。
与其他模型,如CSM-Cropgro-Cotton和OZCOT相比,Cotton2K模型的模拟精度仍有很大的提升空间。%
这可能是由于多种原因引起的。%
模拟结果显示,地上总生物量和籽棉产量随着灌溉量的增加而增加,这与实际测量结果相当。%
这些模拟结果可作为南疆无膜栽培棉花的种植的参考数据。

本文仍存在一些不足:
\begin{enumerate}
    \item 在参数率定时考虑的生理指标过少
    \item 未检验模型对除阿拉尔地区以外的其他地区的有效性
\end{enumerate}

下一步需要继续开展田间试验,增加更多的新疆其他地区田间数据,并考虑多个生育指标对模型参数进行优化,从而进一步提高模型的模拟精度。