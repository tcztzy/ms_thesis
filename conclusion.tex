\chapter{结论与展望}

%The source code of the Cotton2K model has been modified and the new more detailed canopy submodel has been replaced.
Cotton2K 源码被修改加入更细节的冠层子模块。
Such modifications proved to be helpful for the improvement of model simulation accuracy.
Compared with other models, such as CSM-Cropgro-Cotton and OZCOT, the simulation accuracy of Cotton2K model still has much room for improvement.
The simulation results showed that the total above-ground biomass and seed cotton yield increased with increasing irrigation, which was comparable to the actual measurements.
These simulation results may serve as reference data for planting non-mulched cultivation cotton in South Xinjiang.
本文筛选了 Cotton2K 模型的 46 个品种参数,探究了模型内部各个参数对模型结果
的相互影响,最终确立了参数值,并将模拟值与实测值进行了对比分析,验证了模型模
拟效果较好,为 Cotton2K 模型区域化提供了有效依据。但仍存在一些不足: 1)在模型
参数优化时,未考虑其它生育指标, 2)在利用 Cotton2K 模型的模拟过程中,未考虑不
同施肥处理对模拟结果的影响。下一步需要继续开展田间试验,增加多年的数据值,并
考虑多个生育指标对模型参数进行优化,从而提高模型的模拟精度。