\chapter{结论}
\section{结论}

本文 Cotton2K 源码被修改加入更细节的分层冠层子模块,经实验验证,这样的修改有助于提高模型的模拟精度。%
与其他模型,如 CSM-Cropgro-Cotton 和 OZCOT 相比,Cotton2K模型的模拟精度仍有很大的提升空间。%
造成这样的原因可能有很多,例如模型的复杂性,具体内容仍有待进一步研究。%
模拟结果显示,地上总生物量和籽棉产量随着灌溉量的增加而增加,%
试验中实测地上总生物量随灌水量增长的趋势与模拟一致,而籽棉产量在灌水增加到。%
这些模拟结果可作为南疆无膜栽培棉花的种植的参考数据。

\section{创新点}
本文详细描述了棉花生长模型 Cotton2K,修改了模型的源码以修复一些问题,增加更为详细的冠层子模块,%
以 2019 年至 2020 年的田间试验为基础,先通过全局敏感性分析方法对模型的品种参数进行分析,%
综合考量选取其中部分品种参数采用 NSGA-III 多目标优化算法进行优化,%
对模拟结果与实测数据进行分析,进而评估了模型的修改对研究区域棉花的模拟精度的影响。%
总的来说,在该模型对棉花生育早期特别是现蕾以前模拟的精度很高,随着器官的分化,%
棉花的生长发育机理变得越来越复杂,模拟的精度有所下降。%
但对于这样一个复杂的过程,模型在经过冠层光能截获率的修改后所能达到的结果是可以接受的。%

\section{存在的问题与下一步工作}

本文仍存在一些不足:
\begin{enumerate*}
    \item 未考虑无膜棉与有膜棉最大区别的土壤水分过程
    \item 未检验模型对除阿拉尔地区以外的其他地区的有效性
\end{enumerate*}。下一步需要增加其他研究区域的田间数据,继续对模型进一步修改,使其可以更细致地模拟土壤水分运移过程。