\chapter{材料与方法}

\section{试验观测}
试验区位于南疆地区阿拉尔市塔里木大学灌溉试验站 (\ang{81;11;46}E, \ang{40;37;28}N)。
\begin{figure}
    \centering
    \includegraphics[scale=0.4]{research_site.png}
    \caption{试验站所在位置}
\end{figure}
灌溉实验按照田间实验设计,采用滴灌方式,通过气象数据,按照水分需求进行灌溉,以南疆地区覆膜栽培棉花经验灌水量 4200 m$^3$/hm$^2$ (420 mm) 为基准,%
设置 6 个不同水平的灌水处理,分别为 80\% 灌水量 (W1)、90\% 灌水量 (W2)、100\% 灌水量 (W3)、110\% 灌水量 (W4)、120\% 灌水量 (W5)、130\% 灌水量 (W6)。
为避免相邻小区的影响,小区采取随机组合而非按灌水量排序,每小区面积 72 m$^2$,等行距种植,行间距为 76 cm,南北走向,%
采用一管带两行的灌溉方式间隔行铺设地下滴灌带,每个处理共有 3 个重复,总共 18 个小区。
因本文不研究其他营养对棉花生长的影响,所以 N、P、K 肥按照经验值随滴灌施入,各小区施肥量统一,%
$N$-$P_2O_5$-$K_2O$ 按照 250-100-50 $\mathrm{kg/hm^2}$ 的量进行施肥。

\subsection{气象数据}

该模型所需的每日气象数据包括太阳辐射 ($langley$),最高和最低温度 (℃) 和降水 (mm) 以及2米高度的风速 (km/h)。

\begin{table}
    \caption{Cotton2K 模型所需输入每日气象数据}\label{tab:meteorology}
    \small
    \centering
    \begin{tabular}{cccc}
        \toprule
        名称         & 解释                     & 单位                                              & 是否必须 \\
        \midrule
        太阳辐射强度 & 大气下垫面短波辐射强度   & $\mathrm{langley} = 0.04184\ \mathrm{MJ\ m^{-2}}$ & 是       \\
        最高气温     & 2 m 最高气温             & ℃                                                 & 是       \\
        最低气温     & 2 m 最低气温             & ℃                                                 & 是       \\
        露点温度     & 空气冷却达到饱和时的温度 & ℃                                                 & 否       \\
        降水         & 24 小时内降水            & mm                                                & 是       \\
        风速         & 2 m 风速                 & km/h                                              & 否       \\
        \bottomrule
    \end{tabular}
\end{table}

当露点温度不可得时,可通过公式 \ref{eq:dewpoint} 估算而得。

\begin{equation}\label{eq:dewpoint}
    T_{dew} = \begin{cases}
        SitePar_5                                                            & T_{\max} < 20        \\
        \frac{(40 - T_{\max}) * SitePar_5 + (T_{\max} - 20) * SitePar_6}{20} & 20 \le T_{\max} < 40 \\
        SitePar_6                                                            & T_{\max} \ge 40      \\
    \end{cases}
\end{equation}

式中,$T_{dew}$ 为露点温度 (℃),$T_{\max}$ 为最高气温 (℃),$SitePar_5$ 与 $SitePar_6$ 为用户提供输入,与试验站相关。

当每日风速数据不可得时,使用年平均风速代替。

\begin{figure}
    \centering
    \includegraphics[scale=0.5]{climate.png}
    \caption{2019{-}2021年试验站棉花生长期主要气象数据}
\end{figure}

冠层温湿度由精创 RC-4HA 收集,设置取样间隔为 0.5 小时。

\subsection{光能截获率}

在2019年使用LI-6400XT便携式光合作用系统,在2020年使用 LI-191R 线量子传感器和 LI-1500 光传感器在每10天的晴天中午13点至15点之间收集光截获数据。
行内空间被划分为网格,5个水平列和根据植物高度分为6层,每个网格的宽度相等,高度为 20 厘米,然后用设备测量每个网格的照度。
光截获量的计算方法是:同一水平层中各网格的平均光照度减去比率,参见公式\ref{eq:measured_li}。

\begin{equation}\label{eq:measured_li}
    LI_{i} = 1 - \frac{\sum^5_{j=1} I_{i,j}}{\sum^5_{j=1} I_{i+1,j}} \quad \mathrm{for} i \in {1,2,\dots,6}
\end{equation}

式中, $LI_{i}$ 为第 $i$ 层的光能截获率, $I_{i,j}$ 是第 $i$ 层第 $j$ 列网格的光照强度 ($W\ m^{-2}$), $I_{i+1,j}$ 为上层网格的光照强度 ($W\ m^{-2}$)。

\subsection{土壤参数}
每周取样测定一次土壤容积含水量(4 $\times$ 3 网格),另外,使用矫正后的土壤水分和温度自动记录仪 (HOBO H21-002, United States) 分 5 层实时监测 0-100cm 深度的土壤水分含量;

土壤温度由 Thermochron® iButton®器件(DS1921G) 收集,设置取样间隔为 2 小时,将传感器埋于各处理区中心紧挨棉花种植行的位置,埋深为 5 cm、10 cm、20 cm、40 cm、50 cm。

取样测量土壤田间持水率、容积密度、枯萎点含水率、饱和土壤含水率和渗透系数等参数,并记录好灌溉日期与灌溉量;
\subsection{棉花生理指标}
试验记录内容主要包括:
\begin{enumerate}
    \item 生育期:记录棉花播种、出苗、现蕾、开花和吐絮期等生育期;
    \item 株式图:每 10 天记录株高、叶片数、果枝数、果枝长度、果枝高度、蕾、花、小铃、大铃及吐絮铃个数与部位等;
    \item 叶面积指数:每 10 天结合干物质测定,取全株叶片,用扫描法测定冠层垂直方向 20 个深度和水平 8 个网格的叶面积指数;
    \item 干物质积累和分配:每 10 天取样一次,在苗期每次共取 5 株样品,开花期后每个处理取 3 株样品,样品分器官后在 80℃ 下烘干至恒重后,分别测定各器官的干物质重量;
    \item 产量构成因素:测定棉花籽棉产量、衣分、铃重和纤维重量;
\end{enumerate}

\section{模型模拟}
\subsection{模型改进}

分析叶面积指数、叶倾角和方位角的空间分布规律及光能截获分布,冠层深度拟分为 20 层,每层占棉花总高度的 5\%。
首先,计算每一层的有效叶面积指数,%
其次,定量计算每一层的光合作用碳同化产量,%
最后,重新驱动 Cotton2K 模型,模拟干物质积累。详细的流程参见第 \ref{sec:canopyLayering}节。

\subsection{模型参数化}
本文采用 2019-2020 连续 2 年的棉花实验数据对 Cotton2K 模型进行参数化。
其中作物生长参数如叶面积指数 (LAI),株高,干物质积累和主茎节数通过实验得到。
土壤参数如土壤水分、盐分初始值由实验得到;饱和导水率,Van-Genuchten
公式参数 $(\alpha, \beta)$ 等参数由率定得到;气象参数取自阿拉尔气象站;
作物参数和水分胁迫参数由率定得到。

\section{敏感性分析}
Sobol 敏感性分析方法是由俄罗斯科学家I. M. Sobol首次提出的\cite{sobol2001},基础是将模型输出方差分解为维度增加的输入参数方差的总和,
其目的是确定每个输入参数或不同参数之间的相互作用在什么水平上影响结果的方差。%
Sobol 敏感性分析中输出方差的分解采用了与因子设计中经典的方差分析相同的原理。%
应该注意的是,Sobol 敏感性分析的目的不是为了确定输入方差的原因。%
它只是表明它对模型输出有什么影响,影响到什么程度。%
因此,它不能用来确定方差的来源,如在棉花生产中,各农艺操作对棉花产量的影响来源。

任何敏感性分析的重要步骤之一,无论是局部还是整体,都是为了确定用于分析的适当模型输出。

Sobol 敏感性分析对模型输入和输出之间没有预设要求,这极大的提高的分析的普适性,而且 Sobol 敏感性分析可以评估单个参数以及参数间的交互作用。%
与其他全局敏感性分析方法类似,相比局部敏感性分析方法,Sobol 敏感性分析也有计算量大的缺陷。

\subsection{取样方法}
Sobol 序列是一个应用广泛的准随机 (quasi-random) 低差值序列,%
通常比完全随机的序列更均匀地对空间进行采样,用于生成参数空间的统一样本。%
本文采用 \authoryearcite{saltelli2002} 拓展的 Sobol 序列生成模型输入。%
生成的输入样本有 $N * (2D + 2)$ 条,其中 $N$ 为采样数, $D$ 是参数的数量。参见算法\ref{alg:sobol}。

Sobol 序列的一般特征有
\begin{enumerate}
    \item Sobol 序列是一种低差异的序列,也被称为 “准随机序列”。
    \item 比伪随机数的分布更均匀
    \item 准蒙特卡洛积分产生更快的收敛性和更好的准确性
    \item 缺点是需要计算高维度的积分
\end{enumerate}

需要注意的是,Sobol 序列的初始点有一些重复 (见\authornumcite{campolongo2011} 的表2),%
这可以通过设置 \texttt{skip\_values} 参数来避免,这样可以提高样本的均匀性。%
然而,事实证明,简单地跳过数值可能会降低精度,增加实现收敛所需的样本数量\cite{owen2021}。
建议将 \texttt{skip\_values} 和 $N$ 都设为 2 的幂,其中 $N$ 是所需的样本数 (进一步的背景见\authornumcite{owen2021}和\authornumcite{scipyAddStatsQMC2021}中的讨论)。%
其中还建议 $\mathtt{skip\_values} \ge N$。%
默认将 \texttt{skip\_values} 设置为不小于 $N$ 的 2 的幂。%
如果提供了 \texttt{skip\_values},该方法现在会在根据上述建议样本量可能不理想的情况下引发一个 \texttt{UserWarning}。

\begin{algorithm}
    \caption{Sobol 序列取样方法}\label{alg:sobol}
    \KwIn{$N, D$, 以及预定义的 $\mathbf{directions}$}
    \KwOut{$N \times D$的数组 $\mathbf{result}$}
    $scale = 31$\\
    $L = \left\lceil \frac{\log N}{\log 2} \right\rceil$\\
    \For{$i \in \{0,1,\dots,D-1\}$}{
        $\mathbf{V} = \left. \begin{bmatrix}0\\0\\\vdots\\0\end{bmatrix} \right\} L+1$\\
        \eIf{$i = 0$}{
            $\mathbf{V} = \begin{bmatrix}0\\ 2^1\\ 2^2 \\ \vdots\\ 2^L \end{bmatrix}$
        }{
            $\mathbf{m} = directions_{i - 1}$\\
            $a = m_0$\\
            $s = len(\mathbf{m})$\\
            \eIf{$L \le s$}{
                $\mathbf{V} = \begin{bmatrix}0\\ m_1 \\ m_2 \\ \vdots\\ m_L \end{bmatrix}\circ \begin{bmatrix}0\\ 2^{scale - 1} \\ 2^{scale - 2} \\ \vdots\\ 2^{scale - L} \end{bmatrix}$
            }{
                $\mathbf{V} = \begin{bmatrix}0\\ m_1 \\ m_2 \\ \vdots\\ m_s \\ m_1 \\ m_2 \\ \vdots \\ m_{L - s} \end{bmatrix} \circ \begin{bmatrix}0\\ 2^{scale - 1} \\ 2^{scale - 2} \\ \vdots\\ 2^{scale - s} \\ 1 \\ 1 \\ \vdots \\ 1 \end{bmatrix}$\\
                \For{$j \in \{s+1,s+2,\dots,L\}$}{
                    $V_j = V_{j - s} \veebar (V_{j - s} >> s)$\\
                    \For{$k \in \{s+1,s+2,\dots,L\}$}{
                        $V_j = V_j \veebar ((a >> (s - 1 - k)) \land 1) \times V_{j-k}$
                    }
                }
            }
        }
        $X = 0$\\
        \For{$j \in \{1,2,\dots,N\}$}{
            $X = X \veebar V_{\mathtt{最小显著零位的索引}(j-1)}$
            $result_{ji} = \frac{X}{2^{scale}}$
        }
    }
\end{algorithm}

\begin{algorithm}
    \caption{最小显著零位的索引}
    \KwIn{$value$}
    \KwOut{$index$}
    $index = 1$\\
    \While{$value \mod 2 \neq 0$}{
        $value = value >> 1$\\
        $index = index + 1$
    }
\end{algorithm}


\subsection{分析方法}

Sobol 敏感性分析一般用于复杂的系统模型,它将输出方差与其资源进行定量分解:即来自单个参数或来自参数间的相互作用。%
全局敏感性指数通常用于评估一个参数的总体贡献以及与其他参数的相互作用。%
Sobol 敏感性指数有几个特点:
\begin{enumerate}
    \item 全局/一阶/二阶敏感性指数为正值。
    \item 敏感性指数大于0.05的参数被认为是显著的。
    \item 所有敏感指数之和应等于1。
    \item 全局敏感性指数大于一阶敏感性指数。
\end{enumerate}

进行可靠的Sobol 敏感性分析所需的样本量取决于两个主要因素,\begin{enumerate*}
    \item 模型的复杂性和
    \item 评估的参数数量
\end{enumerate*} 。尽管对生成的最佳参数集数量没有普遍共识,但一般的经验法则是,模型参数的数量越大,使用的参数集数量就越多。%
例如,对于一个有大量不确定参数的复杂模型(如 20 个参数),至少要进行 100,000 次模型评估。%
对于不那么复杂的模型,较少的评估次数(如1000次)可能就足够了。%
但应该注意的是,随着评价数量的增加,计算成本也会增加。%
选择的评价数量是否合适,可以用自举法置信区间来检验。%
一般来说,最敏感的参数应该有较窄的置信区间,即小于敏感指数的10\%。

\section{多目标优化}
NSGA-III 的基本框架类似于原始的 NSGA-II 算法 \cite{NSGA2},其选择运算符有重大变化。
与 NSGA-II 中介绍的拥挤度概念不同的是,NSGA-III 是通过用户输入或模型自行选择一些在高维参数空间良好分布 (fine distributed) 的参考点来实现保留部分次优解以维持种群多样性。
为了完整起见,首先对原始的 NSGA-II 算法进行简要描述。
考虑 NSGA-II 算法的第 $t$ 代,假设这一代的父代种群为 $P_t$,其规模为 $N$ ,而由 $P_t$ 产生的子代种群为 $Q_t$,有 $N$ 个成员。
第一步是从结合的亲代和子代种群 $R_t = P_t \cup Q_t$(大小为 $2N$ )中选择最好的 $N$ 个成员,从而能够保留亲代种群的精英成员。
为了实现这一目标,首先根据不同的非同源等级($F_1$、$F_2$等)对组合群体 $R_t$ 进行排序。
然后,从 $F_1$ 开始,每次选择一个非同化水平来构建一个新的种群 $S_t$ ,直到St的大小等于 $N$ 或首次超过 $N$。
因此,从第 $(l+1)$ 级开始的所有解决方案都被拒绝在组合群体 $R_t$ 中。
在这种情况下,只有那些能使第l层前面的多样性最大化的解决方案被选择。
在 NSGA-II 中,这是通过一个计算效率高但近似的利基 (niche) 保护算子来实现的,该算子将每个最后一级成员的拥挤距离计算为两个相邻解决方案之间的客观归一化距离之和。
此后,具有较大拥挤距离值的解决方案被选中。
在 NSGA-III 中,用以下方法取代拥挤距离算子。

\subsection{将种群按非支配等级分类}

上述利用通常的支配原则\cite{chankong1983}识别非支配前沿的程序也被用于NSGA-III。
如果 $|St|=N$ ,则不需要进一步的操作,下一代从$P_t+1=S_t$开始。
对于 $|St|>N$,从一到 ($l-1$) 个前沿的成员已经被选中,即 $P_{t+1}= \cup_{i=1}^{l-1} F_i$ ,剩下的 $(K=N-|Pt+1|)$ 种群成员从最后的前沿 $F_l$ 中选择。
在下面几个小节中描述其余的选择过程。

\subsection{确定空间中的参考点}
如前文所述, 默认情况下,NSGA-III 会使用一组预定义的参考点集,%
但参考点集也可以由用户提供,这些参考点将会用于对次优的 Pareto 前沿进行排序并选取合适的解进入下代的迭代。
在没有任何偏好信息的情况下,可以采用任何预定义的结构化的参考点放置,但实践中常常使用 Das和Dennis的\cite{das&dennis1998}系统方法1,
将点放置在一个 $M-1$ 维的单位射线与归一化的超平面交点上,该平面对所有目标轴的交角相同且截距为1,以 3 维的问题为例,该平面的数学表达式为 $x + y + z = 1$。
如果沿每个目标考虑 $p$ 个划分,那么在一个 $M$ 目标问题中,参考点的总数 ($H$) 由以下公式给出

\begin{equation}\label{eq:H}
    H = \begin{pmatrix}
        M + p - 1 \\
        p
    \end{pmatrix}
\end{equation}

例如,在一个三目标问题 ($M = 3$) 中,参考点被创建在一个三角形上,顶点在 $(1, 0, 0)$,$(0, 1, 0)$,$(0, 0, 1)$。
如果为每个目标轴选择四个分部 $(p=4)$, $H = (\begin{smallmatrix}
        3+4-1\\
        4
    \end{smallmatrix})$ 或 $15$ 个参考点将被创建,这些参考点如图 \ref{fig:refPoints} 所示。
在 NSGA-III 中,除 Pareto 前沿的解集外,还保留了与这些参考点的相关的种群成员。
因为构造的参考点是均匀地分布在超平面上,因此获得的解集也可能均匀分布在 Pareto 前沿附近。
该程序在算法 \ref{alg:NSGA3} 中提出。
\begin{figure}
    \tdplotsetmaincoords{75}{135}
    \centering
    \begin{tikzpicture}
        [tdplot_main_coords,
            axis/.style={-stealth,thick},
            vector/.style={-stealth,very thick}]
        \draw[axis] (0,0,0) -- (5,0,0) node[anchor=east]{$f1$};
        \draw[axis] (0,0,0) -- (0,5,0) node[anchor=west]{$f2$};
        \draw[axis] (0,0,0) -- (0,0,5) node[anchor=west]{$f3$};
        \filldraw[fill=gray,fill opacity=0.8] (0,0,4)--(4,0,0)--(0,4,0)--cycle;
        \draw[-stealth,very thick] (0,3,3) node[anchor=west]{归一化平面} -- (0.5,1.25,2.25);
        \draw[-stealth,very thick] (2,3,0) node[anchor=west]{理想点} -- (0,0,0);
        \fill(4,0,0) circle (2pt) node[anchor=south] {1};
        \fill(3,1,0) circle (2pt);
        \fill(3,0,1) circle (2pt);
        \fill(2,2,0) circle (2pt);
        \fill(2,1,1) circle (2pt);
        \fill(2,0,2) circle (2pt);
        \fill(1,3,0) circle (2pt);
        \fill(1,2,1) circle (2pt);
        \fill(1,1,2) circle (2pt);
        \fill(1,0,3) circle (2pt);
        \draw[-stealth,very thick] (3,0,4) node[anchor=east]{参考点} -- (1,0,3);
        \fill(0,4,0) circle (2pt) node[anchor=south] {1};
        \fill(0,3,1) circle (2pt);
        \fill(0,2,2) circle (2pt);
        \fill(0,1,3) circle (2pt);
        \fill(0,0,4) circle (2pt) node[anchor=south east] {1};
        \draw[vector] (0,0,0) -- (3,1.5,1.5) node[anchor=south]{参考向量};
    \end{tikzpicture}
    \caption{15个结构化参考点显示在归一化参考平面上,适用于$p=4$的三目标问题}\label{fig:refPoints}
\end{figure}


\begin{algorithm}
    \caption{NSGA-III 第 t 代的过程}\label{alg:NSGA3}
    \KwIn{$H$ 构造的参考点 $Z^s$ 或提供的目标点 $Z^a$, 父代种群 $P_t$}
    \KwOut{$P_{t+1}$}

    $S_t = \emptyset, i = 1$\\
    $Q_t$ = 重组+变异($P_t$)\\
    $R_t = P_t \cup Q_t$\\
    $(F_1, F_2, \dots)$ = 非支配排序($R_t$)\\
    \Repeat{$|S_t \ge N|$}{$S_t = S_t \cup F_i$ 且 $i = i + 1$}
    包含父代的 Pareto 前沿: $F_l = F_i$\\
    \eIf{$|S_t| = N$}{
    $P_{t+1} = S_t$, break
    }{
    $P_{t+1} = \cup^{l-1}_{j=1} F_j$\\
    从 $F_l$ 中选点: $K = N - |P_{t+1}|$\\
    归一化目标并创建参考集 $Z^r$: $\mathtt{Normalize}(\mathbf{f}^n, S_t, Z^r, Z^s, Z^a)$\\
    将 $S_t$ 的每个成员 $\mathbf{s}$ 与参考点相关联: $[\pi(\mathbf{s}), d(\mathbf{s})] =\mathtt{Associate}(S_t, Z^r)$
    \tcc{$\pi(\mathbf{s})$: 最近参考点, $d$: $\mathbf{s}$ 与 $\pi(\mathbf{s})$ 之间的距离}
    计算参考点的利基 (niche) 数 $j \in Z^r$: $\rho_j = \sum_{\mathbf{s}\in S_t/F_l}((\pi(\mathbf{s})) ? 1 : 0 )$\\
    一次从 $F_l$ 选取 $K$ 个成员构建 $P_{t+1}$: $\mathtt{Niching}(K, \rho_j, \pi, d, Z^r, F_l, P_{t+1})$
    }
\end{algorithm}
\subsection{种群成员的适应性归一化}
首先,$S_t$种群的理性点是由在$\cup_{\tau=0}^t S_{\tau}$中的每个目标函数$i =1,2,\dots,M$的最小值 ($z_i^{\min}$) 通过构建理想点%
$\overline{z} = (z_1^{\min}, z_2^{\min},\dots,z_M^{\min})$ 来确定的。%
然后,$S_t$ 的每个目标值通过用$z_i^{\min}$减去目标$f_i$来翻译,这样,翻译后的 $S_t$ 的理想点就成为一个零矢量。%
接着,通过寻找使相应的成就标化函数 (由$f'_i(\mathbf{x}) = f_i(\mathbf{x}) - z_i^{\min}$和接近第$i$个目标轴的权重向量形成) 最小的解 $(x \in S_t)$,%
来确定每个(第$i$个)目标轴中的极端点 ($z^{i,\max}$)。
随后,这些$M$个极端向量被用来构成一个$M$维的超平面。%
接下来可以计算出第$i$个目标轴和线性超平面的截距$a_i$(见图\ref{fig:formingHyperPlane})。%
\begin{figure}
    \tdplotsetmaincoords{45}{120}
    \centering
    \begin{tikzpicture}
        [tdplot_main_coords,
            cube/.style={very thick,black},
            grid/.style={very thin,gray},
            axis/.style={-stealth,thick},
            vector/.style={-stealth,very thick},
            annotation/.style={fill=white,font=\footnotesize,inner sep=1pt}]
        \draw[axis] (0,0,0) -- (6,0,0) node[anchor=east]{$f'_1$};
        \draw[axis] (0,0,0) -- (0,6,0) node[anchor=west]{$f'_2$};
        \draw[axis] (0,0,0) -- (0,0,6) node[anchor=west]{$f'_3$};
        \foreach \coo in {2,4}
            {
                \draw (\coo, 0, 0) node[anchor=east] {\fpeval{\coo/2}} -- (\coo, 0.25, 0);
                \draw (0, \coo, 0) node[anchor=south] {\fpeval{\coo/2}} -- (0.25, \coo, 0);
                \draw (0, 0, \coo) node[anchor=east] {\fpeval{\coo/2}} -- (0, 0.25, \coo);
            }
        \foreach \coo in {1,3,5}
            {
                \draw (\coo, 0, 0) -- (\coo, 0.125, 0);
                \draw (0, \coo, 0) -- (0.125, \coo, 0);
                \draw (0, 0, \coo) -- (0, 0.125, \coo);
            }
        \draw[thick] (0,0,5.2)--(4.4,0,0)--(0,3.5,0)--cycle;
        \draw (4.4,0,0)--(4.4,-1,0);
        \draw (0,0,0)--(0,-1,0);
        \draw (0,0,5.2)--(0,-1,5.2);
        \draw (0,0,0)--(-1,0,0);
        \draw (0,3.5,0)--(-1,3.5,0);
        \draw[arrows=<->] (0,-0.8,0)--(2.2,-0.8,0) node[annotation] {$a_1$}--(4.4,-0.8,0);
        \draw[arrows=<->] (-0.8,0,0)--(-0.8,1.75,0) node[annotation] {$a_2$}--(-0.8,3.5,0);
        \draw[arrows=<->] (0,-0.8,0)--(0,-0.8,2.6) node[annotation] {$a_3$}--(0,-0.8,5.2);
        \fill(\fpeval{4.4*0.1},\fpeval{3.5*0.1},\fpeval{5.2*0.8}) circle (2pt);
        \fill(\fpeval{4.4*0.05},\fpeval{3.5*0.8},\fpeval{5.2*0.15}) circle (2pt);
        \fill(\fpeval{4.4*0.8},\fpeval{3.5*0.1},\fpeval{5.2*0.1}) circle (2pt);
        \draw[dash pattern=on 5pt off 5pt] (\fpeval{4.4*0.1},\fpeval{3.5*0.1},\fpeval{5.2*0.8}) node[anchor=south west] {$z^{3,\max}$}--(\fpeval{4.4*0.05},\fpeval{3.5*0.8},\fpeval{5.2*0.15}) node[anchor=south west] {$z^{2,\max}$}--(\fpeval{4.4*0.8},\fpeval{3.5*0.1},\fpeval{5.2*0.1}) node[anchor=west] {$z^{1,\max}$}--cycle;
    \end{tikzpicture}
    \caption{以三目标问题为例,计算截距,然后从极端点形成超平面的程序}\label{fig:formingHyperPlane}
\end{figure}
要特别注意处理退化的情况和非负的截距。%
再然后,目标函数可以被规范化为
\begin{equation}\label{eq:normalize}
    f_i^n (\mathbf{x})= \frac{f_i'(x)}{a_i},\ \mathrm{for}\ i = 1, 2, \dots, M.
\end{equation}
请注意,现在每个归一化目标轴的截点都在$f_i^n = 1$,用这些截点构建的超平面将使$\sum^M_{i=1} f_i^n = 1$。
在结构化参考点 (其中有$H$个) 的情况下,用Das和Dennis\cite{das&dennis1998}的方法计算的原始参考点已经位于这个归一化的超平面上。%
在用户偏爱参考点的情况下,参考点只需使用 \ref{eq:normalize} 映射到上述构建的标准化超平面上。%
由于归一化程序和超平面的创建是在每一代使用从模拟开始时发现的极端点来完成的,%
因此 NSGA-III 程序在每一代都能自适应地保持 $S_t$ 成员所跨越空间的多样性。%
这使得 NSGA-III 能够解决具有帕累托最优前沿的问题,其目标值可以有不同的比例。%
该程序也在算法 \ref{alg:normalize} 中进行了描述。
\begin{algorithm}
    \caption{$\mathtt{Normalize}(\mathbf{f}^n, S_t, Z^r, Z^s, Z^a)$过程}\label{alg:normalize}
    \KwIn{$S_t$, $Z^s$ (构造点) 或 $Z^a$ (提供点)}
    \KwOut{$\mathbf{f}^n$, $Z^r$ (归一化的超平面上的参考点)}

    \For{$j = 1 \mathbf{to} M$}{
    计算理想点: $z_j^{\min} = \min_{\mathbf{S}\in S_t} f_j(s)$
    翻译目标: $f_j'(\mathbf{s}) = f_j(\mathbf{s}) - z_j^{\min} \quad \forall \mathbf{s} \in S_t$
    计算极点: $(\mathbf{z}^{j,\max}, j = 1, \dots, M)$ of $S_t$
    }

    计算截点 $a_j$ 对 $j = 1, \dots, M$
    使用公式 \ref{eq:normalize} 归一化目标 ($\mathbf{f}^n$)

    \eIf{是否提供 $Z^a$}{
        使用公式 \ref{eq:normalize} 将每个(吸气)点映射到归一化的超平面上,并将这些点保存在集合 $Z^r$ 中。
    }{
        $Z^r=Z^s$
    }
\end{algorithm}
\subsection{关联操作}
在根据目标空间中$S_t$成员的范围自适应地对每个目标进行归一化后,需将每个群体成员与一个参考点联系起来。%
为此,通过连接参考点和原点,在超平面上定义一条对应于每个参考点的参考线。%
然后,计算 $S_t$ 的每个种群成员与每条参考线的垂直距离。%
在归一化目标空间中,参考线最接近种群成员的参考点被认为与该种群成员相关。%
这在图\ref{fig:associate}中得到了说明。该程序在算法\ref{alg:associate}中提出。
\begin{algorithm}
    \caption{$\mathtt{Associate}(S_t,Z^r)$ 过程}\label{alg:associate}
    \KwIn{$Z^r, S_t$}
    \KwOut{$\pi(\mathbf{s} \in S_t), d(\mathrm{s} \in S_t)$}
    \ForEach{每个参考点 $\mathbf{z} \in Z^r$}{
        计算参考线 $\mathbf{w} = \mathbf{z}$
    }
    \ForEach{$\mathbf{s} \in S_t$}{
    \ForEach{$\mathbf{w} \in Z^r$}{
        计算 $d^{\bot}(\mathbf{s}, \mathbf{w}) = \parallel (\mathbf{s} - \mathbf{w}^T \mathbf{sw} /\parallel \mathbf{w} \parallel^2) \parallel $
    }
    $\pi(\mathbf{s}) = \mathbf{w} : \mathrm{argmin}_{\mathbf{w} \in Z^r} d^{\bot}(\mathbf{s}, \mathbf{w})$
    $d(\mathbf{s}) = d^{\bot}(\mathbf{s},\pi(\mathbf{s}))$
    }
\end{algorithm}
\begin{figure}
    \tdplotsetmaincoords{45}{150}
    \centering
    \begin{tikzpicture}
        [tdplot_main_coords,
            cube/.style={very thick,black},
            grid/.style={very thin,gray},
            axis/.style={-stealth,thick},
            vector/.style={-stealth,very thick},
            annotation/.style={fill=white,font=\footnotesize,inner sep=1pt},
            refPoints/.style={fill=gray},
            sample/.style={fill=black},
            dashedLine/.style={dash pattern=on 5pt off 5pt}]
        \draw (0,6,0) -- (3,6,0) node[anchor=north west]{$f'_1$} -- (6,6,0);
        \draw (6,0,0) -- (6,3,0) node[anchor=east]{$f'_2$} -- (6,6,0);
        \draw (6,0,0) -- (6,0,3) node[anchor=west]{$f'_3$} -- (6,0,6);
        \foreach \tick in {0,0.5,1,1.5} {
                \draw (\fpeval{0+\tick*4},6,0) -- (\fpeval{0+\tick*4},6.25,0) node[anchor=north west] {\tick};
                \draw (6,\fpeval{0+\tick*4},0) -- (6.25,\fpeval{0+\tick*4},0) node[anchor=north east] {\tick};
                \draw (6,0,\fpeval{0+\tick*4}) -- (6,-0.25,\fpeval{0+\tick*4}) node[anchor=south east] {\tick};
            }
        \filldraw[fill=gray!30,fill opacity=0.6] (0,0,4)--(4,0,0)--(0,4,0)--cycle;
        \fill[refPoints](4,0,0) circle (2pt);
        \draw[dashedLine] (0,0,0) -- (5,0,0);
        \fill[refPoints](\fpeval{8/3},\fpeval{4/3},0) circle (2pt);
        \draw[dashedLine] (0,0,0) -- (\fpeval{8/3 * 5 / sqrt(80/9)},\fpeval{4/3 * 5 / sqrt(80/9)},0);
        \fill[refPoints](\fpeval{4/3},\fpeval{8/3},0) circle (2pt);
        \draw[dashedLine] (0,0,0) -- (\fpeval{4/3 * 5 / sqrt(80/9)},\fpeval{8/3 * 5 / sqrt(80/9)},0);
        \fill[refPoints](0,4,0) circle (2pt);
        \draw[dashedLine] (0,0,0) -- (0,5,0);
        \fill[refPoints](\fpeval{8/3},0,\fpeval{4/3}) circle (2pt);
        \draw[dashedLine] (0,0,0) -- (\fpeval{8/3 * 5 / sqrt(80/9)},0,\fpeval{4/3 * 5 / sqrt(80/9)});
        \fill[refPoints](\fpeval{4/3},\fpeval{4/3},\fpeval{4/3}) circle (2pt);
        \draw[dashedLine] (0,0,0) -- (\fpeval{4/3 * 5 / sqrt(16/3)},\fpeval{4/3 * 5 / sqrt(16/3)},\fpeval{4/3 * 5 / sqrt(16/3)});
        \fill[refPoints](0,\fpeval{8/3},\fpeval{4/3}) circle (2pt);
        \draw[dashedLine] (0,0,0) -- (0,\fpeval{8/3 * 5 / sqrt(80/9)},\fpeval{4/3 * 5 / sqrt(80/9)});
        \fill[refPoints](0,\fpeval{4/3},\fpeval{8/3}) circle (2pt);
        \draw[dashedLine] (0,0,0) -- (0,\fpeval{4/3 * 5 / sqrt(80/9)},\fpeval{8/3 * 5 / sqrt(80/9)});
        \fill[refPoints](\fpeval{4/3},0,\fpeval{8/3}) circle (2pt);
        \draw[dashedLine] (0,0,0) -- (\fpeval{4/3 * 5 / sqrt(80/9)},0,\fpeval{8/3 * 5 / sqrt(80/9)});
        \fill[refPoints](0,0,4) circle (2pt);
        \draw[dashedLine] (0,0,0) -- (0,0,5);

        % sample points
        \fill[sample](0.5,0,4) circle (2pt);
        \draw (0.5,0,4)--(0.0,0.0,4.0);
        \fill[sample](0.6,3.9,2.1) circle (2pt);
        \draw (0.6,3.9,2.1)--(0.0,3.96,1.98);
        \fill[sample](1,4,1) circle (2pt);
        \draw (1,4,1)--(1.8,3.6,0.0);
        \fill[sample](1.2,1.2,3.6) circle (2pt);
        \draw (1.2,1.2,3.6)--(1.68,0.0,3.36);
        \fill[sample](4,1,2) circle (2pt);
        \draw (4,1,2)--(4.0,0.0,2.0);
        \fill[sample](3.2,1,3) circle (2pt);
        \draw (3.2,1,3)--(3.76,0.0,1.88);
        \fill[sample](3.2,1.8,2.4) circle (2pt);
        \draw (3.2,1.8,2.4)--(2.46667,2.46667,2.46667);
    \end{tikzpicture}
    \caption{展示了将种群成员与参考点关联}\label{fig:associate}
\end{figure}

\subsection{利基 (niche) 保护操作}
值得注意的是,一个参考点可以有一个或多个种群成员与之相关联,或者不需要有任何种群成员与之相关联。%
计算 $P_{t+1}=S_t/F_l$ 中与每个参考点相关的种群成员的数量。%
第 $j$ 个参考点的这个利基计数记作 $\rho_j$。%
首先,确定参考点集 $J_{\min} = \{j : \mathrm{argmin}_j \rho_{j}\}$ 具有最小$\rho_j$。%
在有多个这样的参考点的情况下,随机选择一个 ($\overline{j} \in J_{\min}$)。%
如果 $\rho_{\overline{j}}= 0$(意味着参考点 $\overline{j}$ 没有相关的$P_{t+1}$成员),在集合$F_l$中的 $\overline{j}$ 可能有两种情况。%
首先, $F_l$ 中存在一个或多个与参考点 $\overline{j}$相关的成员。%
在这种情况下,与参考线垂直距离最短的成员被添加到 $P_{t+1}$。%
然后,参考点 $\overline{j}$ 的计数 $\rho_{\overline{j}}$递增1。%
第二,前面的$F_l$没有任何与参考点 $\overline{j}$ 相关的成员。%
在这种情况下,参考点被排除在当前一代的进一步考虑之外。%
在 $\rho_{\overline{j}} \ge 1$ 的情况下(意味着$S_t/F_l$中已经有一个与参考点相关的成员存在),如果存在的话,从前面$F_l$中随机选择一个与参考点$\overline{j}$相关的成员被添加到$P_{t+1}$。%
然后,$\rho_{\overline{j}}$的计数被增加1。%
利基计数更新后,该程序共重复K次,以填补 $P_{t+1}$ 的空缺。%
该程序在算法\ref{alg:niche}中提出。
\begin{algorithm}
    \caption{$\mathtt{Niching}(K, \rho_j, \pi, d, Z^r, F_l, P_{t+1})$过程}\label{alg:niche}
    \KwIn{$K, \rho_j, \pi(\mathbf{s} \in S_t), d(\mathbf{s} \in S_t), Z^r, F_l$}
    \KwOut{$P_{t+1}$}
    $k = 1$
    \While{$k \le K$}{
    $J_{\min} = \{ j:\mathrm{agrmin}_{j \in Z^r} \rho_j \}$
    $\overline{j}  = \mathrm{random}(J_{\min})$
    $I_{\overline{j}} = \{ s: \pi(\mathbf{s}) = \overline{j}, s \in F_l \}$
    \eIf{$I_{\overline{j}} \neq \emptyset$}{
    \eIf{$\rho_{\overline{j}} = 0$}{
    $P_{t+1} = P_{t+1} \cup ( \mathbf{s}: \mathrm{argmin}_{\mathbf{S} \in I_{\overline{j}}} d(\mathbf{s}) )$
    }{
    $P_{t+1} = P_{t+1} \cup \mathrm{random}(I_{\overline{j}})$
    }
    $\rho_{\overline{j}} = \rho_{\overline{j}} + 1, F_l = F_l \setminus \mathbf{s}$
    $k = k + 1$
    }{
    $Z^r = Z^r / \{ \overline{j} \}$
    }
    }
\end{algorithm}

\subsection{创造后代群体的遗传操作}
$P_{t+1}$ 形成后,再通过应用通常的遗传算子来创建一个新的子代群体 $Q_{t+1}$。%
在NSGA-III中,已经对解决方案进行了仔细的精英选择,并试图通过强调最接近每个参考点参考线的解决方案来保持解决方案的多样性。%
此外,为了达到快速计算的目的,设定 $N$ 几乎等于 $H$,从而期望在每个参考点对应的帕累托最优前线附近进化出一个群体成员。%
由于所有这些原因,在处理仅有箱体约束的问题时,没有采用 NSGA-III 的任何明确的繁殖操作。%
群体 $Q_{t+1}$ 是通过从 $P_{t+1}$ 中随机挑选父母,应用通常的交叉和变异操作来构建。%
然而,为了使后代的解更接近父代的解,建议在SBX运算中使用一个相对较大的分布指数值,从而使后代接近其父代。

\subsection{NSGA-III 中一代的计算复杂度}
对具有 $M$ 维目标向量的 $2N$ 大小的群体进行非支配性排序(算法 \ref{alg:NSGA3} 的第4行)需要 $O(N \log ^{M-2} N)$次计算\cite{kung1975}。%
算法\ref{alg:normalize}第2行中的理想点的识别总共需要 $O(MN)$ 次计算。%
目标的转换(第3行)需要 $O(MN)$ 次计算。%
然而,识别极端点(第4行)需要 $O(M^2N)$次计算。%
确定截距(第6行)需要一个大小为 $M \times M$ 的矩阵反转,需要 $O(M^3)$ 次运算。%
此后,对最大的 $2N$ 个种群成员进行归一化(第7行)需要 $O(N)$ 次计算。%
算法\ref{alg:normalize}的第8行需要 $O(MH)$次计算。%
算法\ref{alg:associate}中把最多2N个种群成员与H个参考点联系起来的所有操作都需要 $O(MNH)$计算。%
此后,在算法\ref{alg:niche}的排队程序中,第3行将需要 $O(H)$ 次比较。假设 $L=|F_l|$,第5行需要 $O(L)$ 个检查。%
第8行在最坏的情况下需要 $O(L)$ 次计算。其他操作的复杂度较小。%
然而,在 $\mathtt{Niching}$ 算法中,上述计算最多需要执行L次,因此需要更大的$O(L^2)$或$O(LH)$计算。%
在最坏的情况下($S_t = F_1$,即第一个非支配阵线超过种群大小),$L \le 2N$。%
在我们所有的模拟中,我们使用了$N \approx H$,通常 $N > M$。%
考虑到上述所有的考虑和计算,NSGA-III一代的总体最坏情况下的复杂度为 $O(N^2 \log^{M-2} N)$ 或 $O(N^2M)$,以较大者为准。

\subsection{NSGA-III 的无参数特性}
与NSGA-II一样,NSGA-III算法不需要设置任何新的参数,除了通常的GA参数,如种群大小、终止参数、交叉和变异概率及其相关参数。%
参考点的数量 $H$ 不是一个算法参数,因为这与期望的权衡点的数量直接相关。%
种群规模 $N$ 取决于 $H$,因为$N \approx H$。%
参考点的位置同样取决于用户对所获得的解决方案中的偏好信息。

\section{本章小结}
本章对田间试验设计及数据获取进行了简单的介绍,%
试验站位于新疆生产建设兵团第一师阿拉尔市灌溉试验站内,%
田间试验数据主要用于模型的校准与验证。%
对模型模拟所需要的技术进行了简单的叙述,%
然后详细介绍了敏感性分析和多目标优化的方法。
