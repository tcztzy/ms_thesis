\chapter{材料与方法}


% Drawing part, node distance is 1.5 cm and every node
% is prefilled with white background
\begin{tikzpicture}[
        node distance=1cm,
        every node/.style={fill=white, font=\sffamily},
        align=center
    ]
    % Specification of nodes (position, etc.)
    \node (start) [terminalStart] {开始};
    \node (firstSampling) [data, below left=of start, xshift=-3cm] {对 70 个 Cotton2K \\参数进行 Sobol 取样};
    \node (fieldData) [data, below=of start, yshift=-1.5cm] {田间数据};
    \node (firstSimulation) [process, below=of firstSampling, yshift=-1.5cm] {Cotton2K 模拟};
    \node (firstDatabase) [data, below=of firstSimulation] {12 个输出变量的\\评价指标数据库};
    \node (sa) [process, below=of firstDatabase] {使用 SALib 进行\\Sobol 全局敏感性分析};
    \node (influential) [decision, below=of sa] {S1 > 0.05};
    \node (fixParam) [process, below=of start, yshift=-10cm] {固定参数为默认值};
    \node (secondSampling) [data, below right=of start, xshift=3cm] {对 35 个 Cotton2K \\参数进行 Sobol 取样};
    \node (secondSimulation) [process, below=of secondSampling, yshift=-1.5cm] {Cotton2K 模拟};
    \node (secondDatabase) [data, below=of secondSimulation] {12 个输出变量的\\评价指标数据库};
    \node (moo) [process, below=of secondDatabase] {求解 Pareto 解集};
    \node (prune) [process, below=of moo] {对 Pareto 解集进行剪枝};
    \node (compare) [process, below=of prune] {对不同修改进行统计分析};
    \node (stop) [terminalStop, below=of start, yshift=-11.7cm] {结束};

    \draw [-latex] (start) -| (firstSampling);
    \draw [-latex] (firstSampling) -- (firstSimulation);
    \draw [-latex] (firstSimulation) -- (firstDatabase);
    \draw [-latex] (firstDatabase) -- (sa);
    \draw [-latex] (sa) -- (influential);
    \draw [-latex] (influential.east) -- (fixParam) node[pos=0.5, inner sep=0]{否};
    \draw [-latex] (influential.west) -- +(-1.5cm,0) node[pos=0.5, inner sep=0]{是} -- +(-1.5cm, 13cm) -| (secondSampling);
    \draw [-latex] (secondSampling) -- (secondSimulation);
    \draw [-latex] (secondSimulation) -- (secondDatabase);
    \draw [-latex] (secondDatabase) -- (moo);
    \draw [-latex] (moo) -- (prune);
    \draw [-latex] (prune) -- (compare);
    \draw (fieldData) -| (firstSimulation);
    \draw (fieldData) -| (secondSimulation);
    \draw [-latex] (fixParam) |- (secondSimulation);
    \draw [-latex] (compare) -- (stop);
\end{tikzpicture}



\section{实验观测}
试验区位于南疆地区阿拉尔市塔里木大学灌溉试验站 (\ang{81;11;46}E, \ang{40;37;28}N)。
灌溉实验按照田间实验设计,采用滴灌方式,根据 SWAP 模型\cite{swap2021}的精准灌溉工具,通过气象数据,土壤数据和棉花生长数据计算每日的实际蒸发蒸腾量 ($ET_c$),按照水分需求进行灌溉。
设计三个灌溉水平,标准亏损灌溉 75\% $ET_c$,90\% $ET_c$ 和 100\% $ET_c$。
另外,设计一个经验灌溉水平,灌溉定额 $\mathrm{280 m^3 / 667m^2}$,生育期滴水总计 14 次。
项目中不研究其他营养对棉花生长的影响,所以 N、P、K 肥按照经验值随滴灌施入,测土施肥,各小区施肥量统一,$N$-$P_2O_5$-$K_2O$ 按照 250-100-50 $\mathrm{kg/hm^2}$ 的量进行施肥。
播种日期设计 2 个水平,分别为铺设地膜的棉花播种后第 10 和 15 天,根据出苗率补充到设计的种植密度,并记录补苗日期。
第一年设计三个种植密度,理论密度分别为 15000, 16000 和 17000 株/$\mathrm{hm^2}$。
第二年到第三年根据第一年的模拟结果进行适当调整。
共 24 个处理,进行安全区组设计,每个小区 0.25 亩,每个实验三次重复。
\subsection{土壤含水量}

\subsection{棉花生理指标}
试验记录内容主要包括:
\begin{enumerate}
    \item 生育期:详细记录棉花播种、出苗、现蕾、开花和吐絮期出现时间,均以全田 50\% 棉株达到发育要求为标准;
    \item 株式图:每 10 天记录一次株式图,内容有株高、叶片数、果枝数、果节数、蕾、花、小铃、大铃及吐絮铃个数与部位等;
    \item 叶面积指数和光合有效辐射:每 10 天结合干物质测定,取全株叶片,用扫描法测定冠层垂直方向 20 个深度和水平 8 个网格的叶面积指数。使用 SunScan冠层分析仪(Delta 公司,英国)测定不同垂直和水平方向的光合有效辐射 PAR 和叶面积指数;
    \item 光合作用:每 10 天测试一次,使用 LI6400XT 便携式光合作用测试仪分层测试净光合速率、气孔导度、胞间 CO2 浓度和蒸腾速率以及光响应曲线;
    \item 干物质积累和分配:每 10 天取样一次,每次取样在苗期取 10 株样品,开花期后每次取 5 株,样品在 80℃下烘干至恒重后,分别测定根、茎、叶、蕾、花、小铃和大铃等各器官的干物质重量;
    \item 产量构成因素:测定棉花籽棉产量、衣分、铃重和纤维重量;
    \item 根系:每隔 10 天取样,在 0-100cm 土壤深度和距棉行 0-40cm 内各 5cm间隔分别用根钻取带根土样,冲根器冲洗根后,用扫描仪扫描成 TIF 图像文件,再采用 DT-SCAN 图像分析软件,计算根系长度、密度、直径和表面积等指标;
    \item 土壤水分:每周取样测定一次土壤容积含水量(20 $\times$ 8 网格),另外,使用矫正后的土壤水分和温度自动记录仪 (HOBO H21-002, United States) 分 10 层实时监测 0-100cm 深度的土壤水分含量;
    \item 土壤盐分(电导率):每周取样在实验室处理后采用 DDS-307 电导率仪测量,折算成土壤盐分含量;
    \item 其他土壤参数:土壤田间持水率、容积密度、枯萎点含水率、饱和土壤含水率、土壤水响应曲线和渗透系数等可直接取样带回实验室进行测量,同步记录灌溉日期和灌溉量;
    \item 土壤蒸发与棉花蒸腾:采用小型蒸渗仪测量蒸发,棉花蒸腾通过水量平衡原理计算;
    \item 气象数据:由塔里木灌溉试验站气象数据提供,包括 15 分钟间隔的温度、湿度、辐射、风速、降雨和气压数据等。
\end{enumerate}

所获数据用 Python 进行统计分析,利用 Duncan 新复极差检验进行差异显著检验 $(p = 0.05)$。

\subsection{棉花水分利用效率}
水分利用效率 (Water use efficiency, WUE) 是节水农业重要指标,高水平的 WUE 是干旱区农业持续发展的关键所在。
WUE 包括灌溉水利用效率,降水利用效率和作物水分利用效率。
由于研究区降水稀少,本文仅分析灌溉水利用效率和作物水分利用效率。

棉花水分利用效率 ($WUE_{ET}$) 和灌溉水分利用效率 ($IWUE$) 计算公式如下:

\begin{equation}
    WUE_{ET} = Y / ET_c
\end{equation}

\begin{equation}
    IWUE = (Y - Y_n) / I
\end{equation}

式中 $Y$ 为皮棉产量 ($\mathrm{kg\ ha^{-1}}$),
$Y_n$ 为无灌水条件下棉花皮棉产量 ($\mathrm{kg\ ha^{-1}}$),
$ET_c$ 为生育期棉花蒸散量,
$I$ 为灌水量 (mm)。
\subsection{棉花蒸散量计算}

各处理耗水量采用水量平衡公式计算:

\begin{equation}
    ET_c = R + I - F + Q - S + \Delta W
\end{equation}

式中 $ET_c$ 为作物蒸发蒸腾量 (mm),
$R$ 为降水量 (mm),
$I$ 为灌水量 (mm),
$F$ 为地表径流量 (mm),考虑到地下埋管滴灌棉田实验期间无地表径流发生,此处取 $F = 0$
$Q$ 为地下水补给量 (mm),实验资料表明 110 \~ 130 cm 深处土壤水分变化不明显,因此做简化处理,取 $Q = 0$,
$\Delta W$ 为土壤储水变化量 (mm),
$S$ 为深层渗漏量(mm),实验田灌水量通过水量平衡方程计算确定,不存在深层渗漏,故 $S=0$;
$\Delta W$ 为土壤贮水量的减少量(mm):

\begin{equation}
    \Delta W = W_i - W_{i+1}
\end{equation}

式中:$W_i$ 和 $W_{i+1}$ 分别为第 $i$ 个时段初和时段末计划湿润层内的土壤贮水量 (mm)。
为方便水量平衡计算,将含水率换算为以 mm 为单位的土壤贮水量 $W$ :

\begin{equation}
    W = 10 \cdot \theta \cdot \gamma \cdot h
\end{equation}

式中:$W$ 为土壤贮水量(mm),$\theta$ 为计划湿润层内土壤质量含水率 (\%),$\gamma$ 为土壤体积质量 ($\mathrm{g cm^{-3}}$),$h$ 为计划湿润层深度 (cm)。

\section{模型模拟}
\subsection{模型改进}

分析叶面积指数、叶倾角和方位角的空间分布规律及光能截获分布,冠层深度拟分为 20 层,每层占棉花总高度的 5\%,水平方向根据棉花冠幅考虑 8 个网格,共 160 个冠层空间。
首先,计算每一层的有效叶面积指数,
其次,定量化每一层与最顶部冠层的光合叶面积指数的比例($PAI_{\%,i}$),重新积分计算有效总光合叶面积指数,
最后,重新驱动 Cotton2K 模型,模拟干物质积累,每一层的 $PAI_{\%,i}$ 通过公式 \ref{eq:pai} 计算
\begin{equation}
    \label{eq:pai}
    PAI_{\%,i} = 1 - (1 - x_i^\alpha)^\beta
\end{equation}
$PAI_{\%,i}$ 是第 $i$ 层截获 $PAI$ 与最顶部冠层 $PAI$ 的比值,$x_i$ 是冠层深度的百分比,
研究中分 20 层,所以最顶层等于 0.05,最底层等于 1。$\alpha$ 和 $\beta$ 根据不同生长发育时期实际观测的每层 $PAI$ 和冠层顶部的 $PAI$ 确定,分段拟合。

首先,IRE 方法改进水分运移方程的上下边界条件,与根系吸水模型耦合,研究土体 100cm,厚度分层 5cm,水平方向间距 5cm。
其次,根据根长密度分布函数计算每个土层空间的根长密度,并使用 SWAP 模型推荐的方法计算潜在蒸发蒸腾量,分配到每一个土层空间。最后,以每小时时间步长应用 IRE 方法离散土壤蒸发、根系吸水和水盐运移过程。
具体步骤如下:


潜在作物蒸腾与土壤蒸发的计算如公式\ref{eq:E_p}和公式\ref{eq:T_p}
\begin{equation}
    \label{eq:T_p}
    T_p = \frac{
        (1 - W_{frac}) \{ V_c \frac{\Delta_v}{\lambda_w} (R_n - G) + \frac{p_1 \rho_a C_a}{\lambda_w} (\frac{e_{sat} - e_a}{r_{a,can}}) \}
    }{
        \Delta_v + \gamma_a(1 + \frac{ r_{s,min} }{ r_{a,can} LAI_{eff} })
    }
\end{equation}
\begin{equation}
    \label{eq:E_p}
    E_p = \frac{
        (1 - V_c) \frac{\Delta_v}{\lambda_w} (R_n - G)
        + \frac{p_1 \rho_a C_a}{\lambda_w} (\frac{e_{sat} - e_a}{r_{a,can}})
    }{
        \Delta_v + \gamma_a (1 + \frac{r_{soil}}{r_{a,soil}})
    }
\end{equation}

\subsection{模型参数化}
本文采用 2019-2020 连续 2 年的棉花实验数据对 Cotton2K 模型进行参数化。
其中作物生长参数如叶面积指数 (LAI),株高,干物质积累和主茎节数通过实验得到。
土壤参数如土壤水分、盐分初始值由实验得到;饱和导水率,Van-Genuchten
公式参数 $(\alpha, \beta)$ 等参数由率定得到;气象参数取自阿拉尔气象站;
作物参数和水分胁迫参数由率定得到。

\section{多目标优化}
\begin{algorithm}
    \caption{NSGA-III 第 t 代的过程}
    \KwIn{$H$ 构造的参考点 $Z^s$ 或提供的目标点 $Z^a$, 父代种群 $P_t$}
    \KwOut{$P_{t+1}$}

    $S_t = \emptyset, i = 1$\\
    $Q_t$ = 重组+变异($P_t$)\\
    $R_t = P_t \cup Q_t$\\
    $(F_1, F_2, \dots)$ = 非支配排序($R_t$)\\
    \Repeat{$|S_t \ge N|$}{$S_t = S_t \cup F_i$ 且 $i = i + 1$}
    包含父代的 Pareto 前沿: $F_l = F_i$\\
    \eIf{$|S_t| = N$}{
    $P_{t+1} = S_t$, break
    }{
    $P_{t+1} = \cup^{l-1}_{j=1} F_j$\\
    从 $F_l$ 中选点: $K = N - |P_{t+1}|$\\
    归一化目标并创建参考集 $Z^r$: $\mathtt{Normalize}(\mathrm{f}^n, S_t, Z^r, Z^s, Z^a)$\\
    Associate each member $\mathbf{s}$ of $S_t$ with a reference point: $[\pi(\mathbf{s}), d(\mathbf{s})] =\mathtt{Associate}(S_t, Z^r)$
    \tcc{$\pi(\mathbf{s})$: closest reference point, $d$: distance between s and $\pi(\mathbf{s})$}
    Compute niche count of reference point $j \in Z^r$: $\rho_j = \sum_{\mathbf{s}\in S_t/F_l}((\pi(\mathbf{s})) ? 1 : 0 )$\\
    Choose $K$ members one at a time from $F_l$ to construct $P_{t+1}$: $\mathtt{Niching}(K, \rho_j, \pi, d, Z^r, F_l, P_{t+1})$
    }
\end{algorithm}
