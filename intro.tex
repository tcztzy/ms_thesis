\chapter{绪论}\label{chap:intro}
\section{研究背景及意义}
新疆南部地区 (南疆) 日照时间长、光热资源丰富,为优质棉花的生长提供了绝佳的自然条件.
2021 年新疆棉花播种面积 2506.1 千公顷 (全国 3028.1 千公顷),产量 500.2 万吨 (全国 512.9 万吨) \cite{国家统计局关于2021年棉花产量的公告},是我国和世界最具发展前景的优质棉生产基地。
新疆从 1980 年代开始大范围推广使用地膜覆盖技术种植棉花,这一举措被誉为“白色革命”,大幅提高了棉花的生产效率,带来了丰厚的经济效益。%
但残留地膜回收率低,在土壤中难以降解,土壤中残膜量逐年增加,不仅对土壤结构造成破坏,而且对混入残膜的纤维品质也降低了,“白色革命”终成“白色污染”。
中国工程院院士喻树迅团队连续 7 年在南疆多地的试验示范站实现了棉花的无膜化种植,成功培育出了早熟品种“中棉 619”,%
可实现平均亩产 365 公斤\cite{yu2019}。
然而,无膜棉的种植与推广还有诸多问题需要解决,如播种密度和肥水调控等问题急需深入研究。
因此,在南疆干旱区开展无膜棉生长和水分运移模拟研究,%
对指导无膜棉播种日期、播种量、精准灌溉和提高产量具有重要意义。
有效光合叶面积指数发芽后集中在冠层的底部,成熟期集中在顶部,而花期属于均匀分布,%
本文将借鉴成熟的棉花生长模型理论,集中解决有效光合叶面积指数不同发育阶段在棉花冠层分布不一致的问题,提高光能截获和光合作用模拟精度。

\section{国内外研究现状}
\subsection{主要的棉花生长模型}
棉花 (\textit{Gossypium hirsutum L.} 与 \textit{Gossypium barbadense L.})是全球重要的商品作物,为不同行业提供纤维、饲料、食物和潜在的燃料来源。%
棉花纤维被用于从纺织品到纸张、咖啡过滤器和渔网等产品。%
棉籽粉和棉籽壳主要用于反刍牲畜饲料。棉籽油目前被提炼成植物油供人食用,并有可能成为生物燃料。%
我国棉花生产的一个主要问题是生产成本高,需要改进棉花生产方式,以保持棉花在经济上对其他商品作物和替代纤维来源的竞争力。%
为了使棉花生产具有可持续性,还必须考虑水和能源资源的限制。%
通过更明智的灌溉和氮肥管理,更好地了解气候对棉花产量的影响,进一步推进棉花育种和遗传学,更多地采用精准农业技术,以及增加对棉花遗传学与环境管理(GEM)相互作用的了解,可以实现这些改进棉花生产的目标。

作物生长模型以经验公式的形式将作物的生长发育、光合生产、器官建成和产量形成等过程描述出来,%
与环境与栽培管理有机结合,通过计算机程序对作物生长指标动态地进行定量描述。
棉花生产中种植系统模拟模型的开发和应用有着悠久而丰富的历史,从20世纪60年代在美国东南部开始,现已扩展到全球主要的棉花生产地区。%
目前主流的专用棉花生长模型主要有 GOSSYM\cite{baker1976},COTCO2\cite{wall1994},OZCOT\cite{hearn1994},CROPGRO-Cotton\cite{jones2003} 和 Cotton2K\cite{cotton2kv4} 等,%
另外,一些通用的作物生长模型也被用于模拟棉花生长,如 EPIC\cite{williams1989},WOFOST\cite{vanDiepen1989WOFOST},SUCROS\cite{vanittersum2003},GRAMI\cite{ko2005},CropSyst\cite{sommer2008}和 AquaCrop\cite{steduto2009}。
模型的应用领域包括作物用水和灌溉水管理、氮素动力学和肥料管理、遗传学和作物改良、气候学、全球气候变化、精准农业、模型与传感器数据整合、经济学和课堂教学。%
总的来说,在上个世纪重视棉花模型的开发,本世纪以来则越来越重视棉花模型的应用\cite{thorp2014}。%
尽管的最早棉花模型已经有 50 年的历史,但棉花模型之间的对比较少\cite{thorp2014}。%
越来越多的棉花模拟模型被非传统的作物建模者应用,他们不是训练有素的农学家,但希望使用模型进行广泛的经济或生命周期分析。%
尽管这一趋势表明人们对模型的兴趣越来越大,而且模型在各种应用中都有潜在的效用,但这就需要为特定的应用开发具有适当复杂性和易用性的模型,而且需要改进文件和教材来教育潜在的模型用户。%
空间尺度问题也越来越突出,因为最初为田间使用而开发的模型正被用于大面积的区域模拟。%
研究朝着模型与变速控制系统集成的高级目标稳步推进,该系统利用实时作物状态和环境信息,在空间和时间上优化作物投入的应用,同时也考虑潜在的环境影响、资源限制和气候预测。%
总的来说,审查表明在棉花模拟模型开发方面的努力停滞不前,但现有模型在各种研究领域的应用仍然强劲,并继续增长。%


\begin{table}
    \caption{现有棉花生长模拟模型基本信息}\label{tab:overview}
    \small
    \centering
    \begin{tabular}{cp{0.14\linewidth}cccp{0.22\linewidth}}
        \toprule
        名称               & 父代模型         & 编程语言 & 时间步长 & 核心引用                    & 支持决策工具           \\
        \midrule
        GOSSYM             & SIMCOTI SIMCOTII & Fortran  & 日       & \authoryearcite{baker1976}  & COMAX\cite{lemmon1986} \\
        Cotton2K           & GOSSYM CALGOS    & C++      & 小时     & \authoryearcite{cotton2kv4} & 无                     \\
        COTCO2             & KUTUN ALFALFA    & Fortran  & 小时     & \authoryearcite{wall1994}   & 无                     \\
        OZCOT              & SIRATAC          & C\#      & 日       & \authoryearcite{hearn1994}  & APSIM 生态\cite{APSIM} \\
        CSM-CROPGRO-Cotton & CROPGRO-Soybean  & Fortran  & 日       & \authoryearcite{jones2003}  & DSSAT                  \\
        \bottomrule
    \end{tabular}
\end{table}

这些模型通过模拟计算大气、土壤和植物之间的交互作用和水分、能量平衡来实现对棉花生长过程的模拟。%
然而,不同的作物生长模型在选择模拟这些过程的方法和实现细节存在一定差异 (表 \ref{tab:growdev})\cite{thorp2014},%
但主要过程都包括了物候学、光能截获、碳 (C) 同化、呼吸作用、器官形成、生物量积累与分配和胁迫影响等。

一些新发展的通用模型理论和方法也对棉花生长模拟的研究具有重要的推动作用。
由联合国粮食及农业组织 (FAO) 支持的 AquaCrop 模型是模拟水资源管理产量响应的新型通用作物模型\cite{tan2018}。
它基于植物生理学和土壤水分平衡的模拟,取代了粮农组织以前的方法,用于估算与供水有关的作物生产力。
另外,发展和参数化的 WALL 模型也通过聚焦水分在叶片运移用于仿真单叶的水分蒸发\cite{pachepsky2009}。
2021 年最新版本的 SWAP 4.0 版本\cite{swap2021}综合考虑了水、热、冷和盐分胁迫对蒸发蒸腾的影响,%
多尺度 SWAP 模型是否可以改进已提出的棉花模型的水分运移模拟值得深入分析和探讨。

中国作为世界上的主要棉花生产国,对棉花生长模拟的研究发展飞速,%
比较有代表性的是潘学标等开发的 COTGROW\cite{pan1996} 模型,%
此外,南京农业大学的曹卫星教授带领的科研团队\cite{zhang2003,ma2004,chen2006}也分别建立了棉花生育周期和品质形成模拟模型,%
这些模型相比国外的棉花生长模型,加入了如化控、覆膜等我国特有的管理措施。%
然而,由于生长模型的复杂性,对于不同的场景,不同模型之间的表现存在一定的差异性。

\begin{table}
    \small
    \caption{现有棉花生长模拟模型生长和发育阶段模拟}
    \label{tab:growdev}
    \begin{tabular}{p{0.14\linewidth}p{0.14\linewidth}p{0.14\linewidth}p{0.14\linewidth}p{0.14\linewidth}p{0.14\linewidth}}
        \toprule
                   & GOSSYM                                                                       & Cotton2K                                                                     & COTCO2                                                                   & OZCOT                                                                      & CROPGRO-Cotton                                                                                   \\
        \midrule
        物候学     & 根据积温发育叶枝果枝和坐果节,计算果枝、蕾、铃、开铃、坐果节和脱落果实的数量 & 根据积温发育叶枝果枝和坐果节,计算果枝、蕾、铃、开铃、坐果节和脱落果实的数量 & 根据积温发育分生组织、叶原基、叶柄,生长和成熟叶、节间茎段、蕾、铃和开铃 & 根据积温计算坐果节的数量,根据作物承载能力计算蕾、铃、开铃和脱落果实的数量 & 根据光热时间计算出苗、第一片叶、第一朵花、第一个种子、第一次开铃和 90\% 开铃、铃数和脱落果实数量 \\
        植株映射   & 有                                                                           & 有                                                                           & 有                                                                       & 无                                                                         & 无                                                                                               \\
        潜在碳同化 & 冠层尺度辐射截获                                                             & 冠层尺度辐射截获                                                             & 器官尺度生物化学驱动\cite{farquhar1980}                                  & 冠层尺度辐射截获                                                           & 叶片尺度生物化学驱动\cite{farquhar1980}                                                          \\
        呼吸作用   & 使用生物量和温度的经验函数计算                                               & 计算生长和维持呼吸,以及光合呼吸                                             & 计算器官尺度的生长、维持和光合呼吸                                       & 使用基于坐果节数和气温的经验函数                                           & 计算生长和维持呼吸                                                                               \\
        同化分配   & 分配碳同化产物到每个生长器官                                                 & 分配碳同化产物到每个生长器官                                                 & 分配碳同化产物到每个生长器官                                             & 将碳同化产物分配到用于棉铃发育的储存池                                     & 将碳同化产物分配到用于叶、茎、根和铃发育的单一存储池                                             \\
        冠层尺寸   & 计算高度                                                                     & 计算高度                                                                     & 计算茎节长度                                                             & 无                                                                         & 计算高度和宽度                                                                                   \\
        产量要素   & 根据棉铃的质量和尺寸计算纤维质量                                             & 计算纤维质量和种子棉质量                                                     & 计算棉铃质量                                                             & 根据棉铃的质量和尺寸计算纤维质量                                           & 计算棉铃质量、种子棉质量、种子数量和单位种子重量                                                 \\
        胁迫       & 计算水、氮和气温胁迫                                                         & 计算水、氮和气温胁迫                                                         & 计算水和气温胁迫                                                         & 计算水、氮和气温胁迫                                                       & 计算水、氮和气温胁迫                                                                             \\
        \bottomrule
    \end{tabular}
\end{table}

\begin{table}
    \small
    \caption{现有棉花生长模拟模型大气和土壤模拟}\label{tab:atmosoil}
    \begin{tabular}{p{0.14\linewidth}p{0.14\linewidth}p{0.14\linewidth}p{0.14\linewidth}p{0.14\linewidth}p{0.14\linewidth}}
        \toprule
                               & GOSSYM                           & Cotton2K                         & COTCO2                         & OZCOT                        & CROPGRO-Cotton                          \\
        \midrule
        $CO_2$对光合作用的影响 & 有                               & 有                               & 有                             & 无                           & 有                                      \\
        $CO_2$对呼吸作用的影响 & 无                               & 无                               & 有                             & 无                           & 有                                      \\
        蒸腾作用               & \authornumcite{ritchie1972}      & CIMIS 中的改进的 Penman 公式     & 叶片尺度能量平衡与气孔导度耦合 & \authornumcite{ritchie1972}  & \authornumcite{priestley1972,fao56}     \\
        土壤水分               & 2D RHIZOS 模型\cite{lambert1976} & 2D RHIZOS 模型\cite{lambert1976} & 2D 模型                        & \authornumcite{ritchie1972}  & \authornumcite{ritchie1998,ritchie2009} \\
        土壤氮                 & 土壤和植物氮平衡的动态仿真       & 土壤和植物氮平衡的动态仿真       & 无                             & 统计和经验方法预测潜在氮吸收 & \authornumcite{godwin1998,gijsman2002}  \\
        土壤磷                 & 无                               & 无                               & 无                             & 无                           & 有                                      \\
        土壤盐分               & 无                               & 有                               & 无                             & 无                           & 无                                      \\
        渍灾                   & 无                               & 无                               & 无                             & 有                           & 有                                      \\
        涝灾                   & 无                               & 无                               & 无                             & 无                           & 有                                      \\
        \bottomrule
    \end{tabular}
\end{table}
\begin{table}
    \caption{现有的棉花模拟模型和其他应用所模拟的管理实践}\label{tab:agromanagement}
    \begin{tabular}{p{0.14\linewidth}p{0.14\linewidth}p{0.14\linewidth}p{0.14\linewidth}p{0.14\linewidth}p{0.14\linewidth}}
        \toprule
                     & GOSSYM & Cotton2K & COTCO2 & OZCOT & CROPGRO-Cotton \\
        \midrule
        播种日期     & X      & X        & X      & X     & X              \\
        栽培品种选择 & X      & X        & X      & X     & X              \\
        行间距       & X      & X        & X      & X     & X              \\
        间隔行       & X      & X        &        & X     &                \\
        种植密度     & X      & X        & X      & X     & X              \\
        灌溉         & X      & X        & X      & X     & X              \\
        施肥         & X      & X        &        & X     & X              \\
        农作物残留物 &        &          &        &       & X              \\
        耕作         &        & X        &        &       & X              \\
        生长调节剂   & X      & X        &        &       &                \\
        落叶剂       & X      & X        &        &       &                \\
        虫害         & X      & X        & X      & X     & X              \\
        病害         &        & X        &        &       & X              \\
        气候变化     & X      &          & X      &       & X              \\
        耕作顺序     &        &          &        & X     & X              \\
        地理空间分析 &        & X        &        & X     & X              \\
        \bottomrule
    \end{tabular}
\end{table}
\subsection{棉花生长模型的应用研究}

棉花生长模型最基本的目的是预测产量,近几年,国外已经再水分利用效率评价和灌溉管理\cite{baumhardt2014,booker2014,booker2015,modala2015,thorp2015,anapalli2016,attia2016,linker2016,tsakmakis2018,thorp2019,thorp2020,thorp2020a}、
氮磷动态和施肥管理\cite{shumway2012,amin2017,arshad2017,zurweller2019}、%
气象变化响应\cite{abbas2020}、%
品质模拟\cite{ma2005}、%
打顶和生产管理\cite{yang2008}等方面开展了大量研究。
为了不同的研究目的,所选择的模型也存在较大的差异。
国内学者也在产量预测、模型性能评价、水分使用效率评价、品质成分含量模拟、株高生长模拟、水氮耦合效应评价、水盐运移、根系生长模拟和模型敏感性与不确定性分析等方面开展了研究和探索。
然而,在不同生长发育时期,冠层截获的光能在不同空间的分布比例存在一定的差异性和规律性,定量化描述这个过程有望提高光合作用模拟精度。

值得注意的是, GOSSYM 及其后继版本如 Cotton2K 是世界上最成功的棉花模型之一。
Cotton2K 模型基于 GOSSYM 模型的过程公式,借鉴了其他棉花生长模型的算法思想,以小时为单位的高时间分辨率计算水分、物质和能量平衡,%
提高了对干旱地区依赖灌溉的棉花生长模拟的精度和适用性。
虽然模型也考虑了盐分和水分在土壤的分布,然而,使用的 Richards 方程 \cite{richards1931} 计算水分运移过程存在上下边界粗糙的问题 \cite{thorp2014},
而且对于滴灌模式,沿滴灌带垂直方向的根系分布也应被精细的考虑;
另外,实际蒸腾计算过程中对冠层光截获因子的精细模拟也有改进的空间。

\subsection{Cotton2K 模型的应用研究}
Cotton2K 模型是由 Avishalom Marani 教授在美国做访问学者时开发的。
该模型的源代码是用 C++ 编写的,可以免费下载 \cite{cotton2kv4}。
Cotton2K 主要基于 GOSSYM \cite{baker1976,baker1983} 开发,同时也参考了其他棉花生长模型,包括SIMCOTI \cite{baker1972}、SIMCOTII \cite{jones1974}, 和 CALGOS \cite{marani1992a,marani1992b,marani1992c}。
开发 Cotton2K 的主要目的是为如美国西部和以色列这样的干旱、灌溉环境下的棉花生产提供一个更有用的模型。

\authoryearcite{cotton2kv4} 对Cotton2K的历史、主要特征、科学原理和投入要求作了总体描述。
Cotton2K 和 GOSSYM 的根本区别在于对天气数据的要求。
GOSSYM 使用每天的天气数据,而Cotton2K使用每小时的空气温度和湿度、风速和短波的测量值。
而 Cotton2K 使用的是空气温度和湿度、风速和短波辐照度的每小时测量值,或者使用 \authoryearcite{ephrath1996} 的方法,从每日数据中计算出小时值。
每小时的天气值被用来计算相应的每小时水和能量平衡。
这样可以使模型更贴近干旱条件,提高模型在灌溉条件下更准确计算水量平衡的能力\cite{cotton2kv4}。
这些变化的主要影响是提高了蒸发蒸腾的计算精度,同时也影响了相关变量。
此外,使用每日天气数据的时间步长,而不是较短的时间步长所产生的偏差,在在计算能量或水平衡天气参数的相互作用时尤为重要(例如非线性昼夜风速模式和/或风速与太阳辐照度的相互作用驱动蒸散发)\cite{ephrath1996}。
Cotton2K 的其他修改包括地表下滴灌的程序、N 矿化和硝化过程的更新、使用 Michaelis-Menten 程序计算 N 吸收、植物生长和物候功能的更新,以及提供土壤表面和作物冠层温度的能量平衡方程\cite{cotton2kv4}。
总之,增加每小时的天气输入数据后,可以按每小时的时间步长计算和整合微分方程,以实现以下过程 植物蒸腾、土壤水分蒸发、土壤水 再分配、热通量和氮通量,以及在土壤-植物-大气 (soil-plant-atmosphere) 界面的能量和水的交换。
这些修改大大提高了 Cotton2K 在干旱环境中灌溉的实用性和适用性。

在 Cotton2K 中计算过程主要与土壤、植物和环境之间的能量和水的交换有关。
这些过程是基于质量和能量守恒的原则,系统的输入和输出是平衡的,并作为时间的函数。
Cotton2K 模型是为特定的管理农艺投入,包括灌溉、氮肥、脱叶和应用植物生长调节剂。
植物生长和发育基于“胁迫”理论 \cite{grime1977,craine2005,grime2007},包括与空气温度、水、C和N有关的胁迫。
在此种语境下,胁迫是指由于空气温度不理想以及水和营养物质的短缺而限制潜在产量的条件\cite{grime1977}。
使用热单位的概念,植物生长率与环境温度有关\cite{wang1960,peng1989}。
所有植物器官,包括根、茎、叶片和叶柄,以及果实部位(花蕾,棉铃和籽棉)的潜在生长率,都通过胁迫因素与碳和水的源汇关系 (source-sink) 相关联。
源和汇之间的胁迫因子在数值上从1(无胁迫)到0(严重胁迫)不等。
碳胁迫与净碳同化有关(即总光合作用减去光呼吸、生长呼吸和维持呼吸)。
水胁迫与水的蒸腾和运输有关,是一个关于叶片水势的函数。
氮胁迫是基于氮的供应和需求。
在土壤中,Cotton2K 计算来自尿素水解、有机氮的矿化、铵的硝化、硝酸盐的反硝化和可溶性氮的移动的可用氮率。
并且,该模型还计算植物器官(根、茎、叶和果实部位)中的氮。
如果供应不能满足要求,则计算出氮胁迫因子。
所有的供应和需求函数与温度、水、碳和氮有关的所有供求函数都是动态的,且它们的值随时间变化。

在 Cotton2K 中定义一维土壤-植物-大气系统的边界条件是土壤表面以上2米和以下2米。
土壤表面以上的高度(2米)代表测量输入天气数据的屏幕高度,2米的土壤深度代表土壤剖面的下边界。
所需的输入天气数据包括短波辐照度、空气温度、湿度、风速和雨量。
Cotton2K 使用每小时的天气输入值;但是,如果没有每小时的数据,可利用辐射和风速以及空气温度和湿度的最大和最小值的日值计算小时值\cite{ephrath1996}。
对于每次灌溉事件,指定施用方法(喷灌、沟灌和滴灌)、时间(开始和结束)和施用深度。
用户通过指定每个土壤层的数量和厚度来定义土壤剖面的几何形状。
在模拟开始时(即时间=0),用户为每个土壤层指定一个温度、水、有机物、氮和土壤盐度的值。
此外,土壤层被分组为土层,每个土层都有独特的土壤水力特性。
这些属性定义了土壤含水量与水势和导水率的关系,并在 Richards 方程中用于计算土壤剖面中的水分流动。
用户指定地下水位深度和每个耕作事件的日期和深度。
其他固定的参数输入值是位置(纬度、经度和海拔),模拟期的开始和结束,种植和/或出苗日期,以及田间数据(种植密度和行距,包括间隔行)。
描述单个栽培品种的参数会影响物候、生长和发育,并最终影响棉花纤维产量的计算,正如 \authoryearcite{cotton2kv4} 所建议的和 \authoryearcite{booker2013}所展示。
当前版本的 Cotton2K 已经对 Acala SJ-2, GC-510, Maxxa, Deltapine 61, Deltapine 77, Sivon, 新陆早 8 号和中棉 619\cite{zhu2021} 八个棉花栽培品种进行了测试。

Cotton2K模型可以在管理模式下用于灌溉、氮、脱叶和应用生长调节剂。
在这些选项下,Cotton2K 使用预测的天气情景来执行。
用户可以选择几个选项,例如,开始和结束灌溉的日期,施用氮肥的日期,喷洒落叶剂日期,和喷洒植物生长调节剂 (Pix) 的应用。
Cotton2K 的输出被记录在文本文件、图表和土壤图 (soil map)。
文本文件是所有输入和输出值的总结,详细的每日输出和植物图 (plant map)。
图表显示了关键输出变量的动态变化。
而土壤图是二维的的土壤水和氮含量、温度和其他变量的水平和垂直模拟值的图,每个变量都是关于时间的函数。

Cotton2K模型已经被许多研究人员直接和间接地使用和测试。
\authoryearcite{clouse2006} 使用模拟退火法优化了 Cotton2K 的参数,在空间上标定了 Cotton2K 在西德克萨斯州的试验点的土壤参数。
校准后的模型被用来比较特定地点和统一灌溉管理策略。
特定地点灌溉管理的模拟棉纤维产量更高,但产量增加并没有使特定地点灌溉更有利可图。
\authoryearcite{haim2008} 使用Cotton2K模型对以色列的灌溉棉花进行了研究。报告说,在两种气候变化情况下,通过提前两周种植和增加灌溉,可以抵消气候变暖的负面影响。
\authoryearcite{yang2008} 使用了Cotton2K模型,在田间条件下测试了剪枝和打顶的效果,
\authoryearcite{yang2010} 和 \authoryearcite{nair2013} 在有限的水资源条件下优化灌溉分配。
在有限的水资源条件下优化灌溉分配。
\authoryearcite{yang2010} 使用Cotton2K 模型估计华北平原棉花的灌溉用水需求,使用20 年的农艺、水文和气候数据,估算华北平原棉花灌溉需水量。
平均而言,灌溉的棉花生产占该地区总需水量的8\%。
\authoryearcite{nair2011} 使用 Cotton2K 和一个经济模型生成的棉花纤维产量模拟,确定在不同次优灌溉水供应水平下,在棉花不同生长阶段分配灌溉水的经济最优策略。
Cotton2K 还被用来评估将中心枢纽灌溉的棉田划分为灌溉和雨养部分的盈利能力\cite{nair2013}。
这项研究表明,田间分割增加了赤字灌溉棉花的纤维产量和收益。
Cotton2K 模型与计量经济学模型一起使用,评估了棉花生产者对风险的态度对德克萨斯州 高原地区中心枢纽灌溉棉花的最佳灌溉水分配决策的影响 \cite{nair2011}。
结果表明,最佳灌溉水量分配既有增加利润的作用,也有降低风险的作用。
\authoryearcite{nair2013} 模拟德克萨斯高原的德克萨斯州普莱恩维尤市 110 年 68 种不同的灌溉处理的棉花纤维产量,对 Cotton2K 在德克萨斯州高地的应用进行了评估,以分析将中心支流灌溉的棉田划分为灌溉田和旱田的和盈利的影响。
通过对田间进行部分灌溉使得棉花纤维产量和盈利能力得到提高。当可用的灌溉水少和降雨量低的年份,经济效益更高。
\authoryearcite{wang2013} 开发了一个理论框架,用于确定缺水灌溉下棉花的经济最优灌溉水分配,并将该经济模型与 Cotton2K 生成的纤维产量数据一起使用,分析了旨在提高高效灌溉系统采用率的成本分摊计划的节水潜力。
他们得出结论,该计划没有为生产者提供任何节水的激励。
\authoryearcite{booker2013} 将 Cotton2K 纳入一个景观规模的模型中并将其应用于德克萨斯高原主要土壤类型的棉花生产中。
鉴于 Cotton2K 与 GOSSYM 和 CALGOS 模型的相似性,间接地, Cotton2K 中的一些算法已被评估用于广泛的土壤类型。
\authoryearcite{staggenborg1996} 对广泛的土壤和环境条件进行了评估。
\authoryearcite{clouse2006}, \authoryearcite{baumhardt2009} 等人对 Cotton2K 中的一些算法进行了评估。
\authoryearcite{baumhardt2014} 研究了在厄尔尼诺-南方振荡影响下的棉花纤维产量对灌溉管理的响应。
\authoryearcite{thorp2019} 开发了一套全新的方法用于比较模型中蒸发蒸腾算法的效果与性能,从 DSSAT 模型中借鉴 3 种算法,修改 Cotton2K 以替换原始的蒸发蒸腾算法。

上述的研究中基本都是应用 Cotton2K 模型进行分析,极少涉及对模型运行机制的讨论和修改,但最近新的层出不穷的作物生长模型越来越精细,%
Cotton2K 在这个方面亟待改进。

\subsection{发展动态和问题分析}
棉花生长模型的主要发展趋势是从局部到整体,由简单到复杂,由经验性到机理性,由智能化到数字化。%
另外,增强目标性和适用性,构建专门针对某一生产实际问题的专用模型更具有应用价值\cite{thorp2014}。

综上所述,为了服务中棉 619 在南疆地区的无膜栽培推广应用,本文以 Cotton2K 模型为基础,%
重点研究考虑冠层光合作用有效叶面积指数的垂直分布比例的光能截获模拟,以及考虑冠层光能截获、根系和盐分空间分布影响的水分运移模拟,%
建立适合南疆干旱、盐渍化土壤特点的棉花生长模型,
为无膜棉的播种日期、播种量的确定和灌溉制度提供定量化的分析手段。

\section{研究内容和技术路线}

\subsection{研究内容}
\begin{enumerate}[label=(\arabic*)]
    \item 对 Cotton2K 模型进行修改\\%
          参考 WOFOST-GTC 模型,将 Cotton2K 模型的冠层子模块修改为自上而下的层级结构,每层为 5 cm 高,共 20 层,分层对冠层的光合作用和干物质分配等生理过程进行模拟。%
          同时也修正部分原版模型的错误,例如浮点数计算溢出、干重模拟值为负值等。
    \item Cotton2K 模型的校准与验证\\%
          参考相关文献中对 Cotton2K 模型参数敏感性分析结果,利用 2019 年与 2020 年两年的田间实验数据,使用修改后的 Thorp 方法\cite{thorp2019} 分别对原版模型和修改后的模型的参数进行校准。%
          将 2021 年的田间实验数据作为验证集对校准后的 Cotton2K 进行验证,评估 Cotton2K 模型对中棉 619 的模拟精度。
    \item 中棉 619 冠层垂直方向光能截获时空分布规律\\%
          利用 2019 年至 2020 年生育期内采集的冠层光合有效辐射 (PAR) 数据,将棉花冠层分为 20 个等高的子层,研究棉花冠层垂直方向的的光能截获分布规律。
\end{enumerate}

\subsection{技术路线}

研究总技术路线如图 \ref{fig:roadmap} 所示。%
研究以 Cotton2K 模型为理论依据,利用中棉 619 的田间实验数据探究不同生育期光合有效辐射在冠层垂直方向分布的规律。%
\begin{figure}
    \centering
    \begin{tikzpicture}[
            node distance=1cm,
            every node/.style={fill=white, font=\sffamily},
            align=center
        ]
        % Specification of nodes (position, etc.)
        \node (start) [terminalStart] {开始};
        \node (firstSampling) [data, below left=of start, xshift=-3cm] {对 70 个 Cotton2K \\参数进行 Sobol 取样};
        \node (fieldData) [data, below=of start, yshift=-1.5cm] {田间数据};
        \node (firstSimulation) [process, below=of firstSampling, yshift=-1.5cm] {Cotton2K 模拟};
        \node (firstDatabase) [data, below=of firstSimulation] {12 个输出变量的\\评价指标数据库};
        \node (sa) [process, below=of firstDatabase] {使用 SALib 进行\\Sobol 全局敏感性分析};
        \node (influential) [decision, below=of sa] {S1 > 0.05};
        \node (fixParam) [process, below=of start, right=of influential] {固定参数为默认值};
        \node (secondSampling) [data, below right=of start, xshift=3cm] {对 35 个 Cotton2K \\参数进行多目标优化};
        \node (secondSimulation) [process, below=of secondSampling, yshift=-1.5cm] {Cotton2K 模拟};
        \node (secondDatabase) [data, below=of secondSimulation] {12 个输出变量的\\评价指标数据库};
        \node (moo) [process, below=of secondDatabase] {求解 Pareto 解集};
        \node (prune) [process, below=of moo] {对 Pareto 解集进行剪枝};
        \node (compare) [process, below=of prune] {对不同修改进行统计分析};
        \node (stop) [terminalStop, below=of start, left=of compare] {结束};

        \draw [-latex] (start) -| (firstSampling);
        \draw [-latex] (firstSampling) -- (firstSimulation);
        \draw [-latex] (firstSimulation) -- (firstDatabase);
        \draw [-latex] (firstDatabase) -- (sa);
        \draw [-latex] (sa) -- (influential);
        \draw [-latex] (influential.east) -- (fixParam) node[pos=0.5, inner sep=0]{否};
        \draw [-latex] (influential.west) -- +(-1.5cm,0) node[pos=0.5, inner sep=0]{是} -- +(-1.5cm, 13cm) -| (secondSampling);
        \draw [-latex] (secondSampling) -- (secondSimulation);
        \draw [-latex] (secondSimulation) -- (secondDatabase);
        \draw [-latex] (secondDatabase) -- (moo);
        \draw [-latex] (moo) -- (prune);
        \draw [-latex] (prune) -- (compare);
        \draw (fieldData) -| (firstSimulation);
        \draw (fieldData) -| (secondSimulation);
        \draw [-latex] (fixParam) |- (secondSimulation);
        \draw [-latex] (compare) -- (stop);
    \end{tikzpicture}
    \caption{技术路线图}\label{fig:roadmap}
\end{figure}

\subsection{论文结构}
本论文共 6 章,各章节内容简介如下:

第 \ref{chap:intro} 章,主要介绍了无膜栽培棉花的研究背景及意义、棉花生长模型的研究进展以及该论文的研究内容、技术路线以及论文结构。

第 2 章,主要从试验观测、模型模拟、敏感性分析和多目标优化角度介绍本研究采用的材料与方法。

第 3 章,主要介绍了对 Cotton2K 模型的改进,修改了模型的输入输出、修正部分计算、优化模拟运行效率、跨操作系统编译以及增加分层的冠层子模块。

第 4 章,主要介绍了修改前后的 Cotton2K 模型的敏感性分析结果。

第 5 章,主要介绍了 Cotton2K 的校准与验证、模型修改前后的对比和棉花生长模拟结果。

第 6 章,总结与展望。总结了实验的内容和实验结果,阐述了实验中的不足以及未来的改进方案。

\section{本章小结}
本章首先介绍了棉花在世界和中国经济发展中的重要地位。%
然后阐述了棉花的地膜栽培技术在给我国农业发展带来的巨大经济效益的同时,也给我国带来了残膜污染的问题。%
为了保护和改善农业生态环境,减少农用薄膜是切实可行的手段。%
2017 年末,中国工程院喻树迅院士率领的科研育种团队成功培育出中棉 619,该品种适用于无膜种植。%
随后详细论述了 Cotton2K 模型的发展历史和主要特色。
最后介绍了本文的研究内容、技术路线以及论文结构。
