\chapter{Cotton2K 模型的改进}

\section{模型输入输出}

原版的 Cotton2K 模型输入输出是基于特定格式的文件的。考虑到模型的建立处于关系型数据库还未流行的年代,这种基于特定格式文件的输入输出方式是可以理解的。
但这样的输入输出方式不利于 Cotton2K 结合到以 Python 为核心的现代化机器学习的生态中。
所以本文对 Cotton2K 进行了重构,修改输出输出为结构化数据,方便与其他机器学习工具结合。
具体的输入格式以 JSON Schema 的格式定义,参见附录\ref{appendix:inputSchema},在此不再赘述。

输出格式修改为 CSV 格式,改变模型代码以直接输出与土钻取土位置相对应的土壤含水量,全部字段见表 \ref{tab:output}。

\begin{table}
    \caption{修改后的 Cotton2K 模型输入}\label{tab:output}
    \centering
    \begin{tabular}{lll}
        \toprule
        字段名称                          & 含义           & 单位      \\
        \midrule
        \texttt{date}                     & 日期           & 天        \\
        \texttt{light\_interception}      & 光能截获率     & \%        \\
        \texttt{plant\_height}            & 株高           & cm        \\
        \texttt{leaf\_area\_index}        & 叶面积指数     & 1         \\
        \texttt{lai00} 到 \texttt{lai19}  & 分层叶面积指数 & 1         \\
        \texttt{lint\_yield}              & 皮棉产量       & kg/hm$^2$ \\
        \texttt{seed\_cotton\_yield}      & 籽棉产量       & kg/hm$^2$ \\
        \texttt{leaf\_weight}             & 叶干物质量     & kg/hm$^2$ \\
        \texttt{petiole\_weight}          & 叶柄干物质量   & kg/hm$^2$ \\
        \texttt{stem\_weight}             & 茎干物质量     & kg/hm$^2$ \\
        \texttt{square\_weight}           & 方铃干物质量   & kg/hm$^2$ \\
        \texttt{boll\_weight}             & 铃干物质量     & kg/hm$^2$ \\
        \texttt{root\_weight}             & 根干物质量     & kg/hm$^2$ \\
        \texttt{plant\_weight}            & 地上总干物质量 & kg/hm$^2$ \\
        \texttt{main\_stem\_nodes}        & 主茎节点数     & 个        \\
        \texttt{number\_of\_squares}      & 方铃个数       & 个        \\
        \texttt{number\_of\_green\_bolls} & 绿铃个数       & 个        \\
        \texttt{number\_of\_open\_bolls}  & 开铃个数       & 个        \\
        \texttt{swc}                      & 土壤含水量     & \%        \\
        \bottomrule
    \end{tabular}
\end{table}

\section{修正部分计算}
一些参数集在运行时引起了下溢或溢出错误,这就需要对编码进行编辑,以限制某些状态变量的范围。
例如,在 Cotton2K 中新叶生长的过程中,新叶的质量是固定的,且是由茎的质量转移而来,在极端的情况下,叶片生长速度过快会导致茎干重变为负数。

\section{跨操作系统编译}
\authoryearcite{thorp2019} 使用了 PALMScot 景观尺度棉花建模工具开发者维护的 Fortran 版本的 Cotton2K,主要是因为新版的 Cotton2K 高度依赖%
Microsoft Windows 平台的 GUI 框架 MFC。本文作者通过将 Cotton2K 完全使用 Python 重构,成功解决了上述与 Microsoft Windows 平台绑定的问题。

\section{全新的冠层子模块}
Cotton2K 模型中的光能截获率计算是从 GOSSYM 模型中演进而来。
计算光能截获率需要先计算两个因子:\begin{enumerate*}
    \item $z$ 因子与
    \item $l$ 因子
\end{enumerate*}。

$z$ 因子与株高和行间距比值成正比,参见公式 \ref{eq:z}。

\begin{equation}\label{eq:z}
    z = 1.0756 * H / ROWSPC
\end{equation}
式中 $H$ 为株高,单位是 cm, $ROWSPC$ 是行间距,单位是 cm。

$l$ 因子在叶面积小于 0.5 时时一个关于叶面积的线性函数,在其他情况时是一个指数函数。参见公式 \ref{eq:l}。

\begin{equation}\label{eq:l}
    l = \begin{cases}
        0.8 * LAI                 & LAI \le 0.5 \\
        1 - e^{0.07 - 1.16 * LAI} & LAI > 0.5
    \end{cases}
\end{equation}
式中 $LAI$ 为叶面积指数。

当 $l$ 因子大于 $z$ 因子时,光能截获率是两者的平均值,在 $l$ 因子小于 $z$ 因子且当前叶面积指数小于最大叶面积指数
(通常是发生了叶片的脱落) 时,为 $l$ 因子。在其他情况下为 $z$ 因子。参见公式 \ref{eq:li}。

\begin{equation}\label{eq:li}
    LI = \begin{cases}
        \frac{l + z}{2} & l > z                   \\
        l               & l\le z \, LAI<LAI_{\max} \\
        z               & \text{其他情况}
    \end{cases}
\end{equation}

受 WOFOST-GTC\cite{WOFOSTGTC} 启发,对模型进行修改,冠层的生长过程自顶向下,分层模拟。
每层层高 5 cm,共 20 层,模拟冠层高度最高可达 1 米。
叶片按展开时间和几何拓扑关系被分配到特定的层级中。
自顶向下逐层进行光合作用过程的模拟。
最终,光合作用产物的量为各层产物的量之和。

\begin{equation}
    LI_i = 1 - e^{p_i * LAI_i}
\end{equation}

\begin{equation}
    LI = 1 - \prod^{n}_{i=1}e^{p_i * LAI_i}
\end{equation}
式中 $LI$ 是总光能截获率, $n$ 总冠层层数, $LI_i$ 是第 $i$ 层的光能截获率,$p_i$ 是第 $i$ 层的光能截获率公式参数,
$LAI_i$ 是第 $i$ 层的叶面积指数。

\begin{equation}%
    P_{std} = 2.3908 + 1.37379 W + (-0.00054136) W^2%
\end{equation}%
式中 $P_{std}$ 是理论光合作用产物的量, $W$ 是地球接受到的辐射总量。%

\begin{equation}%
    P_{plant} = \sum^n_{i=1} P_{std} \times LI_i \times S_{plant} \times P_{tsred} \times P_{netcor} \times P_{tnfac} \times P_{agei}%
\end{equation}%
式中 $P_{plant}$ 是实际光合作用产物的量, $S_{plant}$ 每株棉花占地面积,单位是 $\mathrm{dm^2}$,%
$P_{tsred}$ 是水分胁迫对光合作用效率的影响,%
$P_{netcor}$ 是空气中 $\mathrm{CO_2}$ 含量对总光合作用效率的校正因子,%
$P_{tnfac}$ 低叶片含氮量下的校正因子,%
$P_{agei}$ 叶龄校正因子。
