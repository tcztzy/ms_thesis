\documentclass[a4paper,oneside,zihao=5,AutoFakeBold,fontset=windows]{ctexbook}
\usepackage[ruled,linesnumbered]{algorithm2e}
\usepackage{amsmath}
\usepackage{amssymb}
\usepackage[backend=biber,style=gb7714-2015]{biblatex}% chktex 8
\usepackage{booktabs}
\usepackage{caption}
\usepackage[inline]{enumitem} % inline enumerate
\usepackage{fancyhdr}
\usepackage{fontspec}
\usepackage{geometry}
\usepackage{hyperref}
\usepackage{listings}
\usepackage{longtable}
\usepackage{lscape} % one page table
\usepackage{multirow}
\usepackage{pgfplots}
\usepackage{setspace}
\usepackage{siunitx}
\usepackage{tikz}
\usepackage{tikz-3dplot}
\usepackage{xcolor}
\usepackage{xfp}
\newcommand{\titleCN}{南疆盐渍区无膜栽培棉花生长模拟}
\newcommand{\titleEN}{Simulation of non{-}mulched cultivated cotton growth in saline areas of South Xinjiang}
\newcommand{\authorCN}{唐梓涯}

% 页边距设置
\geometry{left=2.5cm,right=2.5cm,top=3cm,bottom=2.5cm}
% 页眉设置
\lhead{\kaishu 塔里木大学硕士学位论文}
% 字体设置,华文行楷和华文中宋只在封面使用
\newCJKfontfamily{\huawenxingkai}[AutoFakeBold = {1.1}]{STXingkai}
\newCJKfontfamily{\huawenzhongsong}{STZhongsong}
% 设置英文字体,中文字体在 documentclass 的 option 设置过了
\setmainfont{Times New Roman}

\SetAlgorithmName{算法}{算法列表}

% 导入参考文献库
\addbibresource{thesis.bib}
% 参考文献的导入方式
\newcommand{\authoryearcite}[1]{\citeauthor{#1} (\citeyear{#1})}
\ctexset{
  chapter = {
    beforeskip = 0pt,
    afterskip = 0pt,
    number = \arabic{chapter},
    format = \zihao{3} \heiti \bfseries \centering
   },
  section/format = \zihao{4} \songti \bfseries \raggedright,
  section/beforeskip = 0pt,
  section/afterskip = 0pt,
  subsection/format = \zihao{-4} \youyuan \bfseries,
  subsection/beforeskip = 0pt,
  subsection/afterskip = 0pt,
  % 要求中四级标题是用的 “仿宋 GB2312”,Windows 7 以上电脑并不内置该字体
  % 考虑到四级标题很少使用,这里直接使用 “仿宋” 字体
  subsubsection/format = \zihao{-4} \fangsong \bfseries
}

\usetikzlibrary{shapes, arrows.meta, positioning}
\tikzset{%
>={Latex[width=2mm,length=2mm]},
% Specifications for style of nodes:
base/.style = {draw, minimum width=2cm, minimum height=1cm, text centered},
terminal/.style = {base, rounded rectangle},
terminalStart/.style = {terminal},
terminalStop/.style = {terminal},
process/.style = {base, rectangle},
data/.style = {base, trapezium, trapezium left angle=60, trapezium right angle=120},
decision/.style = {base, diamond, aspect=2},
}
% \lstset{
%   basicstyle          =   \sffamily,          % 基本代码风格
%   keywordstyle        =   \bfseries,          % 关键字风格
%   commentstyle        =   \rmfamily\itshape,  % 注释的风格,斜体
%   stringstyle         =   \ttfamily,  % 字符串风格
%   flexiblecolumns,                % 别问为什么,加上这个
%   numbers             =   left,   % 行号的位置在左边
%   showspaces          =   false,  % 是否显示空格,显示了有点乱,所以不现实了
%   numberstyle         =   \zihao{-5}\ttfamily,    % 行号的样式,小五号,tt等宽字体
%   showstringspaces    =   false,
%   captionpos          =   t,      % 这段代码的名字所呈现的位置,t指的是top上面
%   frame               =   lrtb,   % 显示边框
% }
% \lstdefinestyle{JSONSchema}{
%   language        =   Python, % 语言选Python
%   basicstyle      =   \zihao{-5}\ttfamily,
%   numberstyle     =   \zihao{-5}\ttfamily,
%   keywordstyle    =   \color{blue},
%   keywordstyle    =   [2] \color{teal},
%   stringstyle     =   \color{magenta},
%   commentstyle    =   \color{red}\ttfamily,
%   breaklines      =   true,   % 自动换行,建议不要写太长的行
%   columns         =   fixed,  % 如果不加这一句,字间距就不固定,很丑,必须加
%   basewidth       =   0.5em,
% }
\begin{document}

% 图表标题设置,主要为了图表标题后无冒号,改用空格分隔
\captionsetup[figure]{name={图},labelsep=space}
\captionsetup[table]{name={表},labelsep=space}
% 算法中文化
\SetKwInput{KwIn}{输入}
\SetKwInput{KwOut}{输出}
\SetKwRepeat{Repeat}{重复}{直到}
\pagestyle{empty}
\frontmatter
\begin{titlepage}
    \begin{spacing}{1.1}
        \noindent
        \makebox[50pt][l]{\makebox[3em][s]{分类号}:}\makebox[243.4pt][l]{S562;S162.54}\makebox[63pt][s]{\makebox[4em][s]{单位代码}:}\makebox[96.6pt][l]{10757}\\
        \makebox[50pt][l]{\makebox[3em][s]{密\hspace{\fill}级}:}\makebox[243.4pt][l]{公开}\makebox[63pt][s]{\makebox[4em][s]{学\hspace{\fill}号}:}\makebox[96.6pt][l]{10757193097}\\
    \end{spacing}
    \vspace*{30pt}
    \begin{center}
        {\makebox[389pt][s]{\fontsize{65pt}{0}{\ \textbf{\huawenxingkai{塔里木大学}}}}}\\
        \vspace*{26pt}
        {\zihao{-2}TARIM\quad\ \ UNIVERSITY}
        \vspace*{58pt}

        \makebox[222pt][s]{\heiti\zihao{1}硕士学位论文}
        \vspace*{38pt}

        %{\heiti\zihao{-2}南疆盐渍区无膜栽培棉花生长模拟}
        {\zihao{-2}\heiti{利用分子标记辅助选择改良珍汕 {\heiti{97}} 对稻瘟病的抗性}}
        \vspace*{3pt}

        %{\zihao{-3}Simulation of non-mulched cultivated cotton growth in saline areas of South Xinjiang}
        {\zihao{-3}\textbf{Improverment of Rice Blast Diease Resistance of \textit{zhenshan} 97 by Molecular Marker-assisted Selection}}

        \vspace*{60pt}
        \zihao{4}
        \renewcommand\arraystretch{1.16666}
        \begin{tabular*}{387pt}{p{135pt} p{252pt}}
            \huawenzhongsong\makebox[7em][s]{研究生姓}名:                 & \\
            \huawenzhongsong\makebox[6em][s]{指导教}\makebox[2em][r]{师}: & \\
            %\huawenzhongsong\makebox[7em][s]{合作指导教}师:               & \\
            \huawenzhongsong\makebox[8em][s]{申请学位门类级别}:           & \\
            \huawenzhongsong\makebox[6em][s]{专业名}\makebox[2em][r]{称}: & \\
            \huawenzhongsong\makebox[6em][s]{研究方}\makebox[2em][r]{向}: & \\
            \huawenzhongsong\makebox[6em][s]{所在学}\makebox[2em][r]{院}: &
        \end{tabular*}

        \vfill
        新疆·阿拉尔

        二〇二二年六月
        \vspace*{40pt}
    \end{center}
\end{titlepage}

\chapter*{摘要}
无膜栽培技术是解决南疆地区棉田残膜污染的有效手段。
项目拟通过模型模拟定量评价种植密度和灌溉制度对无膜棉生长和产量的影响,
以 Cotton2K 模型为基础,重点解决棉花生长模拟过程中的两个关键科学问题:
\begin{enumerate*}
    \item 棉花不同发育阶段有效光合叶面积指数在冠层的空间分布不一致。
    \item 光能截获、盐分和根系分布对水分运移的耦合影响和定量描述。
\end{enumerate*}
首先,综合考虑热效应和光周期效应模拟物候学发育时间;
其次,建立有效面积指数在冠层空间分布比例与生理发育时间的分段函数方程,改进光能截获模拟精度,构建基于过程的棉花光合生产与干物质积累模型;
再次, IRE 方法改进水分运移方程上下边界条件,采用根系吸水与根长密度分布统一求解的方法,分层计算盐分和根系分布对水分运移的影响,建立考虑冠层光能截获、根系和盐分空间分布的水分运移模型;
最后,量化棉花各器官干物质分配、蕾铃发育与脱落过程,构建棉花干物质分配与产量形成的模拟模型。

\chapter*{Abstract}
Non-mulch planting technology is an effective mean to solve the residual plastic mulch pollution in cotton (\textit{Gossypium hirsutum} L.) fields in southern Xinjiang.
The impacts of irrigation management on the growth development and yield of non-mulch cotton were quantitatively evaluating by employing a cotton growth simulating system in this research.
The spatio-temporal distribution of the effective photosynthetic leaf area index in the canopy at different developmental stages varies strongly.
The Cotton2K model with modification in canopy submodel will be referenced as a basic theoretical model, and calibrated using field experiments data during the two cotton growing seasons of 2019 and 2020.
Firstly, in order to improve the simulation accuracy of light energy interception, a stratified model for spatial distribution ratio of effective area index in canopy and physiological development time was built.
Secondly, the process-based cotton photosynthetic production and dry matter accumulation models will be created.
Finally, the dry matter distribution, the number of branches, squares, bolls, open bolls, fruiting sites and aborted fruits will be calculated, thereby establishing a cotton growth model.
The decline of normalized root mean squared error (NRMSE) for canopy related metrics shows that modifications significantly ($p < 0.05$) improved the accuracies of simulation.

% 目录首页不需要页码,但默认目录首页是plain的,所以用empty覆盖
\fancypagestyle{plain}{
  \pagestyle{empty}
}
\tableofcontents
\newpage
\mainmatter
% 章节首页也需要页眉,但默认章节首页是plain的,所以用fancy覆盖
\fancypagestyle{plain}{
  \pagestyle{fancy}
}
\pagestyle{fancy}
\setcounter{page}{1}
\begin{spacing}{\fpeval{25 / (12 * 1.2)}}
  \chapter{绪论}
\section{研究背景及意义}
新疆南部地区 (南疆) 日照时间长、光热资源丰富,为优质棉花的生长提供了绝佳的自然条件.
2021 年新疆棉花播种面积 2506.1 千公顷 (全国 3028.1 千公顷),产量 500.2 万吨 (全国 512.9 万吨) \cite{国家统计局关于2021年棉花产量的公告},是我国和世界最具发展前景的优质棉生产基地。
上世纪 80 年代以来,新疆在棉花种植上大范围推广使用地膜覆盖技术,给新疆农业增产、农民增收带来了巨大效益。
但随着地膜投入量的不断增加,残留地膜回收率低,土壤中残膜量逐步增加,不仅造成土壤结构破坏、环境污染等一系列问题,而且对棉花产量和纤维品质也有很大影响。
中国工程院院士喻树迅团队初步实现了 “中棉 619” 早熟品种在南疆地区的无膜化种植目标,连续 6 年亩产 320{-}350 公斤,对减少新疆棉田残膜对生态环境和原棉的污染意义重大\cite{yu2019}。
然而,无膜棉的种植与推广还有诸多问题需要解决,如播种密度和肥水调控等问题急需深入研究。
因此,在南疆干旱区开展无膜棉生长和水分运移模拟研究,
对指导无膜棉播种日期、播种量、精准灌溉和提高产量具有重要意义。
项目将借鉴成熟的棉花生长模型理论,重点解决以下关键科学问题:
第一,有效光合叶面积指数发芽后集中在冠层的底部,成熟期集中在顶部,而花期属于均匀分布,解决有效光合叶面积指数不同发育阶段在棉花冠层分布不一致的问题,提高光能截获和光合作用模拟精度。
第二,在南疆盐渍化严重的背景下,光能截获、盐分和根系空间分布对水分运移的耦合影响和定量描述。
应用价值:通过模型模拟方法定量评价播种量和灌溉制度对产量的影响,指导播种和精准灌溉。
\section{国内外研究现状}
\subsection{主要的棉花生长模型}
作物生长模型通过数学方程将作物的生长发育、光合生产、器官建成和产量形成等过程及其所处环境和栽培管理技术体系连接成为一个整体,
通过计算机定量计算并进行动态模拟,成为掌握作物生长发育状况,优化种植管理的重要手段。
棉花生产系统模拟模型的开发与应用始于 1960 年代的美国,现在已经扩展到全球主要棉花生产地区。
国外发展比较成熟的棉花生长模型包括 GOSSYM\cite{baker1976},Cotton2K\cite{cotton2kv4},COTCO2\cite{wall1994},OZCOT\cite{hearn1994} 和 CROPGRO-Cotton\cite{jones2003},
另外,一些通用的作物生长模型也被用于模拟棉花生长,如 EPIC\cite{williams1989},WOFOST\cite{vanDiepen1989WOFOST},SUCROS\cite{vanittersum2003},GRAMI\cite{ko2005},CropSyst\cite{sommer2008}和 AquaCrop\cite{steduto2009}。

\begin{table}
    \caption{现有棉花生长模拟模型基本信息}\label{tab:overview}
    \small
    \centering
    \begin{tabular}{cp{0.14\linewidth}cccp{0.22\linewidth}}
        \toprule
        名称               & 父代模型         & 编程语言 & 时间步长 & 核心引用                  & 支持决策工具           \\
        \midrule
        GOSSYM             & SIMCOTI SIMCOTII & Fortran  & 日       &                           & COMAX\cite{lemmon1986} \\
        Cotton2K           & GOSSYM CALGOS    & C++      & 小时     &                           & 无                     \\
        COTCO2             & KUTUN ALFALFA    & Fortran  & 小时     &                           & 无                     \\
        OZCOT              & SIRATAC          & C\#      & 日       & \authornumcite{hearn1994} & APSIM 生态\cite{APSIM} \\
        CSM-CROPGRO-Cotton & CROPGRO-Soybean  & Fortran  & 日       &                           & DSSAT                  \\
        \bottomrule
    \end{tabular}
\end{table}

这些模型通过模拟气象、土壤水分和养分对植物生长发育的贡献来估算作物产量。
然而,用于模拟这些过程的方法、模拟细节和产量组分在现有作物模型中存在一定差异 (表~\ref{tab:growdev})\cite{thorp2014}。
但主要过程都包括了物候学、光能截获、碳 (C) 同化、呼吸作用、器官形成、生物量积累与分配和胁迫影响等。

一些新发展的通用模型理论和方法也对棉花生长模拟的研究具有重要的推动作用。
由联合国粮食及农业组织 (FAO) 支持的 AquaCrop 模型是模拟水资源管理产量响应的新型通用作物模型\cite{tan2018}。
它基于植物生理学和土壤水分平衡的模拟,取代了粮农组织以前的方法,用于估算与供水有关的作物生产力。
另外,发展和参数化的 WALL 模型也通过聚焦水分在叶片运移用于仿真单叶的水分蒸发\cite{pachepsky2009}。
2021 年最新版本的 SWAP 4.0 版本\cite{swap2021}综合考虑了水、热、冷和盐分胁迫对蒸发蒸腾的影响,多尺度 SWAP 模型是否可以改进已提出的棉花模型的水分运移模拟值得深入分析和探讨。

虽然中国的棉花生长模型研究起步较晚,但作为棉花生产世界领先的国家,
棉花生长模拟的研究发展较快,比较有代表性的是潘学标等开发的 COTGROW\cite{pan1996} 模型,此外,\authornumcite{zhang2003}、\authornumcite{chen2006}和\authornumcite{ma2004}等也分别建立了棉花生长和品质形成模拟模型,这些模型在借鉴国外模型理念基础上,加入了我国特有的管理措施如化控、覆膜等。
另外,国内学者也开发了棉花发育阶段和蕾铃模拟模型\cite{ma2005}、棉籽生长、油和蛋白质含量模拟模型\cite{li2009}和基于气温、太阳辐照度和 N 效应的棉纤维长度和强度的模拟模型\cite{zhao2012}等。
然而,不同的模型的性能会随着研究的品种、种植模式、生长环境和使用目标的变化呈现一定的差异性。

\begin{table}
    \small
    \caption{现有棉花生长模拟模型生长和发育阶段模拟}
    \label{tab:growdev}
    \begin{tabular}{p{0.14\linewidth}p{0.14\linewidth}p{0.14\linewidth}p{0.14\linewidth}p{0.14\linewidth}p{0.14\linewidth}}
        \toprule
                   & GOSSYM                                                                       & Cotton2K                                                                     & COTCO2                                                                   & OZCOT                                                                      & CROPGRO-Cotton                                                                                   \\
        \midrule
        物候学     & 根据积温发育叶枝果枝和坐果节,计算果枝、蕾、铃、开铃、坐果节和脱落果实的数量 & 根据积温发育叶枝果枝和坐果节,计算果枝、蕾、铃、开铃、坐果节和脱落果实的数量 & 根据积温发育分生组织、叶原基、叶柄,生长和成熟叶、节间茎段、蕾、铃和开铃 & 根据积温计算坐果节的数量,根据作物承载能力计算蕾、铃、开铃和脱落果实的数量 & 根据光热时间计算出苗、第一片叶、第一朵花、第一个种子、第一次开铃和 90\% 开铃、铃数和脱落果实数量 \\
        植株映射   & 有                                                                           & 有                                                                           & 有                                                                       & 无                                                                         & 无                                                                                               \\
        潜在碳同化 & 冠层尺度辐射截获                                                             & 冠层尺度辐射截获                                                             & 器官尺度生物化学驱动\cite{farquhar1980}                                  & 冠层尺度辐射截获                                                           & 叶片尺度生物化学驱动\cite{farquhar1980}                                                          \\
        呼吸作用   & 使用生物量和温度的经验函数计算                                               & 计算生长和维持呼吸,以及光合呼吸                                             & 计算器官尺度的生长、维持和光合呼吸                                       & 使用基于坐果节数和气温的经验函数                                           & 计算生长和维持呼吸                                                                               \\
        同化分配   & 分配碳同化产物到每个生长器官                                                 & 分配碳同化产物到每个生长器官                                                 & 分配碳同化产物到每个生长器官                                             & 将碳同化产物分配到用于棉铃发育的储存池                                     & 将碳同化产物分配到用于叶、茎、根和铃发育的单一存储池                                             \\
        冠层尺寸   & 计算高度                                                                     & 计算高度                                                                     & 计算茎节长度                                                             & 无                                                                         & 计算高度和宽度                                                                                   \\
        产量要素   & 根据棉铃的质量和尺寸计算纤维质量                                             & 计算纤维质量和种子棉质量                                                     & 计算棉铃质量                                                             & 根据棉铃的质量和尺寸计算纤维质量                                           & 计算棉铃质量、种子棉质量、种子数量和单位种子重量                                                 \\
        胁迫       & 计算水、氮和气温胁迫                                                         & 计算水、氮和气温胁迫                                                         & 计算水和气温胁迫                                                         & 计算水、氮和气温胁迫                                                       & 计算水、氮和气温胁迫                                                                             \\
        \bottomrule
    \end{tabular}
\end{table}

\begin{table}
    \small
    \caption{现有棉花生长模拟模型大气和土壤模拟}\label{tab:atmosoil}
    \begin{tabular}{p{0.14\linewidth}p{0.14\linewidth}p{0.14\linewidth}p{0.14\linewidth}p{0.14\linewidth}p{0.14\linewidth}}
        \toprule
                               & GOSSYM                           & Cotton2K                         & COTCO2                         & OZCOT                        & CROPGRO-Cotton                          \\
        \midrule
        $CO_2$对光合作用的影响 & 有                               & 有                               & 有                             & 无                           & 有                                      \\
        $CO_2$对呼吸作用的影响 & 无                               & 无                               & 有                             & 无                           & 有                                      \\
        蒸腾作用               & \authornumcite{ritchie1972}      & CIMIS 中的改进的 Penman 公式     & 叶片尺度能量平衡与气孔导度耦合 & \authornumcite{ritchie1972}  & \authornumcite{priestley1972,fao56}     \\
        土壤水分               & 2D RHIZOS 模型\cite{lambert1976} & 2D RHIZOS 模型\cite{lambert1976} & 2D 模型                        & \authornumcite{ritchie1972}  & \authornumcite{ritchie1998,ritchie2009} \\
        土壤氮                 & 土壤和植物氮平衡的动态仿真       & 土壤和植物氮平衡的动态仿真       & 无                             & 统计和经验方法预测潜在氮吸收 & \authornumcite{godwin1998,gijsman2002}  \\
        土壤磷                 & 无                               & 无                               & 无                             & 无                           & 有                                      \\
        土壤盐分               & 无                               & 有                               & 无                             & 无                           & 无                                      \\
        渍灾                   & 无                               & 无                               & 无                             & 有                           & 有                                      \\
        涝灾                   & 无                               & 无                               & 无                             & 无                           & 有                                      \\
        \bottomrule
    \end{tabular}
\end{table}

\subsection{棉花生长模型的应用研究}

棉花生长模型最基本的目的是预测产量,近几年,国外已经再水分利用效率评价和灌溉管理\cite{baumhardt2014,booker2014,booker2015,modala2015,thorp2015,anapalli2016,attia2016,linker2016,tsakmakis2018,thorp2019,thorp2020,thorp2020a}、
氮磷动态和施肥管理\cite{shumway2012,amin2017,arshad2017,zurweller2019}、
气象变化响应、
品质模拟、
打顶和生产管理\cite{yang2008}等方面开展了大量研究。
为了不同的研究目的,所选择的模型也存在较大的差异。
国内学者也在产量预测、模型性能评价、水分使用效率评价、品质成分含量模拟、株高生长模拟、水氮耦合效应评价、水盐运移、根系生长模拟和模型敏感性与不确定性分析等方面开展了研究和探索。
然而,在不同生长发育时期,冠层截获的光能在不同空间的分布比例存在一定的差异性和规律性,定量化描述这个过程有望提高光合作用模拟精度。

值得注意的是, GOSSYM 及其后勤版本如 Cotton2K 是世界上最成功的棉花模型之一。
Cotton2K 模型基于 GOSSYM 模型的过程公式,借鉴了 SIMCOTI , SIMCOTII 和 CALGOS 模型的算法,使用每小时的气象数据计算水分和能量平衡,
提高了干旱和灌溉条件下的棉花生长模拟的精度和适用性。
虽然模型也考虑了盐分和水分在土壤的分布,然而,使用的 Richards 方程 \cite{richards1931} 计算水分运移过程存在上下边界粗糙的问题,
而且对于滴灌模式,沿滴灌带垂直方向的根系分布也应被精细的考虑;
另外,实际蒸腾计算过程中对冠层光截获因子的精细模拟也有改进的空间。

\subsection{发展动态和问题分析}
从局部到整体,由简单到复杂,由经验性到机理性,由智能化到数字化的发展是棉花生长模型的主要发展趋势。另外,增强目标性和适用性,构建专门针对某一生产实际问题的专用模型更具有应用价值[12]。
项目从服务南疆地区棉花无膜栽培的几个关键问题出发,课题组认为以下问题急需深入分析和探讨。

\begin{enumerate}
    \item 播种量是无膜棉种植中的一个重要问题,播种密度影响光能在冠层内的分布,在不同发育阶段光合有效叶面积指数在冠层水平和垂直方向分布比例是动态变化的,对冠层光能截获的精确定量化模拟是关键问题之一。
    \item 播种日期是影响无膜棉出苗率和产量的关键因素,综合考虑光热效应以及地温对出苗、现蕾、开花和吐絮时间的影响,有望提高物候学发育阶段的模拟精度以确定一个合适的播种日期。
    \item 灌溉制度高度影响棉花产量,水分需求需要精确的模拟。Cotton2K 虽然提供了每小时的模拟结果,但如能精确考虑冠层光能截获、根系分布和盐分对水分运移影响并反映到模型中,有望提高水分运移的模拟精度。
    \item 最新的多尺度的 SWAP 模型(version 4.0)\cite{swap2021} 是否可以改进 Cotton 2K 模型的水分运移模拟精度,需要对比研究和验证。
\end{enumerate}

综上所述,为了服务中棉 619 在南疆地区的无膜栽培推广应用,本文以 Cotton2K 模型为基础,
重点研究考虑冠层光合作用有效叶面积指数的水平、垂直分布比例的光能截获模拟,以及考虑冠层光能截获、根系和盐分空间分布影响的水分运移模拟,
建立适合南疆干旱、盐渍化土壤特点的棉花生长模型,
为无膜棉的播种日期、播种量的确定和灌溉制度提供定量化的分析手段。

  \chapter{材料与方法}

\section{试验观测}
试验区位于南疆地区阿拉尔市塔里木大学灌溉试验站 (\ang{81;11;46}E, \ang{40;37;28}N)。
\begin{figure}
    \centering
    \includegraphics[scale=0.4]{research_site.png}
    \caption{试验站所在位置}
\end{figure}
灌溉实验按照田间实验设计,采用滴灌方式,通过气象数据,按照水分需求进行灌溉,以南疆地区覆膜栽培棉花经验灌水量 4200 m$^3$/hm$^2$ (420 mm) 为基准,%
设置 6 个不同水平的灌水处理,分别为 80\% 灌水量 (W1)、90\% 灌水量 (W2)、100\% 灌水量 (W3)、110\% 灌水量 (W4)、120\% 灌水量 (W5)、130\% 灌水量 (W6)。
为避免相邻小区的影响,小区采取随机组合而非按灌水量排序,每小区面积 72 m$^2$,等行距种植,行间距为 76 cm,南北走向,%
采用一管带两行的灌溉方式间隔行铺设地下滴灌带,每个处理共有 3 个重复,总共 18 个小区。
因本文不研究其他营养对棉花生长的影响,所以 N、P、K 肥按照经验值随滴灌施入,各小区施肥量统一,%
$N$-$P_2O_5$-$K_2O$ 按照 250-100-50 $\mathrm{kg/hm^2}$ 的量进行施肥。

\subsection{气象数据}

该模型所需的每日气象数据包括太阳辐射 ($langley$),最高和最低温度 (℃) 和降水 (mm) 以及2米高度的风速 (km/h)。

\begin{table}
    \caption{Cotton2K 模型所需输入每日气象数据}\label{tab:meteorology}
    \small
    \centering
    \begin{tabular}{cccc}
        \toprule
        名称         & 解释                     & 单位                                              & 是否必须 \\
        \midrule
        太阳辐射强度 & 大气下垫面短波辐射强度   & $\mathrm{langley} = 0.04184\ \mathrm{MJ\ m^{-2}}$ & 是       \\
        最高气温     & 2 m 最高气温             & ℃                                                 & 是       \\
        最低气温     & 2 m 最低气温             & ℃                                                 & 是       \\
        露点温度     & 空气冷却达到饱和时的温度 & ℃                                                 & 否       \\
        降水         & 24 小时内降水            & mm                                                & 是       \\
        风速         & 2 m 风速                 & km/h                                              & 否       \\
        \bottomrule
    \end{tabular}
\end{table}

当露点温度不可得时,可通过公式 \ref{eq:dewpoint} 估算而得。

\begin{equation}\label{eq:dewpoint}
    T_{dew} = \begin{cases}
        SitePar_5                                                            & T_{\max} < 20        \\
        \frac{(40 - T_{\max}) * SitePar_5 + (T_{\max} - 20) * SitePar_6}{20} & 20 \le T_{\max} < 40 \\
        SitePar_6                                                            & T_{\max} \ge 40      \\
    \end{cases}
\end{equation}

式中,$T_{dew}$ 为露点温度 (℃),$T_{\max}$ 为最高气温 (℃),$SitePar_5$ 与 $SitePar_6$ 为用户提供输入,与试验站相关。

当每日风速数据不可得时,使用年平均风速代替。

\begin{figure}
    \centering
    \includegraphics[scale=0.5]{climate.png}
    \caption{2019{-}2021年试验站棉花生长期主要气象数据}
\end{figure}

冠层温湿度由精创 RC-4HA 收集,设置取样间隔为 0.5 小时。

\subsection{光能截获率}

在2019年使用LI-6400XT便携式光合作用系统,在2020年使用 LI-191R 线量子传感器和 LI-1500 光传感器在每10天的晴天中午13点至15点之间收集光截获数据。
行内空间被划分为网格,5个水平列和根据植物高度分为6层,每个网格的宽度相等,高度为 20 厘米,然后用设备测量每个网格的照度。
光截获量的计算方法是:同一水平层中各网格的平均光照度减去比率,参见公式\ref{eq:measured_li}。

\begin{equation}\label{eq:measured_li}
    LI_{i} = 1 - \frac{\sum^5_{j=1} I_{i,j}}{\sum^5_{j=1} I_{i+1,j}} \quad \mathrm{for} i \in {1,2,\dots,6}
\end{equation}

式中, $LI_{i}$ 为第 $i$ 层的光能截获率, $I_{i,j}$ 是第 $i$ 层第 $j$ 列网格的光照强度 ($W\ m^{-2}$), $I_{i+1,j}$ 为上层网格的光照强度 ($W\ m^{-2}$)。

\subsection{土壤参数}
每周取样测定一次土壤容积含水量(4 $\times$ 3 网格),另外,使用矫正后的土壤水分和温度自动记录仪 (HOBO H21-002, United States) 分 5 层实时监测 0-100cm 深度的土壤水分含量;

土壤温度由 Thermochron® iButton®器件(DS1921G) 收集,设置取样间隔为 2 小时,将传感器埋于各处理区中心紧挨棉花种植行的位置,埋深为 5 cm、10 cm、20 cm、40 cm、50 cm。

取样测量土壤田间持水率、容积密度、枯萎点含水率、饱和土壤含水率和渗透系数等参数,并记录好灌溉日期与灌溉量;
\subsection{棉花生理指标}
试验记录内容主要包括:
\begin{enumerate}
    \item 生育期:记录棉花播种、出苗、现蕾、开花和吐絮期等生育期;
    \item 株式图:每 10 天记录株高、叶片数、果枝数、果枝长度、果枝高度、蕾、花、小铃、大铃及吐絮铃个数与部位等;
    \item 叶面积指数:每 10 天结合干物质测定,取全株叶片,用扫描法测定冠层垂直方向 20 个深度和水平 8 个网格的叶面积指数;
    \item 干物质积累和分配:每 10 天取样一次,在苗期每次共取 5 株样品,开花期后每个处理取 3 株样品,样品分器官后在 80℃ 下烘干至恒重后,分别测定各器官的干物质重量;
    \item 产量构成因素:测定棉花籽棉产量、衣分、铃重和纤维重量;
\end{enumerate}

\section{模型模拟}
\subsection{模型改进}

分析叶面积指数、叶倾角和方位角的空间分布规律及光能截获分布,冠层深度拟分为 20 层,每层占棉花总高度的 5\%。
首先,计算每一层的有效叶面积指数,%
其次,定量计算每一层的光合作用碳同化产量,%
最后,重新驱动 Cotton2K 模型,模拟干物质积累。详细的流程参见第 \ref{sec:canopyLayering}节。

\subsection{模型参数化}
本文采用 2019-2020 连续 2 年的棉花实验数据对 Cotton2K 模型进行参数化。
其中作物生长参数如叶面积指数 (LAI),株高,干物质积累和主茎节数通过实验得到。
土壤参数如土壤水分、盐分初始值由实验得到;饱和导水率,Van-Genuchten
公式参数 $(\alpha, \beta)$ 等参数由率定得到;气象参数取自阿拉尔气象站;
作物参数和水分胁迫参数由率定得到。

\section{敏感性分析}
Sobol 敏感性分析方法是由俄罗斯科学家I. M. Sobol首次提出的\cite{sobol2001},基础是将模型输出方差分解为维度增加的输入参数方差的总和,
其目的是确定每个输入参数或不同参数之间的相互作用在什么水平上影响结果的方差。%
Sobol 敏感性分析中输出方差的分解采用了与因子设计中经典的方差分析相同的原理。%
应该注意的是,Sobol 敏感性分析的目的不是为了确定输入方差的原因。%
它只是表明它对模型输出有什么影响,影响到什么程度。%
因此,它不能用来确定方差的来源,如在棉花生产中,各农艺操作对棉花产量的影响来源。

任何敏感性分析的重要步骤之一,无论是局部还是整体,都是为了确定用于分析的适当模型输出。

Sobol 敏感性分析对模型输入和输出之间没有预设要求,这极大的提高的分析的普适性,而且 Sobol 敏感性分析可以评估单个参数以及参数间的交互作用。%
与其他全局敏感性分析方法类似,相比局部敏感性分析方法,Sobol 敏感性分析也有计算量大的缺陷。

\subsection{取样方法}
Sobol 序列是一个应用广泛的准随机 (quasi-random) 低差值序列,%
通常比完全随机的序列更均匀地对空间进行采样,用于生成参数空间的统一样本。%
本文采用 \authoryearcite{saltelli2002} 拓展的 Sobol 序列生成模型输入。%
生成的输入样本有 $N * (2D + 2)$ 条,其中 $N$ 为采样数, $D$ 是参数的数量。参见算法\ref{alg:sobol}。

Sobol 序列的一般特征有
\begin{enumerate}
    \item Sobol 序列是一种低差异的序列,也被称为 “准随机序列”。
    \item 比伪随机数的分布更均匀
    \item 准蒙特卡洛积分产生更快的收敛性和更好的准确性
    \item 缺点是需要计算高维度的积分
\end{enumerate}

需要注意的是,Sobol 序列的初始点有一些重复 (见\authornumcite{campolongo2011} 的表2),%
这可以通过设置 \texttt{skip\_values} 参数来避免,这样可以提高样本的均匀性。%
然而,事实证明,简单地跳过数值可能会降低精度,增加实现收敛所需的样本数量\cite{owen2021}。
建议将 \texttt{skip\_values} 和 $N$ 都设为 2 的幂,其中 $N$ 是所需的样本数 (进一步的背景见\authornumcite{owen2021}和\authornumcite{scipyAddStatsQMC2021}中的讨论)。%
其中还建议 $\mathtt{skip\_values} \ge N$。%
默认将 \texttt{skip\_values} 设置为不小于 $N$ 的 2 的幂。%
如果提供了 \texttt{skip\_values},该方法现在会在根据上述建议样本量可能不理想的情况下引发一个 \texttt{UserWarning}。

\begin{algorithm}
    \caption{Sobol 序列取样方法}\label{alg:sobol}
    \KwIn{$N, D$, 以及预定义的 $\mathbf{directions}$}
    \KwOut{$N \times D$的数组 $\mathbf{result}$}
    $scale = 31$\\
    $L = \left\lceil \frac{\log N}{\log 2} \right\rceil$\\
    \For{$i \in \{0,1,\dots,D-1\}$}{
        $\mathbf{V} = \left. \begin{bmatrix}0\\0\\\vdots\\0\end{bmatrix} \right\} L+1$\\
        \eIf{$i = 0$}{
            $\mathbf{V} = \begin{bmatrix}0\\ 2^1\\ 2^2 \\ \vdots\\ 2^L \end{bmatrix}$
        }{
            $\mathbf{m} = directions_{i - 1}$\\
            $a = m_0$\\
            $s = len(\mathbf{m})$\\
            \eIf{$L \le s$}{
                $\mathbf{V} = \begin{bmatrix}0\\ m_1 \\ m_2 \\ \vdots\\ m_L \end{bmatrix}\circ \begin{bmatrix}0\\ 2^{scale - 1} \\ 2^{scale - 2} \\ \vdots\\ 2^{scale - L} \end{bmatrix}$
            }{
                $\mathbf{V} = \begin{bmatrix}0\\ m_1 \\ m_2 \\ \vdots\\ m_s \\ m_1 \\ m_2 \\ \vdots \\ m_{L - s} \end{bmatrix} \circ \begin{bmatrix}0\\ 2^{scale - 1} \\ 2^{scale - 2} \\ \vdots\\ 2^{scale - s} \\ 1 \\ 1 \\ \vdots \\ 1 \end{bmatrix}$\\
                \For{$j \in \{s+1,s+2,\dots,L\}$}{
                    $V_j = V_{j - s} \veebar (V_{j - s} >> s)$\\
                    \For{$k \in \{s+1,s+2,\dots,L\}$}{
                        $V_j = V_j \veebar ((a >> (s - 1 - k)) \land 1) \times V_{j-k}$
                    }
                }
            }
        }
        $X = 0$\\
        \For{$j \in \{1,2,\dots,N\}$}{
            $X = X \veebar V_{\mathtt{最小显著零位的索引}(j-1)}$
            $result_{ji} = \frac{X}{2^{scale}}$
        }
    }
\end{algorithm}

\begin{algorithm}
    \caption{最小显著零位的索引}
    \KwIn{$value$}
    \KwOut{$index$}
    $index = 1$\\
    \While{$value \mod 2 \neq 0$}{
        $value = value >> 1$\\
        $index = index + 1$
    }
\end{algorithm}


\subsection{分析方法}

Sobol 敏感性分析一般用于复杂的系统模型,它将输出方差与其资源进行定量分解:即来自单个参数或来自参数间的相互作用。%
全局敏感性指数通常用于评估一个参数的总体贡献以及与其他参数的相互作用。%
Sobol 敏感性指数有几个特点:
\begin{enumerate}
    \item 全局/一阶/二阶敏感性指数为正值。
    \item 敏感性指数大于0.05的参数被认为是显著的。
    \item 所有敏感指数之和应等于1。
    \item 全局敏感性指数大于一阶敏感性指数。
\end{enumerate}

进行可靠的Sobol 敏感性分析所需的样本量取决于两个主要因素,\begin{enumerate*}
    \item 模型的复杂性和
    \item 评估的参数数量
\end{enumerate*} 。尽管对生成的最佳参数集数量没有普遍共识,但一般的经验法则是,模型参数的数量越大,使用的参数集数量就越多。%
例如,对于一个有大量不确定参数的复杂模型(如 20 个参数),至少要进行 100,000 次模型评估。%
对于不那么复杂的模型,较少的评估次数(如1000次)可能就足够了。%
但应该注意的是,随着评价数量的增加,计算成本也会增加。%
选择的评价数量是否合适,可以用自举法置信区间来检验。%
一般来说,最敏感的参数应该有较窄的置信区间,即小于敏感指数的10\%。

\section{多目标优化}
NSGA-III 的基本框架类似于原始的 NSGA-II 算法 \cite{NSGA2},其选择运算符有重大变化。
与 NSGA-II 中介绍的拥挤度概念不同的是,NSGA-III 是通过用户输入或模型自行选择一些在高维参数空间良好分布 (fine distributed) 的参考点来实现保留部分次优解以维持种群多样性。
为了完整起见,首先对原始的 NSGA-II 算法进行简要描述。
考虑 NSGA-II 算法的第 $t$ 代,假设这一代的父代种群为 $P_t$,其规模为 $N$ ,而由 $P_t$ 产生的子代种群为 $Q_t$,有 $N$ 个成员。
第一步是从结合的亲代和子代种群 $R_t = P_t \cup Q_t$(大小为 $2N$ )中选择最好的 $N$ 个成员,从而能够保留亲代种群的精英成员。
为了实现这一目标,首先根据不同的非同源等级($F_1$、$F_2$等)对组合群体 $R_t$ 进行排序。
然后,从 $F_1$ 开始,每次选择一个非同化水平来构建一个新的种群 $S_t$ ,直到St的大小等于 $N$ 或首次超过 $N$。
因此,从第 $(l+1)$ 级开始的所有解决方案都被拒绝在组合群体 $R_t$ 中。
在这种情况下,只有那些能使第l层前面的多样性最大化的解决方案被选择。
在 NSGA-II 中,这是通过一个计算效率高但近似的利基 (niche) 保护算子来实现的,该算子将每个最后一级成员的拥挤距离计算为两个相邻解决方案之间的客观归一化距离之和。
此后,具有较大拥挤距离值的解决方案被选中。
在 NSGA-III 中,用以下方法取代拥挤距离算子。

\subsection{将种群按非支配等级分类}

上述利用通常的支配原则\cite{chankong1983}识别非支配前沿的程序也被用于NSGA-III。
如果 $|St|=N$ ,则不需要进一步的操作,下一代从$P_t+1=S_t$开始。
对于 $|St|>N$,从一到 ($l-1$) 个前沿的成员已经被选中,即 $P_{t+1}= \cup_{i=1}^{l-1} F_i$ ,剩下的 $(K=N-|Pt+1|)$ 种群成员从最后的前沿 $F_l$ 中选择。
在下面几个小节中描述其余的选择过程。

\subsection{确定空间中的参考点}
如前文所述, 默认情况下,NSGA-III 会使用一组预定义的参考点集,%
但参考点集也可以由用户提供,这些参考点将会用于对次优的 Pareto 前沿进行排序并选取合适的解进入下代的迭代。
在没有任何偏好信息的情况下,可以采用任何预定义的结构化的参考点放置,但实践中常常使用 Das和Dennis的\cite{das&dennis1998}系统方法1,
将点放置在一个 $M-1$ 维的单位射线与归一化的超平面交点上,该平面对所有目标轴的交角相同且截距为1,以 3 维的问题为例,该平面的数学表达式为 $x + y + z = 1$。
如果沿每个目标考虑 $p$ 个划分,那么在一个 $M$ 目标问题中,参考点的总数 ($H$) 由以下公式给出

\begin{equation}\label{eq:H}
    H = \begin{pmatrix}
        M + p - 1 \\
        p
    \end{pmatrix}
\end{equation}

例如,在一个三目标问题 ($M = 3$) 中,参考点被创建在一个三角形上,顶点在 $(1, 0, 0)$,$(0, 1, 0)$,$(0, 0, 1)$。
如果为每个目标轴选择四个分部 $(p=4)$, $H = (\begin{smallmatrix}
        3+4-1\\
        4
    \end{smallmatrix})$ 或 $15$ 个参考点将被创建,这些参考点如图 \ref{fig:refPoints} 所示。
在 NSGA-III 中,除 Pareto 前沿的解集外,还保留了与这些参考点的相关的种群成员。
因为构造的参考点是均匀地分布在超平面上,因此获得的解集也可能均匀分布在 Pareto 前沿附近。
该程序在算法 \ref{alg:NSGA3} 中提出。
\begin{figure}
    \tdplotsetmaincoords{75}{135}
    \centering
    \begin{tikzpicture}
        [tdplot_main_coords,
            axis/.style={-stealth,thick},
            vector/.style={-stealth,very thick}]
        \draw[axis] (0,0,0) -- (5,0,0) node[anchor=east]{$f1$};
        \draw[axis] (0,0,0) -- (0,5,0) node[anchor=west]{$f2$};
        \draw[axis] (0,0,0) -- (0,0,5) node[anchor=west]{$f3$};
        \filldraw[fill=gray,fill opacity=0.8] (0,0,4)--(4,0,0)--(0,4,0)--cycle;
        \draw[-stealth,very thick] (0,3,3) node[anchor=west]{归一化平面} -- (0.5,1.25,2.25);
        \draw[-stealth,very thick] (2,3,0) node[anchor=west]{理想点} -- (0,0,0);
        \fill(4,0,0) circle (2pt) node[anchor=south] {1};
        \fill(3,1,0) circle (2pt);
        \fill(3,0,1) circle (2pt);
        \fill(2,2,0) circle (2pt);
        \fill(2,1,1) circle (2pt);
        \fill(2,0,2) circle (2pt);
        \fill(1,3,0) circle (2pt);
        \fill(1,2,1) circle (2pt);
        \fill(1,1,2) circle (2pt);
        \fill(1,0,3) circle (2pt);
        \draw[-stealth,very thick] (3,0,4) node[anchor=east]{参考点} -- (1,0,3);
        \fill(0,4,0) circle (2pt) node[anchor=south] {1};
        \fill(0,3,1) circle (2pt);
        \fill(0,2,2) circle (2pt);
        \fill(0,1,3) circle (2pt);
        \fill(0,0,4) circle (2pt) node[anchor=south east] {1};
        \draw[vector] (0,0,0) -- (3,1.5,1.5) node[anchor=south]{参考向量};
    \end{tikzpicture}
    \caption{15个结构化参考点显示在归一化参考平面上,适用于$p=4$的三目标问题}\label{fig:refPoints}
\end{figure}


\begin{algorithm}
    \caption{NSGA-III 第 t 代的过程}\label{alg:NSGA3}
    \KwIn{$H$ 构造的参考点 $Z^s$ 或提供的目标点 $Z^a$, 父代种群 $P_t$}
    \KwOut{$P_{t+1}$}

    $S_t = \emptyset, i = 1$\\
    $Q_t$ = 重组+变异($P_t$)\\
    $R_t = P_t \cup Q_t$\\
    $(F_1, F_2, \dots)$ = 非支配排序($R_t$)\\
    \Repeat{$|S_t \ge N|$}{$S_t = S_t \cup F_i$ 且 $i = i + 1$}
    包含父代的 Pareto 前沿: $F_l = F_i$\\
    \eIf{$|S_t| = N$}{
    $P_{t+1} = S_t$, break
    }{
    $P_{t+1} = \cup^{l-1}_{j=1} F_j$\\
    从 $F_l$ 中选点: $K = N - |P_{t+1}|$\\
    归一化目标并创建参考集 $Z^r$: $\mathtt{Normalize}(\mathbf{f}^n, S_t, Z^r, Z^s, Z^a)$\\
    将 $S_t$ 的每个成员 $\mathbf{s}$ 与参考点相关联: $[\pi(\mathbf{s}), d(\mathbf{s})] =\mathtt{Associate}(S_t, Z^r)$
    \tcc{$\pi(\mathbf{s})$: 最近参考点, $d$: $\mathbf{s}$ 与 $\pi(\mathbf{s})$ 之间的距离}
    计算参考点的利基 (niche) 数 $j \in Z^r$: $\rho_j = \sum_{\mathbf{s}\in S_t/F_l}((\pi(\mathbf{s})) ? 1 : 0 )$\\
    一次从 $F_l$ 选取 $K$ 个成员构建 $P_{t+1}$: $\mathtt{Niching}(K, \rho_j, \pi, d, Z^r, F_l, P_{t+1})$
    }
\end{algorithm}
\subsection{种群成员的适应性归一化}
首先,$S_t$种群的理性点是由在$\cup_{\tau=0}^t S_{\tau}$中的每个目标函数$i =1,2,\dots,M$的最小值 ($z_i^{\min}$) 通过构建理想点%
$\overline{z} = (z_1^{\min}, z_2^{\min},\dots,z_M^{\min})$ 来确定的。%
然后,$S_t$ 的每个目标值通过用$z_i^{\min}$减去目标$f_i$来翻译,这样,翻译后的 $S_t$ 的理想点就成为一个零矢量。%
接着,通过寻找使相应的成就标化函数 (由$f'_i(\mathbf{x}) = f_i(\mathbf{x}) - z_i^{\min}$和接近第$i$个目标轴的权重向量形成) 最小的解 $(x \in S_t)$,%
来确定每个(第$i$个)目标轴中的极端点 ($z^{i,\max}$)。
随后,这些$M$个极端向量被用来构成一个$M$维的超平面。%
接下来可以计算出第$i$个目标轴和线性超平面的截距$a_i$(见图\ref{fig:formingHyperPlane})。%
\begin{figure}
    \tdplotsetmaincoords{45}{120}
    \centering
    \begin{tikzpicture}
        [tdplot_main_coords,
            cube/.style={very thick,black},
            grid/.style={very thin,gray},
            axis/.style={-stealth,thick},
            vector/.style={-stealth,very thick},
            annotation/.style={fill=white,font=\footnotesize,inner sep=1pt}]
        \draw[axis] (0,0,0) -- (6,0,0) node[anchor=east]{$f'_1$};
        \draw[axis] (0,0,0) -- (0,6,0) node[anchor=west]{$f'_2$};
        \draw[axis] (0,0,0) -- (0,0,6) node[anchor=west]{$f'_3$};
        \foreach \coo in {2,4}
            {
                \draw (\coo, 0, 0) node[anchor=east] {\fpeval{\coo/2}} -- (\coo, 0.25, 0);
                \draw (0, \coo, 0) node[anchor=south] {\fpeval{\coo/2}} -- (0.25, \coo, 0);
                \draw (0, 0, \coo) node[anchor=east] {\fpeval{\coo/2}} -- (0, 0.25, \coo);
            }
        \foreach \coo in {1,3,5}
            {
                \draw (\coo, 0, 0) -- (\coo, 0.125, 0);
                \draw (0, \coo, 0) -- (0.125, \coo, 0);
                \draw (0, 0, \coo) -- (0, 0.125, \coo);
            }
        \draw[thick] (0,0,5.2)--(4.4,0,0)--(0,3.5,0)--cycle;
        \draw (4.4,0,0)--(4.4,-1,0);
        \draw (0,0,0)--(0,-1,0);
        \draw (0,0,5.2)--(0,-1,5.2);
        \draw (0,0,0)--(-1,0,0);
        \draw (0,3.5,0)--(-1,3.5,0);
        \draw[arrows=<->] (0,-0.8,0)--(2.2,-0.8,0) node[annotation] {$a_1$}--(4.4,-0.8,0);
        \draw[arrows=<->] (-0.8,0,0)--(-0.8,1.75,0) node[annotation] {$a_2$}--(-0.8,3.5,0);
        \draw[arrows=<->] (0,-0.8,0)--(0,-0.8,2.6) node[annotation] {$a_3$}--(0,-0.8,5.2);
        \fill(\fpeval{4.4*0.1},\fpeval{3.5*0.1},\fpeval{5.2*0.8}) circle (2pt);
        \fill(\fpeval{4.4*0.05},\fpeval{3.5*0.8},\fpeval{5.2*0.15}) circle (2pt);
        \fill(\fpeval{4.4*0.8},\fpeval{3.5*0.1},\fpeval{5.2*0.1}) circle (2pt);
        \draw[dash pattern=on 5pt off 5pt] (\fpeval{4.4*0.1},\fpeval{3.5*0.1},\fpeval{5.2*0.8}) node[anchor=south west] {$z^{3,\max}$}--(\fpeval{4.4*0.05},\fpeval{3.5*0.8},\fpeval{5.2*0.15}) node[anchor=south west] {$z^{2,\max}$}--(\fpeval{4.4*0.8},\fpeval{3.5*0.1},\fpeval{5.2*0.1}) node[anchor=west] {$z^{1,\max}$}--cycle;
    \end{tikzpicture}
    \caption{以三目标问题为例,计算截距,然后从极端点形成超平面的程序}\label{fig:formingHyperPlane}
\end{figure}
要特别注意处理退化的情况和非负的截距。%
再然后,目标函数可以被规范化为
\begin{equation}\label{eq:normalize}
    f_i^n (\mathbf{x})= \frac{f_i'(x)}{a_i},\ \mathrm{for}\ i = 1, 2, \dots, M.
\end{equation}
请注意,现在每个归一化目标轴的截点都在$f_i^n = 1$,用这些截点构建的超平面将使$\sum^M_{i=1} f_i^n = 1$。
在结构化参考点 (其中有$H$个) 的情况下,用Das和Dennis\cite{das&dennis1998}的方法计算的原始参考点已经位于这个归一化的超平面上。%
在用户偏爱参考点的情况下,参考点只需使用 \ref{eq:normalize} 映射到上述构建的标准化超平面上。%
由于归一化程序和超平面的创建是在每一代使用从模拟开始时发现的极端点来完成的,%
因此 NSGA-III 程序在每一代都能自适应地保持 $S_t$ 成员所跨越空间的多样性。%
这使得 NSGA-III 能够解决具有帕累托最优前沿的问题,其目标值可以有不同的比例。%
该程序也在算法 \ref{alg:normalize} 中进行了描述。
\begin{algorithm}
    \caption{$\mathtt{Normalize}(\mathbf{f}^n, S_t, Z^r, Z^s, Z^a)$过程}\label{alg:normalize}
    \KwIn{$S_t$, $Z^s$ (构造点) 或 $Z^a$ (提供点)}
    \KwOut{$\mathbf{f}^n$, $Z^r$ (归一化的超平面上的参考点)}

    \For{$j = 1 \mathbf{to} M$}{
    计算理想点: $z_j^{\min} = \min_{\mathbf{S}\in S_t} f_j(s)$
    翻译目标: $f_j'(\mathbf{s}) = f_j(\mathbf{s}) - z_j^{\min} \quad \forall \mathbf{s} \in S_t$
    计算极点: $(\mathbf{z}^{j,\max}, j = 1, \dots, M)$ of $S_t$
    }

    计算截点 $a_j$ 对 $j = 1, \dots, M$
    使用公式 \ref{eq:normalize} 归一化目标 ($\mathbf{f}^n$)

    \eIf{是否提供 $Z^a$}{
        使用公式 \ref{eq:normalize} 将每个(吸气)点映射到归一化的超平面上,并将这些点保存在集合 $Z^r$ 中。
    }{
        $Z^r=Z^s$
    }
\end{algorithm}
\subsection{关联操作}
在根据目标空间中$S_t$成员的范围自适应地对每个目标进行归一化后,需将每个群体成员与一个参考点联系起来。%
为此,通过连接参考点和原点,在超平面上定义一条对应于每个参考点的参考线。%
然后,计算 $S_t$ 的每个种群成员与每条参考线的垂直距离。%
在归一化目标空间中,参考线最接近种群成员的参考点被认为与该种群成员相关。%
这在图\ref{fig:associate}中得到了说明。该程序在算法\ref{alg:associate}中提出。
\begin{algorithm}
    \caption{$\mathtt{Associate}(S_t,Z^r)$ 过程}\label{alg:associate}
    \KwIn{$Z^r, S_t$}
    \KwOut{$\pi(\mathbf{s} \in S_t), d(\mathrm{s} \in S_t)$}
    \ForEach{每个参考点 $\mathbf{z} \in Z^r$}{
        计算参考线 $\mathbf{w} = \mathbf{z}$
    }
    \ForEach{$\mathbf{s} \in S_t$}{
    \ForEach{$\mathbf{w} \in Z^r$}{
        计算 $d^{\bot}(\mathbf{s}, \mathbf{w}) = \parallel (\mathbf{s} - \mathbf{w}^T \mathbf{sw} /\parallel \mathbf{w} \parallel^2) \parallel $
    }
    $\pi(\mathbf{s}) = \mathbf{w} : \mathrm{argmin}_{\mathbf{w} \in Z^r} d^{\bot}(\mathbf{s}, \mathbf{w})$
    $d(\mathbf{s}) = d^{\bot}(\mathbf{s},\pi(\mathbf{s}))$
    }
\end{algorithm}
\begin{figure}
    \tdplotsetmaincoords{45}{150}
    \centering
    \begin{tikzpicture}
        [tdplot_main_coords,
            cube/.style={very thick,black},
            grid/.style={very thin,gray},
            axis/.style={-stealth,thick},
            vector/.style={-stealth,very thick},
            annotation/.style={fill=white,font=\footnotesize,inner sep=1pt},
            refPoints/.style={fill=gray},
            sample/.style={fill=black},
            dashedLine/.style={dash pattern=on 5pt off 5pt}]
        \draw (0,6,0) -- (3,6,0) node[anchor=north west]{$f'_1$} -- (6,6,0);
        \draw (6,0,0) -- (6,3,0) node[anchor=east]{$f'_2$} -- (6,6,0);
        \draw (6,0,0) -- (6,0,3) node[anchor=west]{$f'_3$} -- (6,0,6);
        \foreach \tick in {0,0.5,1,1.5} {
                \draw (\fpeval{0+\tick*4},6,0) -- (\fpeval{0+\tick*4},6.25,0) node[anchor=north west] {\tick};
                \draw (6,\fpeval{0+\tick*4},0) -- (6.25,\fpeval{0+\tick*4},0) node[anchor=north east] {\tick};
                \draw (6,0,\fpeval{0+\tick*4}) -- (6,-0.25,\fpeval{0+\tick*4}) node[anchor=south east] {\tick};
            }
        \filldraw[fill=gray!30,fill opacity=0.6] (0,0,4)--(4,0,0)--(0,4,0)--cycle;
        \fill[refPoints](4,0,0) circle (2pt);
        \draw[dashedLine] (0,0,0) -- (5,0,0);
        \fill[refPoints](\fpeval{8/3},\fpeval{4/3},0) circle (2pt);
        \draw[dashedLine] (0,0,0) -- (\fpeval{8/3 * 5 / sqrt(80/9)},\fpeval{4/3 * 5 / sqrt(80/9)},0);
        \fill[refPoints](\fpeval{4/3},\fpeval{8/3},0) circle (2pt);
        \draw[dashedLine] (0,0,0) -- (\fpeval{4/3 * 5 / sqrt(80/9)},\fpeval{8/3 * 5 / sqrt(80/9)},0);
        \fill[refPoints](0,4,0) circle (2pt);
        \draw[dashedLine] (0,0,0) -- (0,5,0);
        \fill[refPoints](\fpeval{8/3},0,\fpeval{4/3}) circle (2pt);
        \draw[dashedLine] (0,0,0) -- (\fpeval{8/3 * 5 / sqrt(80/9)},0,\fpeval{4/3 * 5 / sqrt(80/9)});
        \fill[refPoints](\fpeval{4/3},\fpeval{4/3},\fpeval{4/3}) circle (2pt);
        \draw[dashedLine] (0,0,0) -- (\fpeval{4/3 * 5 / sqrt(16/3)},\fpeval{4/3 * 5 / sqrt(16/3)},\fpeval{4/3 * 5 / sqrt(16/3)});
        \fill[refPoints](0,\fpeval{8/3},\fpeval{4/3}) circle (2pt);
        \draw[dashedLine] (0,0,0) -- (0,\fpeval{8/3 * 5 / sqrt(80/9)},\fpeval{4/3 * 5 / sqrt(80/9)});
        \fill[refPoints](0,\fpeval{4/3},\fpeval{8/3}) circle (2pt);
        \draw[dashedLine] (0,0,0) -- (0,\fpeval{4/3 * 5 / sqrt(80/9)},\fpeval{8/3 * 5 / sqrt(80/9)});
        \fill[refPoints](\fpeval{4/3},0,\fpeval{8/3}) circle (2pt);
        \draw[dashedLine] (0,0,0) -- (\fpeval{4/3 * 5 / sqrt(80/9)},0,\fpeval{8/3 * 5 / sqrt(80/9)});
        \fill[refPoints](0,0,4) circle (2pt);
        \draw[dashedLine] (0,0,0) -- (0,0,5);

        % sample points
        \fill[sample](0.5,0,4) circle (2pt);
        \draw (0.5,0,4)--(0.0,0.0,4.0);
        \fill[sample](0.6,3.9,2.1) circle (2pt);
        \draw (0.6,3.9,2.1)--(0.0,3.96,1.98);
        \fill[sample](1,4,1) circle (2pt);
        \draw (1,4,1)--(1.8,3.6,0.0);
        \fill[sample](1.2,1.2,3.6) circle (2pt);
        \draw (1.2,1.2,3.6)--(1.68,0.0,3.36);
        \fill[sample](4,1,2) circle (2pt);
        \draw (4,1,2)--(4.0,0.0,2.0);
        \fill[sample](3.2,1,3) circle (2pt);
        \draw (3.2,1,3)--(3.76,0.0,1.88);
        \fill[sample](3.2,1.8,2.4) circle (2pt);
        \draw (3.2,1.8,2.4)--(2.46667,2.46667,2.46667);
    \end{tikzpicture}
    \caption{展示了将种群成员与参考点关联}\label{fig:associate}
\end{figure}

\subsection{利基 (niche) 保护操作}
值得注意的是,一个参考点可以有一个或多个种群成员与之相关联,或者不需要有任何种群成员与之相关联。%
计算 $P_{t+1}=S_t/F_l$ 中与每个参考点相关的种群成员的数量。%
第 $j$ 个参考点的这个利基计数记作 $\rho_j$。%
首先,确定参考点集 $J_{\min} = \{j : \mathrm{argmin}_j \rho_{j}\}$ 具有最小$\rho_j$。%
在有多个这样的参考点的情况下,随机选择一个 ($\overline{j} \in J_{\min}$)。%
如果 $\rho_{\overline{j}}= 0$(意味着参考点 $\overline{j}$ 没有相关的$P_{t+1}$成员),在集合$F_l$中的 $\overline{j}$ 可能有两种情况。%
首先, $F_l$ 中存在一个或多个与参考点 $\overline{j}$相关的成员。%
在这种情况下,与参考线垂直距离最短的成员被添加到 $P_{t+1}$。%
然后,参考点 $\overline{j}$ 的计数 $\rho_{\overline{j}}$递增1。%
第二,前面的$F_l$没有任何与参考点 $\overline{j}$ 相关的成员。%
在这种情况下,参考点被排除在当前一代的进一步考虑之外。%
在 $\rho_{\overline{j}} \ge 1$ 的情况下(意味着$S_t/F_l$中已经有一个与参考点相关的成员存在),如果存在的话,从前面$F_l$中随机选择一个与参考点$\overline{j}$相关的成员被添加到$P_{t+1}$。%
然后,$\rho_{\overline{j}}$的计数被增加1。%
利基计数更新后,该程序共重复K次,以填补 $P_{t+1}$ 的空缺。%
该程序在算法\ref{alg:niche}中提出。
\begin{algorithm}
    \caption{$\mathtt{Niching}(K, \rho_j, \pi, d, Z^r, F_l, P_{t+1})$过程}\label{alg:niche}
    \KwIn{$K, \rho_j, \pi(\mathbf{s} \in S_t), d(\mathbf{s} \in S_t), Z^r, F_l$}
    \KwOut{$P_{t+1}$}
    $k = 1$
    \While{$k \le K$}{
    $J_{\min} = \{ j:\mathrm{agrmin}_{j \in Z^r} \rho_j \}$
    $\overline{j}  = \mathrm{random}(J_{\min})$
    $I_{\overline{j}} = \{ s: \pi(\mathbf{s}) = \overline{j}, s \in F_l \}$
    \eIf{$I_{\overline{j}} \neq \emptyset$}{
    \eIf{$\rho_{\overline{j}} = 0$}{
    $P_{t+1} = P_{t+1} \cup ( \mathbf{s}: \mathrm{argmin}_{\mathbf{S} \in I_{\overline{j}}} d(\mathbf{s}) )$
    }{
    $P_{t+1} = P_{t+1} \cup \mathrm{random}(I_{\overline{j}})$
    }
    $\rho_{\overline{j}} = \rho_{\overline{j}} + 1, F_l = F_l \setminus \mathbf{s}$
    $k = k + 1$
    }{
    $Z^r = Z^r / \{ \overline{j} \}$
    }
    }
\end{algorithm}

\subsection{创造后代群体的遗传操作}
$P_{t+1}$ 形成后,再通过应用通常的遗传算子来创建一个新的子代群体 $Q_{t+1}$。%
在NSGA-III中,已经对解决方案进行了仔细的精英选择,并试图通过强调最接近每个参考点参考线的解决方案来保持解决方案的多样性。%
此外,为了达到快速计算的目的,设定 $N$ 几乎等于 $H$,从而期望在每个参考点对应的帕累托最优前线附近进化出一个群体成员。%
由于所有这些原因,在处理仅有箱体约束的问题时,没有采用 NSGA-III 的任何明确的繁殖操作。%
群体 $Q_{t+1}$ 是通过从 $P_{t+1}$ 中随机挑选父母,应用通常的交叉和变异操作来构建。%
然而,为了使后代的解更接近父代的解,建议在SBX运算中使用一个相对较大的分布指数值,从而使后代接近其父代。

\subsection{NSGA-III 中一代的计算复杂度}
对具有 $M$ 维目标向量的 $2N$ 大小的群体进行非支配性排序(算法 \ref{alg:NSGA3} 的第4行)需要 $O(N \log ^{M-2} N)$次计算\cite{kung1975}。%
算法\ref{alg:normalize}第2行中的理想点的识别总共需要 $O(MN)$ 次计算。%
目标的转换(第3行)需要 $O(MN)$ 次计算。%
然而,识别极端点(第4行)需要 $O(M^2N)$次计算。%
确定截距(第6行)需要一个大小为 $M \times M$ 的矩阵反转,需要 $O(M^3)$ 次运算。%
此后,对最大的 $2N$ 个种群成员进行归一化(第7行)需要 $O(N)$ 次计算。%
算法\ref{alg:normalize}的第8行需要 $O(MH)$次计算。%
算法\ref{alg:associate}中把最多2N个种群成员与H个参考点联系起来的所有操作都需要 $O(MNH)$计算。%
此后,在算法\ref{alg:niche}的排队程序中,第3行将需要 $O(H)$ 次比较。假设 $L=|F_l|$,第5行需要 $O(L)$ 个检查。%
第8行在最坏的情况下需要 $O(L)$ 次计算。其他操作的复杂度较小。%
然而,在 $\mathtt{Niching}$ 算法中,上述计算最多需要执行L次,因此需要更大的$O(L^2)$或$O(LH)$计算。%
在最坏的情况下($S_t = F_1$,即第一个非支配阵线超过种群大小),$L \le 2N$。%
在我们所有的模拟中,我们使用了$N \approx H$,通常 $N > M$。%
考虑到上述所有的考虑和计算,NSGA-III一代的总体最坏情况下的复杂度为 $O(N^2 \log^{M-2} N)$ 或 $O(N^2M)$,以较大者为准。

\subsection{NSGA-III 的无参数特性}
与NSGA-II一样,NSGA-III算法不需要设置任何新的参数,除了通常的GA参数,如种群大小、终止参数、交叉和变异概率及其相关参数。%
参考点的数量 $H$ 不是一个算法参数,因为这与期望的权衡点的数量直接相关。%
种群规模 $N$ 取决于 $H$,因为$N \approx H$。%
参考点的位置同样取决于用户对所获得的解决方案中的偏好信息。

\section{本章小结}
本章对田间试验设计及数据获取进行了简单的介绍,%
试验站位于新疆生产建设兵团第一师阿拉尔市灌溉试验站内,%
田间试验数据主要用于模型的校准与验证。%
对模型模拟所需要的技术进行了简单的叙述,%
然后详细介绍了敏感性分析和多目标优化的方法。

  \chapter{Cotton2K 模型的改进}\label{chap:modelModification}

\section{模型输入输出}\label{sec:io}

原版的 Cotton2K 模型输入输出是基于特定格式的文件的。考虑到模型的建立处于关系型数据库还未流行的年代,这种基于特定格式文件的输入输出方式是可以理解的。
但这样的输入输出方式不利于 Cotton2K 结合到以 Python 为核心的现代化机器学习的生态中。
所以本文对 Cotton2K 进行了重构,修改输出输出为结构化数据,方便与其他机器学习工具结合。
%具体的输入格式以 JSON Schema 的格式定义,参见附录\ref{appendix:inputSchema},在此不再赘述。

输出格式修改为 CSV 格式,改变模型代码以直接输出与土钻取土位置相对应的土壤含水量,全部字段见表 \ref{tab:output}。

\begin{table}
    \caption{修改后的 Cotton2K 模型输出}\label{tab:output}
    \centering
    \begin{tabular}{lll}
        \toprule
        字段名称                          & 含义           & 单位      \\
        \midrule
        \texttt{date}                     & 日期           & 天        \\
        \texttt{light\_interception}      & 光能截获率     & \%        \\
        \texttt{plant\_height}            & 株高           & cm        \\
        \texttt{leaf\_area\_index}        & 叶面积指数     & 1         \\
        \texttt{lai00} 到 \texttt{lai19}  & 分层叶面积指数 & 1         \\
        \texttt{lint\_yield}              & 皮棉产量       & kg/hm$^2$ \\
        \texttt{seed\_cotton\_yield}      & 籽棉产量       & kg/hm$^2$ \\
        \texttt{leaf\_weight}             & 叶干物质量     & kg/hm$^2$ \\
        \texttt{petiole\_weight}          & 叶柄干物质量   & kg/hm$^2$ \\
        \texttt{stem\_weight}             & 茎干物质量     & kg/hm$^2$ \\
        \texttt{square\_weight}           & 方铃干物质量   & kg/hm$^2$ \\
        \texttt{boll\_weight}             & 铃干物质量     & kg/hm$^2$ \\
        \texttt{root\_weight}             & 根干物质量     & kg/hm$^2$ \\
        \texttt{plant\_weight}            & 地上总干物质量 & kg/hm$^2$ \\
        \texttt{main\_stem\_nodes}        & 主茎节点数     & 个        \\
        \texttt{number\_of\_squares}      & 方铃个数       & 个        \\
        \texttt{number\_of\_green\_bolls} & 绿铃个数       & 个        \\
        \texttt{number\_of\_open\_bolls}  & 开铃个数       & 个        \\
        \texttt{swc}                      & 土壤含水量     & \%        \\
        \bottomrule
    \end{tabular}
\end{table}

\section{修正部分计算}
一些参数集在运行时引起了下溢或溢出错误,这就需要对编码进行编辑,以限制某些状态变量的范围。
例如,在 Cotton2K 中新叶生长的过程中,新叶的质量是固定的,且是由茎的质量转移而来,在极端的情况下,叶片生长速度过快会导致茎干重变为负数。

\section{跨操作系统编译}
\authoryearcite{thorp2019} 使用了 PALMScot 景观尺度棉花建模工具开发者维护的 Fortran 版本的 Cotton2K,主要是因为新版的 Cotton2K 高度依赖%
Microsoft Windows 平台的 GUI 框架 MFC。本文作者通过将 Cotton2K 完全使用 Python 重构,成功解决了上述与 Microsoft Windows 平台绑定的问题。

\section{全新的冠层子模块}\label{sec:canopyLayering}
Cotton2K 模型中的光能截获率计算是从 GOSSYM 模型中演进而来。
计算光能截获率需要先计算两个因子:\begin{enumerate*}
    \item $z$ 因子与
    \item $l$ 因子
\end{enumerate*}。

$z$ 因子与株高和行间距比值成正比,参见公式 \ref{eq:z}。

\begin{equation}\label{eq:z}
    z = 1.0756 * H / ROWSPC
\end{equation}
式中 $H$ 为株高,单位是 cm, $ROWSPC$ 是行间距,单位是 cm。

$l$ 因子在叶面积小于 0.5 时时一个关于叶面积的线性函数,在其他情况时是一个指数函数。参见公式 \ref{eq:l}。

\begin{equation}\label{eq:l}
    l = \begin{cases}
        0.8 * LAI                 & LAI \le 0.5 \\
        1 - e^{0.07 - 1.16 * LAI} & LAI > 0.5
    \end{cases}
\end{equation}
式中 $LAI$ 为叶面积指数。

当 $l$ 因子大于 $z$ 因子时,光能截获率是两者的平均值,在 $l$ 因子小于 $z$ 因子且当前叶面积指数小于最大叶面积指数
(通常是发生了叶片的脱落) 时,为 $l$ 因子。在其他情况下为 $z$ 因子。参见公式 \ref{eq:li}。

\begin{equation}\label{eq:li}
    LI = \begin{cases}
        \frac{l + z}{2} & l > z                    \\
        l               & l\le z \, LAI<LAI_{\max} \\
        z               & \text{其他情况}
    \end{cases}
\end{equation}

受 WOFOST-GTC\cite{WOFOSTGTC} 启发,对模型进行修改,冠层的生长过程自顶向下,分层模拟。
每层层高 5 cm,共 20 层,模拟冠层高度最高可达 1 米。
叶片按展开时间和几何拓扑关系被分配到特定的层级中。
自顶向下逐层进行光合作用过程的模拟。
最终,光合作用产物的量为各层产物的量之和。

\begin{equation}
    LI_i = 1 - e^{p_i * LAI_i}
\end{equation}

\begin{equation}
    LI = 1 - \prod^{n}_{i=1}e^{p_i * LAI_i}
\end{equation}
式中 $LI$ 是总光能截获率, $n$ 总冠层层数, $LI_i$ 是第 $i$ 层的光能截获率,$p_i$ 是第 $i$ 层的光能截获率公式参数,
$LAI_i$ 是第 $i$ 层的叶面积指数。

\begin{equation}%
    P_{std} = 2.3908 + 1.37379 W + (-0.00054136) W^2%
\end{equation}%
式中 $P_{std}$ 是理论光合作用产物的量, $W$ 是地球接受到的辐射总量。%

\begin{equation}%
    P_{plant} = \sum^n_{i=1} P_{std} \times LI_i \times S_{plant} \times P_{tsred} \times P_{netcor} \times P_{tnfac} \times P_{agei}%
\end{equation}%
式中 $P_{plant}$ 是实际光合作用产物的量, $S_{plant}$ 每株棉花占地面积,单位是 $\mathrm{dm^2}$,%
$P_{tsred}$ 是水分胁迫对光合作用效率的影响,%
$P_{netcor}$ 是空气中 $\mathrm{CO_2}$ 含量对总光合作用效率的校正因子,%
$P_{tnfac}$ 低叶片含氮量下的校正因子,%
$P_{agei}$ 叶龄校正因子。

\section{本章小结}
本章首先对获取的 Cotton2K 模型源码进行了修改,改进了模型的输入与输出格式,修正了部分计算问题,移除了与 Windows 平台的绑定,最终增加了一个更为详细的分层模拟的冠层子模块。

  \chapter{模型敏感性分析}\label{chap:sa}
为了深入了解Cotton2K对模型输入参数调整的反应,使用Python中SALib包的算法进行了Sobol GSA。%
精明的读者会注意到,通常先进行计算密集度较低的敏感性分析,然后再进行像Sobol GSA这样的密集型方法。%
在这里,只使用了Sobol GSA,因为\begin{enumerate*}
    \item 高性能计算资源的可用性确保了模拟的效率,
    \item 工作流程的后续部分也纳入了Sobol GSA的Sobol采样方面
\end{enumerate*} 。

使用SALib软件包的算法,对 70 个输入参数和 12 个农业生态系统指标的每个组合计算了一阶、二阶和总的敏感性指数。%
每个农业生态系统指标的 RMSE 统计量被用作敏感性指数计算的目标函数。%
任何对 12 个农业生态系统指标中的任何一个具有一阶敏感性指数大于 0.05 的参数都被认为是有影响的参数,%
并在随后的分析中保持灵活性 (表 \ref{tab:parameters})。%
其他参数被认为是非影响性的,在剩余的分析中被固定为默认值。%
Sobol GSA的总体目的是在使用多目标优化进行模型校准之前消除非影响性参数。
\section{参数选择及取值范围}
在原版 Cotton2K 模型中,所有参数都是以可更改的外部文件形式导入模型。%
对于 Cotton2K 模型中 50 个品种参数而言,并非所有的参数对于模型输出结果都具有影响,需要通过敏感性分析选取其中敏感性较大的参数进行下一步校正。%
本文根据公开的源码和相关参数对应公式,对品种参数取值范围进行了合理化的约束。%
在修改后的 Cotton2K 模型中品种参数文件以 \texttt{cultivar\_parameters} 字段保存,模型参数服从均匀分布。%
本文主要对 50 个原始品种参数和 20 个修改模型参数进行敏感性分析,原始模型参数的名称及取值范围如表 \ref{tab:saParameters} 所示。%
根据第 \ref{chap:modelModification} 章的内容,本文还对模型进行了修改,增加了更为详细的分层的冠层子模块,因为这个改动,改版的 Cotton2K 模型比原版的模型多了 20 个参数,%
为 LIPAR01 至 LIPAR20,取值范围根据 GOSSYM 模型和 WOFOST 模型中相关的参数范围扩大 100\% 而得。
\begin{table}
    \caption{原版 Cotton2K 参数列表及取值范围}\label{tab:saParameters}
    \small
    \begin{tabular}{llrr|llrr}
        \toprule
        参数     & 功能描述             & 下限  & 上限  & 参数     & 功能描述                   & 下限   & 上限   \\
        \midrule
        VARPAR01 & 种植密度对生长的影响 & 0.00  & 0.08  & VARPAR26 & 株高生长                   & 0.70   & 1.30   \\
        VARPAR02 & 现蕾前的叶生长       & 0.00  & 0.60  & VARPAR27 & 碳胁迫下的主茎节点延迟     & 0.50   & 1.00   \\
        VARPAR03 & 现蕾前的叶生长       & 0.00  & 0.10  & VARPAR28 & 碳胁迫下的坐果节点延迟     & 1.00   & 3.00   \\
        VARPAR04 & 现蕾前的叶生长       & 0.00  & 1.00  & VARPAR29 & 碳胁迫下的坐果节点延迟     & 1.00   & 3.00   \\
        VARPAR05 & 主茎叶生长           & 0.50  & 3.00  & VARPAR30 & 温度对方铃形成的影响       & 0.70   & 1.30   \\
        VARPAR06 & 主茎叶生长           & 0.00  & 0.02  & VARPAR31 & 现蕾前节点的发育           & 1.00   & 5.00   \\
        VARPAR07 & 主茎叶生长           & 18.0  & 28.0  & VARPAR32 & 现蕾前节点的发育           & 1.00   & 3.00   \\
        VARPAR08 & 果汁页生长           & 0.00  & 0.20  & VARPAR33 & 现蕾前节点的发育           & 0.50   & 1.00   \\
        VARPAR09 & 铃生长周期           & 18.9  & 38.0  & VARPAR34 & 初始叶面积                 & 0.01   & 0.10   \\
        VARPAR10 & 铃生长速率           & 0.10  & 0.45  & VARPAR35 & 果枝发育                   & -32.0  & -26.0  \\
        VARPAR11 & 最大铃干物质         & 3.00  & 15.0  & VARPAR36 & 坐果节点的发育             & -57.0  & -49.0  \\
        VARPAR12 & 方铃展前茎生长       & 0.00  & 2.00  & VARPAR37 & 坐果节点的发育             & 0.00   & 2.00   \\
        VARPAR13 & 方铃展前茎生长       & 0.00  & 2.00  & VARPAR38 & 落叶剂对不同叶龄叶片的影响 & 0.00   & 5.00   \\
        VARPAR14 & 方铃展前茎生长       & 0.00  & 0.08  & VARPAR39 & 温度对吐絮的影响           & -306.0 & -240.0 \\
        VARPAR15 & 方铃展后茎生长       & 1.00  & 4.00  & VARPAR40 & 温度对吐絮的影响           & 0.70   & 1.30   \\
        VARPAR16 & 方铃展后茎生长       & 0.50  & 2.00  & VARPAR41 & 温度对衣份率的影响         & 45.0   & 60.0   \\
        VARPAR17 & 方铃展后茎生长       & 0.00  & 1.00  & VARPAR42 & 温度对衣份率的影响         & 0.30   & 0.90   \\
        VARPAR18 & 方铃展后茎生长       & 0.00  & 0.20  & VARPAR43 & 碳胁迫下的脱落强度         & 0.30   & 0.70   \\
        VARPAR19 & 株高生长             & 0.00  & 0.50  & VARPAR44 & 水分胁迫下的脱落强度       & 0.30   & 0.70   \\
        VARPAR20 & 株高生长             & 0.00  & 0.10  & VARPAR45 & 方铃脱落的概率             & 0.10   & 0.50   \\
        VARPAR21 & 株高生长             & 10.0  & 18.0  & VARPAR46 & 方铃脱落的概率             & 0.00   & 0.15   \\
        VARPAR22 & 株高生长             & -4.00 & -2.00 & VARPAR47 & 铃脱落的概率               & 1.00   & 10.0   \\
        VARPAR23 & 株高生长             & 0.09  & 0.12  & VARPAR48 & 铃脱落的概率               & 0.30   & 1.50   \\
        VARPAR24 & 株高生长             & 0.00  & 0.50  & VARPAR49 & 铃脱落的概率               & 5.00   & 15.0   \\
        VARPAR25 & 株高生长             & 1.00  & 4.00  & VARPAR50 & 铃脱落的概率               & 0.20   & 1.80   \\
        \bottomrule
    \end{tabular}
\end{table}
\section{研究方法}

\subsection{Cotton2K 模型的运行}
Cotton2K 模型需要输入的参数较多,模型结构较为复杂,在第 \ref{sec:io} 节中论述,本文对模型的输入输出格式进行了修改简化,提高了模型模拟效率。

模型的需要如下的输入参数
\begin{enumerate}
    \item 关键农艺措施的时间,如播种日期等;
    \item 试验站的地理信息,如经纬度、海拔等;
    \item 土壤理化性质,包括砂份、壤份、土壤水分传导率、 Penman 公式参数等;
    \item 气象数据,包括 2m 最高和最低气温、降水、风速、太阳净辐射强度等;
    \item 灌溉和施肥的相关数据,包括日期、用量、是否随水施肥等;
    \item 棉花品种参数,如表 \ref{tab:saParameters} 所示。
\end{enumerate}

\section{本章小结}

  \chapter{模型校正及验证}

\section{研究方法}
\begin{table}
    \caption{对中国新疆阿拉尔市田间12种棉花管理方案的农业生态系统指标进行评估。
        这12个指标按优先顺序列出,以便使用多目标优化技术对Cotton2K农业生态系统模型进行评价。}
    \centering
    \begin{tabular}{llll}
        \toprule
        指标 & 描述           & 单位                   & 数量 \\
        \midrule
        LY   & 皮棉产量       & $\mathrm{kg\ ha^{-1}}$ & 24   \\
        TAGB & 地上总干物质量 & $\mathrm{kg\ ha^{-1}}$ & 240  \\
        LAI  & 叶面积指数     & $\mathrm{m^2\ m^{-2}}$ & 240  \\
        LI   & 光能截获率     & \%                     & 240  \\
        SCY  & 籽棉产量       & $\mathrm{m^2\ m^{-2}}$ & 24   \\
        PH   & 株高           & cm                     & 240  \\
        MSN  & 主茎节点数     & 个                     & 240  \\
        LDM  & 叶干物质量     & $\mathrm{m^2\ m^{-2}}$ & 240  \\
        PDM  & 叶柄干物质量   & $\mathrm{m^2\ m^{-2}}$ & 240  \\
        SDM  & 茎干物质量     & $\mathrm{m^2\ m^{-2}}$ & 240  \\
        BLN  & 铃数           & 个                     & 240  \\
        BDM  & 铃干物质量     & $\mathrm{m^2\ m^{-2}}$ & 240  \\
        \bottomrule
    \end{tabular}
\end{table}

\begin{table}
    \caption{实测值与模拟值对比统计参数计算公式及描述}
    \begin{tabular}{p{0.15\linewidth}p{0.7\linewidth}p{0.15\linewidth}}
        \toprule
        参数     & 计算公式                                                                                                                                                                      & 描述         \\
        \midrule
        $b$      & \[\frac{\sum_{i=1}^n (O_i - \overline{O}) (P_i - \overline{P})}{\sum_{i=1}^n (O_i - \overline{O})}\]                                                                          & 回归系数     \\
        $R^2$    & \[\left \{ \frac{\sum_{i=1}^n (O_i - \overline{O}) (P_i - \overline{P})}{\sqrt{\sum_{i=1}^n (O_i - \overline{O})^2} \sqrt{\sum_{i=1}^n (P_i - \overline{P})^2}} \right \}^2\] & 决定系数     \\
        RMSE     & \[\sqrt{\frac{\sum_{i=1}^n (O_i - P_i)^2}{n}}\]                                                                                                                               & 均方根误差   \\
        AAE      & \[\frac{\sum_{i=1}^n |O_i - P_i|}{n}\]                                                                                                                                        & 平均绝对误差 \\
        EF       & \[1 - \frac{\sum_{i=1}^n (O_i - P_i)^2}{\sum_{i=1}^n (O_i - \overline{O})^2}\]                                                                                                & 模型有效性   \\
        $d_{IA}$ & \[1 - \frac{\sum_{i=1}^n (O_i - P_i)^2}{\sum_{i=1}^n (|P_i - \overline{O}| + |O_i - \overline{O}|^2)}\]                                                                       & 拟合度       \\
        \bottomrule
    \end{tabular}
\end{table}
\subsection{Python 中的多目标优化}
本章采用 pymoo 框架提供的非支配排序遗传算法 (NSGA-III) 多目标优化算法确定参数最优值。该框架由美国密歇根州立大学的 Kalaynmoy Deb 教授带领的计算优化与创新实验室 (COIN) 的 Julian Blank 开发维护的。%
在大多数情况下,pymoo 框架被用于解决连续问题, 但其他变量类型也可使用。%
pymoo 框架中的算法 (如NSGA-III算法) 是模块化组织的,用户可以通过编写代码自定义采样、交叉和突变等模块,自行组装符合自身需求的算法。%

\subsection{参数优化步骤}
根据第 \ref{chap:sa} 章模型敏感性参数分析结果,选择 35 个对模拟结果影响较大的参数。
然后再次使用 Sobol' 方法选择 Cotton2K 模型参数集,但仅在模型敏感性参数分析出的影响较大的参数中进行,第二次选择的参数集没有进行敏感性分析。%
相反,第二次 Monte Carlo 方法采样后的模拟结果被用于模型参数优化,以识别在测量结果和模拟结 果之间提供最佳一致性的参数集。%
具体步骤如下:
\begin{enumerate}
    \item 选择第 \ref{chap:sa} 章对模型结果影响较大的 35 个品种,确定各参数的取值范围和均匀分布形式。
    \item 采用 Python 语言编写程序,生成 Cotton2K 的输入参数,以及模型输出的损失函数。
    \item 运行 NSGA-III 算法调用 Cotton2K 模型,并将结果导入到数据库。
    \item 采用剪枝算法对 Pareto 解集的结果进行剪枝。
    \item 根据剪枝后的 Pareto 解集确定最终模型敏感性参数最优值。
\end{enumerate}
\begin{longtable}{llrrcrr}
    \caption{Cotton2K 参数列表及率定值}\label{tab:parameters}                         \\
    \toprule
    参数     & 功能描述                   & 下限   & 上限   & GSA & 原版    & 改版    \\
    \midrule\endfirsthead
    \caption*{续表\ref{tab:parameters}}                                               \\
    \toprule
    参数     & 功能描述                   & 下限   & 上限   & GSA & 原版    & 改版    \\
    \midrule
    \endhead
    \bottomrule
    \multicolumn{7}{r}{\textit{接下页}}                                               \\
    \endfoot
    \bottomrule
    \endlastfoot
    VARPAR01 & 种植密度对生长的影响       & 0.00   & 0.08   & *   & 0.04344 & 0.03416 \\
    VARPAR02 & 现蕾前的叶生长             & 0.00   & 0.60   &     & 0.3     & 0.3     \\
    VARPAR03 & 现蕾前的叶生长             & 0.00   & 0.10   &     & 0.014   & 0.014   \\
    VARPAR04 & 现蕾前的叶生长             & 0.00   & 1.00   & *   & 0.138   & 0.646   \\
    VARPAR05 & 主茎叶生长                 & 0.50   & 3.00   &     & 1.6     & 1.6     \\
    VARPAR06 & 主茎叶生长                 & 0.00   & 0.02   &     & 0.010   & 0.010   \\
    VARPAR07 & 主茎叶生长                 & 18.0   & 28.0   &     & 24.0    & 24.0    \\
    VARPAR08 & 果汁页生长                 & 0.00   & 0.20   &     & 0.10    & 0.10    \\
    VARPAR09 & 铃生长周期                 & 18.9   & 38.0   &     & 28.0    & 28.0    \\
    VARPAR10 & 铃生长速率                 & 0.10   & 0.45   &     & 0.3293  & 0.3293  \\
    VARPAR11 & 最大铃干物质               & 3.00   & 15.0   &     & 8.8     & 8.8     \\
    VARPAR12 & 方铃展前茎生长             & 0.00   & 2.00   & *   & 0.284   & 0.590   \\
    VARPAR13 & 方铃展前茎生长             & 0.00   & 2.00   &     & 0.040   & 0.040   \\
    VARPAR14 & 方铃展前茎生长             & 0.00   & 0.08   &     & 0.014   & 0.014   \\
    VARPAR15 & 方铃展后茎生长             & 1.00   & 4.00   & *   & 3.971   & 2.001   \\
    VARPAR16 & 方铃展后茎生长             & 0.50   & 2.00   &     & 2.4     & 2.4     \\
    VARPAR17 & 方铃展后茎生长             & 0.00   & 1.00   &     & 0.10    & 0.10    \\
    VARPAR18 & 方铃展后茎生长             & 0.00   & 0.20   &     & 0.140   & 0.140   \\
    VARPAR19 & 株高生长                   & 0.00   & 0.50   &     & 0.20    & 0.20    \\
    VARPAR20 & 株高生长                   & 0.00   & 0.10   &     & 0.02    & 0.02    \\
    VARPAR21 & 株高生长                   & 10.0   & 18.0   & *   & 14.541  & 13.868  \\
    VARPAR22 & 株高生长                   & -4.00  & -2.00  & *   & -2.686  & -2.077  \\
    VARPAR23 & 株高生长                   & 0.09   & 0.12   &     & 0.10    & 0.10    \\
    VARPAR24 & 株高生长                   & 0.00   & 0.50   &     & 0.175   & 0.175   \\
    VARPAR25 & 株高生长                   & 1.00   & 4.00   &     & 2.20    & 2.20    \\
    VARPAR26 & 株高生长                   & 0.70   & 1.30   & *   & 1.2500  & 0.7500  \\
    VARPAR27 & 碳胁迫下的主茎节点延迟     & 0.50   & 1.00   & *   & 0.731   & 0.743   \\
    VARPAR28 & 碳胁迫下的坐果节点延迟     & 1.00   & 3.00   &     & 2.15    & 2.15    \\
    VARPAR29 & 碳胁迫下的坐果节点延迟     & 1.00   & 3.00   &     & 1.36    & 1.36    \\
    VARPAR30 & 温度对方铃形成的影响       & 0.70   & 1.30   & *   & 1.281   & 1.249   \\
    VARPAR31 & 现蕾前节点的发育           & 1.00   & 5.00   & *   & 3.210   & 2.175   \\
    VARPAR32 & 现蕾前节点的发育           & 1.00   & 3.00   & *   & 1.625   & 1.664   \\
    VARPAR33 & 现蕾前节点的发育           & 0.50   & 1.00   &     & 0.80    & 0.80    \\
    VARPAR34 & 初始叶面积                 & 0.01   & 0.10   & *   & 0.0407  & 0.0822  \\
    VARPAR35 & 果枝发育                   & -32.0  & -26.0  & *   & -26.118 & -27.108 \\
    VARPAR36 & 坐果节点的发育             & -57.0  & -49.0  &     & -54.00  & -54.00  \\
    VARPAR37 & 坐果节点的发育             & 0.00   & 2.00   &     & 0.80    & 0.80    \\
    VARPAR38 & 落叶剂对不同叶龄叶片的影响 & 0.00   & 5.00   &     & 3.20    & 3.20    \\
    VARPAR39 & 温度对吐絮的影响           & -306.0 & -240.0 &     & -292.0  & -292.0  \\
    VARPAR40 & 温度对吐絮的影响           & 0.70   & 1.30   &     & 1.08    & 1.08    \\
    VARPAR41 & 温度对衣份率的影响         & 45.0   & 60.0   & *   & 51.813  & 47.436  \\
    VARPAR42 & 温度对衣份率的影响         & 0.30   & 0.90   & *   & 0.844   & 0.341   \\
    VARPAR43 & 碳胁迫下的脱落强度         & 0.30   & 0.70   & *   & 0.514   & 0.601   \\
    VARPAR44 & 水分胁迫下的脱落强度       & 0.30   & 0.70   &     & 0.48    & 0.48    \\
    VARPAR45 & 方铃脱落的概率             & 0.10   & 0.50   &     & 0.24    & 0.24    \\
    VARPAR46 & 方铃脱落的概率             & 0.00   & 0.15   &     & 0.08    & 0.08    \\
    VARPAR47 & 铃脱落的概率               & 1.00   & 10.0   & *   & 3.833   & 6.439   \\
    VARPAR48 & 铃脱落的概率               & 0.30   & 1.50   & *   & 0.532   & 0.554   \\
    VARPAR49 & 铃脱落的概率               & 5.00   & 15.0   & *   & 7.714   & 6.867   \\
    VARPAR50 & 铃脱落的概率               & 0.20   & 1.80   & *   & 1.610   & 1.216   \\
    LIPAR01  & 光能截获参数               & -1.60  & 0.00   & *   & -       & -3.347  \\
    LIPAR02  & 光能截获参数               & -1.60  & 0.00   & *   & -       & -2.199  \\
    LIPAR03  & 光能截获参数               & -1.60  & 0.00   & *   & -       & -2.788  \\
    LIPAR04  & 光能截获参数               & -1.60  & 0.00   & *   & -       & -1.262  \\
    LIPAR05  & 光能截获参数               & -1.60  & 0.00   & *   & -       & -0.253  \\
    LIPAR06  & 光能截获参数               & -1.60  & 0.00   & *   & -       & -1.602  \\
    LIPAR07  & 光能截获参数               & -1.60  & 0.00   & *   & -       & -1.251  \\
    LIPAR08  & 光能截获参数               & -1.60  & 0.00   & *   & -       & -3.464  \\
    LIPAR09  & 光能截获参数               & -1.60  & 0.00   & *   & -       & -0.207  \\
    LIPAR10  & 光能截获参数               & -1.60  & 0.00   & *   & -       & -0.776  \\
    LIPAR11  & 光能截获参数               & -1.60  & 0.00   & *   & -       & -0.758  \\
    LIPAR12  & 光能截获参数               & -1.60  & 0.00   & *   & -       & -1.110  \\
    LIPAR13  & 光能截获参数               & -1.60  & 0.00   & *   & -       & -2.517  \\
    LIPAR14  & 光能截获参数               & -1.60  & 0.00   & *   & -       & -1.871  \\
    LIPAR15  & 光能截获参数               & -1.60  & 0.00   & *   & -       & -2.984  \\
    LIPAR16  & 光能截获参数               & -1.60  & 0.00   & *   & -       & -0.166  \\
    LIPAR17  & 光能截获参数               & -1.60  & 0.00   & *   & -       & -3.303  \\
    LIPAR18  & 光能截获参数               & -1.60  & 0.00   & *   & -       & -1.743  \\
    LIPAR19  & 光能截获参数               & -1.60  & 0.00   & *   & -       & -1.504  \\
    LIPAR20  & 光能截获参数               & -1.60  & 0.00   & *   & -       & -2.155  \\
\end{longtable}

  \chapter{结论}
\section{结论}

本文 Cotton2K 源码被修改加入更细节的分层冠层子模块,经实验验证,这样的修改有助于提高模型的模拟精度。%
与其他模型,如 CSM-Cropgro-Cotton 和 OZCOT 相比,Cotton2K模型的模拟精度仍有很大的提升空间。%
造成这样的原因可能有很多,例如模型的复杂性,具体内容仍有待进一步研究。%
模拟结果显示,地上总生物量和籽棉产量随着灌溉量的增加而增加,%
试验中实测地上总生物量随灌水量增长的趋势与模拟一致,而籽棉产量在灌水增加到。%
这些模拟结果可作为南疆无膜栽培棉花的种植的参考数据。

\section{创新点}
本文详细描述了棉花生长模型 Cotton2K,修改了模型的源码以修复一些问题,增加更为详细的冠层子模块,%
以 2019 年至 2020 年的田间试验为基础,先通过全局敏感性分析方法对模型的品种参数进行分析,%
综合考量选取其中部分品种参数采用 NSGA-III 多目标优化算法进行优化,%
对模拟结果与实测数据进行分析,进而评估了模型的修改对研究区域棉花的模拟精度的影响。%
总的来说,在该模型对棉花生育早期特别是现蕾以前模拟的精度很高,随着器官的分化,%
棉花的生长发育机理变得越来越复杂,模拟的精度有所下降。%
但对于这样一个复杂的过程,模型在经过冠层光能截获率的修改后所能达到的结果是可以接受的。%

\section{存在的问题与下一步工作}

本文仍存在一些不足:
\begin{enumerate*}
    \item 未考虑无膜棉与有膜棉最大区别的土壤水分过程
    \item 未检验模型对除阿拉尔地区以外的其他地区的有效性
\end{enumerate*}。下一步需要增加其他研究区域的田间数据,继续对模型进一步修改,使其可以更细致地模拟土壤水分运移过程。
  \printbibliography
  \chapter*{致谢}


  \chapter*{作者简介}

  唐梓涯,男,汉族,出生于 1992 年 3 月 13 日,江苏省如东县人。2010 年 9 月,就读于西北农林科技大学植物保护学院,植物保护专业。2014 年 6 月获得农学学士学位。2019 年 9 月,考入塔里木大学信息工程学院,农业电气化与自动化专业,攻读工学硕士学位。

  公开发表的论文:

  [1] 孟文博,王德胜,张楠楠,费浩,唐梓涯,王涛,白铁成.响应气候变化的棉花生长模拟与县域尺度产量评估[J].核农学报,2021,35(07):1648-1657.

  获得的奖励
  \begin{enumerate}
    \item 2019 年“华为杯”第十六届数学建模竞赛三等奖。
    \item 2019{-}2020 学年研究生学业奖学金二等奖。
    \item 2020{-}2021 学年研究生学业奖学金三等奖。
    \item 2021 年“华为杯”第十八届数学建模竞赛三等奖。
    \item 2021 年第十二届蓝桥杯全国软件和信息技术专业人才大赛 Python 组全国总决赛三等奖。
  \end{enumerate}

  %\appendix
  %\chapter{Cotton2K 输入格式}\label{appendix:inputSchema}
\lstinputlisting[
    style       =   JSONSchema
]{./cotton2k.schema.json}
\end{spacing}
\end{document}